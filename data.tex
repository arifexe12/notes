% Created 2022-11-04 Fri 13:15
% Intended LaTeX compiler: pdflatex
\documentclass[11pt]{article}
\usepackage[utf8]{inputenc}
\usepackage[T1]{fontenc}
\usepackage{graphicx}
\usepackage{longtable}
\usepackage{wrapfig}
\usepackage{rotating}
\usepackage[normalem]{ulem}
\usepackage{amsmath}
\usepackage{amssymb}
\usepackage{capt-of}
\usepackage{hyperref}
\author{Space-X}
\date{\today}
\title{}
\hypersetup{
 pdfauthor={Space-X},
 pdftitle={},
 pdfkeywords={},
 pdfsubject={},
 pdfcreator={Emacs 28.2 (Org mode 9.5.5)}, 
 pdflang={English}}
\begin{document}

\tableofcontents

\section{সাম্প্রতিক তথ্য}
\label{sec:org7639eff}

\begin{enumerate}
\item মোহাম্মদ বিন নায়েফকে ক্রাউন প্রিন্সের পদ থেকে সরিয়ে নিজে ক্ষমতা নেন মোহাম্মদ বিন সালমান ।
\item ১৯৬০ সালে ওপেক এবং ২০১৬ সালে ওপেক প্লাস গঠিত হয় ।
\item ওপেক প্লাসের সদস্য সংখ্যা ২৩ ।
\item ইসরায়েলের প্রতিরক্ষামন্ত্রী বেনি গ্যান্টজ ।
\item পাকিস্তানের অর্থমন্ত্রী ইসহাক দার  ।
\item সর্বশেষ বাংলাদেশে  আঘাত হানা ঘূর্ণিঝড় হল  সিত্রাং ।
\item মধ্যপ্রাচ্যে যুক্তরাষ্ট্রের সবচেয়ে বড় বিমানঘাঁটি এই কাতারেই।
\item কাতার দেশটিতে সামরিক ঘাঁটি আছে তুরস্কেরও।
\item ১৯৯৬ সালে বিশ্বমানের সংবাদমাধ্যম হিসেবে কাতারে শুরু হয় আল-জাজিরার পথচলা।
\item ২০১৭ সালে মিত্রদের নিয়ে কাতারের ওপর সর্বাত্মক অবরোধ আরোপ করে সৌদি। অবরোধ তুলে নিতে দেওয়া ১৩টি শর্তের অন্যতম ছিল আল-জাজিরা বন্ধ করে দেওয়া।
\item ২০১৭ সালের ৫ জুন সৌদি ও তার আরব মিত্ররা কাতারের ওপর সর্বাত্মক অবরোধ আরোপ করে।
\item ২০২১ সালের জানুয়ারিতে অবরোধ তুলে নেয় সৌদি জোট।
\item দেশের একমাত্র নারী জেলা পরিষদ চেয়ারম্যান সালমা রহমান পিরোজপুর জেলা পরিষদের চেয়ারম্যান
\item পাকিস্তানের প্রেসিডেন্ট \textbf{আরিফ আলভি}
\item পাঞ্জাবের ওয়াজিরাবাদ শহরের আল্লাহওয়ালা চক এলাকার কাছে ইমরান খানের ওপর এ হামলা হয়
\item কুইক রেন্টালের জন্য বিদ্যুৎ ও জ্বালানির দ্রুত সরবরাহ বৃদ্ধি (বিশেষ বিধান) আইন ২০১০ করা হয় ।
\item 
\end{enumerate}



\section{রাশিয়া}
\label{sec:org395506a}
\subsection{রাশিয়ার স্বরাষ্ট্রমন্ত্রীর নাম কি ?}
\label{sec:org15390d3}
\subsubsection{সের্গেই কুজুগেটোভিচ শোইগুর ।}
\label{sec:orgbcb6406}
\begin{enumerate}
\item রুশ প্রধানমন্ত্রী মিখাইল মিশুস্তিন
\item খেরসনের রাশিয়া কতৃক নিয়োগকৃত কর্মকর্তা ভ্লাদিমির সালদো ।
\item রাশিয়ার রাষ্ট্রীয় পারমাণবিক শক্তি করপোরেশনের (রোসাটম) মহাপরিচালক অ্যালেক্সি লিখাচেভ ।
\end{enumerate}


\section{নোবেল পুরস্কার}
\label{sec:org7099dfc}

\begin{center}
\begin{tabular}{llllll}
চিকিৎসা & সাহিত্য & শান্তি & অর্থনীতি & রসায়ন & পদার্থ\\
\hline
সোয়ান্তে প্যাবো & এনি এরনো & এ্যালেস বিয়ালিয়াৎস্কি & বেন এস বার্নানকে & ক্যারোলিন আর বার্তোজ্জি & এ্যালাইন এসপেক্ট\\
\hline
 &  & মেমরিয়াল & ডগলাস ডাব্লিউ ডাইমন্ড & কে ব্যারি শার্পলেশ & জন এফ ক্লোজার\\
\hline
 &  & সেন্টার ফর সিভিল লিবার্টিস & ফিলিপ এইচ ডিবভিগ & মর্টেন মেলডাল & অ্যান্টন জেলিঙ্গার\\
 &  &  &  &  & \\
\hline
\end{tabular}
\end{center}

\section{ইউক্রেন}
\label{sec:org6c90635}
\begin{enumerate}
\item কিয়েভের মেয়রের নাম  ভিটালি ক্লিসকি ।
\item ইউক্রেনের প্রধানমন্ত্রী ডেনিস স্মিগাল ।
\item মিকোলাইভ শহরটির মেয়র ওলেক্সান্দার সেনকেভিচ ।
\item ইউক্রেনের পররাষ্ট্রমন্ত্রী দিমিত্রি কুলেবো ।
\end{enumerate}

\section{দিনাজপুর}
\label{sec:org59a5742}

\subsection{মুক্তিযুদ্ধে দিনাজপুর}
\label{sec:orgd7c1ac4}

\subsubsection{দিনাজপুরের খেতাব প্রাপ্ত মুক্তিযোদ্ধা}
\label{sec:org5f1b946}

\begin{enumerate}
\item ক্যাপ্টেন মাহবুবর রহমান বীর উত্তম \textbf{(২ নং সেক্টর )}
\item সিপাহি আব্দুল মজিদ-বীর বিক্রম \textbf{(৬ নং সেক্টর)}
\item মোঃ মুহসিন আলী সরদার -বীর প্রতীক \textbf{(৭ নং সেক্টর)}
\end{enumerate}

\subsubsection{দিনাজপুরের রাজাকার}
\label{sec:org31edeec}

\begin{enumerate}
\item রাজাকার মির্জা রুহুল আমিন
\item রাজাকার নুরুল হক চৌধুরী
\item রাজাকার মওলানা তমিজউদ্দীন
\item রাজাকার ওমর আলী
\item রাজাকার ডা. ইসমাইল
\item রাজাকার আলী
\item রাজাকার খবিরদ্দীন
\item রাজাকার জালাল উকিল
\item রাজাকার শামসুল মাস্টার
\item রাজাকার ইদ্রিস আলী
\item রাজাকার পজির মেম্বার
\item রাজাকার আকবর মেম্বার
\item রাজাকার মোঃ রউফ মিয়া
\item রাজাকার আঃ বারেক
\item রাজাকার ধদা মাহালিয়া
\end{enumerate}

\section{ইংল্যান্ড}
\label{sec:org4e31ac0}

\begin{enumerate}
\item ইংল্যান্ডের প্রধানমন্ত্রী লিজ ট্রাস।(পদত্যাগ করেছেন)
\item ইংল্যান্ডের অর্থমন্ত্রী জেরমি হ্যান্ট ।
\end{enumerate}

\section{ভারতের সঙ্গে ৭ সমঝোতা স্মারক}
\label{sec:org285da40}
\begin{enumerate}
\item সুরমা-কুশিয়ারা প্রকল্পের অধীনে কুশিয়ার নদী থেকে বাংলাদেশের ১৫৩ কিউসেক পানি প্রত্যাহারের বিষয়ে সমঝোতা
স্মারক (এমওইউ)।
\item বৈজ্ঞানিক সহযোগিতা বিষয়ে ভারতের বিজ্ঞান ও শিল্প গবেষণা পরিষদের (সিএসআইআর) সঙ্গে বাংলাদেশের
সিএসআইআরের মধ্যে সমঝোতা স্মারক।
\item বাংলাদেশের সুপ্রিম কোর্টের সঙ্গে ভারতের ভোপালে ন্যাশনাল জুডিশিয়াল অ্যাকাডেমির মধ্যে একটি সমঝোতা স্মারক।
\item ভারতের রেলওয়ের প্রশিক্ষণ ইনস্টিটিউটগুলোতে বাংলাদেশ রেল কর্মীদের প্রশিক্ষণের জন্য
দুই দেশের রেল মন্ত্রণালয়ের মধ্যে সমঝোতা স্মারক।
\item বাংলাদেশ রেলওয়ের তথ্যপ্রযুক্তিগত সহযোগিতার জন্য ভারত ও বাংলাদেশের রেল মন্ত্রণালয়ের মধ্যে সমঝোতা স্মারক।
\item ভারতের রাষ্ট্রীয় সম্প্রচারমাধ্যম ‘প্রসার ভারতীর’ সঙ্গে বাংলাদেশ টেলিভিশনের সমঝোতা স্মারক।
\item মহাশূন্য প্রযুক্তি ক্ষেত্রে সহযোগিতা বিষয়ে বিটিসিএল এবং এনএসআইএল এর মধ্যে সমঝোতা স্মারক
\end{enumerate}

\section{September on Jossore Road}
\label{sec:orgde64edb}

Millions of babies watching the skies
Bellies swollen, with big round eyes
On Jessore Road--long bamboo huts
Noplace to shit but sand channel ruts

Millions of fathers in rain
Millions of mothers in pain
Millions of brothers in woe
Millions of sisters nowhere to go

One Million aunts are dying for bread
One Million uncles lamenting the dead
Grandfather millions homeless and sad
Grandmother millions silently mad

Millions of daughters walk in the mud
Millions of children wash in the flood
A Million girls vomit \& groan
Millions of families hopeless alone

Millions of souls nineteenseventyone
homeless on Jessore road under grey sun
A million are dead, the million who can
Walk toward Calcutta from East Pakistan

Taxi September along Jessore Road
Oxcart skeletons drag charcoal load
past watery fields thru rain flood ruts
Dung cakes on tree trunks, plastic-roof huts

Wet processions Families walk
Stunted boys big heads don't talk
Look bony skulls \& silent round eyes
Starving black angels in human disguise

Mother squats weeping \& points to her sons
Standing thin legged like elderly nuns
small bodied hands to their mouths in prayer
Five months small food since they settled there

\section{সংবিধান সংশোধনী সমূহ}
\label{sec:org0e74295}
\subsection{প্রথম সংশোধনী: যুদ্ধাপরাধীদের বিচার}
\label{sec:orgc84739c}
বাংলাদেশের মুক্তিযুদ্ধের নেতা শেখ মুজিবুর রহমানের নেতৃত্বে আওয়ামী
লীগ সরকার সংবিধান কার্যকর করার পর সাত মাসের মধ্যেই তাতে প্রথম
সংশোধনী আনে। এই সংশোধনীর মূল কারণ ছিল যুদ্ধাপরাধের বিচার নিশ্চিত
করা। বাংলাদেশের মুক্তিযুদ্ধে গণহত্যাজনিত অপরাধ, মানবতাবিরোধী
অপরাধ বা যুদ্ধাপরাধের বিচারের জন্য আইন তৈরি এবং তা কার্যকর করার
বিষয় আনা হয় এই সংশোধনীতে। পরে এর আওতায় আন্তর্জাতিক
মানবতাবিরোধী অপরাধ আইন করা সম্ভব সংবিধানের এই প্রথম সংশোধনী
বিল পাস হয় ১৯৭৩ সালের ১৫ই জুলাই। এর দু'দিন পরই রাষ্ট্রপতি তা
অনুমোদন করেছিলেন সংসদে বিলটি এনেছিলেন সে সময় আওয়ামী লীগ
সরকারের আইনমন্ত্রী মনোরঞ্জন ধর। সংসদে উপস্থিত সদস্যদের মধ্যে শুধু
তিনজন সদস্য ভোট দেয়া থেকে বিরত ছিলেন।


\subsection{দ্বিতীয় সংশোধনী: মৌলিক অধিকার স্থগিত এবং জরুরি অবস্থা}
\label{sec:org35f269b}
সংবিধানে প্রথমে নিবর্তনমূলক আটক এবং জরুরি অবস্থা ঘোষণার বিধান ছিল
না দ্বিতীয় সংশোধনীর মাধ্যমে সংযোজন করা হয় জরুরি অবস্থা ঘোষণা এবং
নিবর্তনমূলক আটকের বিধান জরুরি অবস্থা ঘোষণা করা হলে সে সময় মৌলিক
অধিকারগুলো স্থগিত করার বিধানও আনা কোন প্রেক্ষাপটে বা পরিস্থিতিতে
এমন ব্যবস্থা নেয়া যাবে, সে সম্পর্কে সংশোধনীতে বলা হয়, অভ্যন্তরীণ
গোলযোগ বা বহিরাক্রমণে দেশের নিরাপত্তা ও অর্থনৈতিক জীবন বাধাগ্রস্ত
হলে তখন জরুরি অবস্থা ঘোষণা করা যাবে প্রথম সংশোধনী আনার দুই মাস
পরই ১৯৭৩ সালের ২০শে সেপ্টেম্বর এই দ্বিতীয় সংশোধনী বিল পাস করা
হয়। প্রথম সংসদে এই বিলটিও উত্থাপন করেছিলেন তৎকালীন আইনমন্ত্রী
মনোরঞ্জন ধর বিলটি পাসের সময় তৎকালীন বিরোধীদল এবং স্বতন্ত্র
কয়েকজন সংসদ থেকে ওয়াকআউট করেছিলেন।

\subsection{তৃতীয় সংশোধনী: বাংলাদেশ-ভারত সীমানা}
\label{sec:orgc66b266}
বাংলাদেশ এবং ভারতের সীমানা নির্ধারণ সর্ম্পকিত একটি চুক্তি
বাস্তবায়ন করার জন্য এই সংশোধনী আনা দুই দেশের সীমান্ত চুক্তিতে
ছিটমহল এবং অপদখলীয় জমি বিনিময়ের কথা ছিল সে ব্যাপারেই বিধান
করা হয় সংবিধানের তৃতীয় সংশোধনীর মাধ্যমে। সংশোধনী বিলটি ২৬১টি
ভোট পেয়ে পাস হয়েছিল ১৯৭৪ সালের ২৩ শে নভেম্বর। আর বিপক্ষে ভোট
পড়েছিল ৭টি। সে সময়ের আইনমন্ত্রী মনোরঞ্জন ধর বিলটি সংসদে
এনেছিলেন। এই সংশোধনী আনার ৪০ বছর পর ২০১৫ সালের জুলাই মাসে শেখ
হাসিনার নেতৃত্বাধীন আওয়ামী লীগ সরকারের সময়ে বাংলাদেশ এবং
ভারতের মধ্যে ছিটমহলগুলো বিনিময় হয়েছে।

\subsection{চতুর্থ সংশোধনী: একদলীয় শাসন বাকশাল}
\label{sec:orgc11be26}
এই সংশোধনীর মাধ্যমে বাংলাদেশের শাসন ব্যবস্থাকে আমূল বদলে ফেলা
হয়েছিল। বহুদলীয় সরকার পদ্ধতি বিলুপ্ত করে প্রবর্তন করা হয়েছিল
রাষ্ট্রপতি শাসিত সরকার ব্যবস্থা। সর্বময় ক্ষমতা দেয়া হয়েছিল
রাষ্ট্রপতির হাতে। চারটি পত্রিকা রেখে অন্য সব পত্রিকা বন্ধ করে দেয়া
হয়েছিল। সেই শাসন ব্যবস্থা বাকশাল নামে পরিচিত সংশোধনী বিলটি
সংসদে পাস হয় ১৯৭৫ সালের ২৫শে জানুয়ারি। সেদিনই রাষ্ট্রপতি বিলটি
অনুমোদন করেছিলেন তখনকার সরকারি দল আওয়ামী লীগেরই দু'জন সংসদ
সদস্য অবসরপ্রাপ্ত জেনারেল এম এ জি ওসমানি এবং মঈনুল হোসেন সংসদে
ভোটের সময় অধিবেশন বর্জন করেছিলেন। আর সংশোধনীর পক্ষে ভোট
দিয়েছিলেন ২৯৪ জন সংসদ সদস্য সংসদীয় পদ্ধতির বদলে রাষ্ট্রপতি শাসিত
সরকার ব্যবস্থার জন্য সংবিধানের চতুর্থ সংশোধনী আনা হয়েছিল

\subsubsection{এই মৌলিক পরিবর্তনের প্রেক্ষাপট}
\label{sec:org572400b}
বাংলাদেশের প্রতিষ্ঠাতা রাষ্ট্রপতি শেখ মুজিবুর রহমানের যে পদক্ষেপ
নিয়ে এখনও সবচেয়ে বেশি বিতর্ক হয়, তা হচ্ছে একদলীয় শাসন ব্যবস্থা
প্রবর্তন করা। রাজনৈতিক বিশ্লেষকদের অনেকে মনে করেন, বাংলাদেশের
স্বাধীনতা লাভের পর প্রত্যাশা অনুযায়ী গণতন্ত্রের সুফল মানুষের কাছে
পৌঁছানো যায়নি। সে কারণে রাজনীতি এবং অর্থনীতি সব দিক থেকেই
বিশৃঙ্খলা দেখা দিয়েছিল। আওয়ামী লীগ নেতাদের অনেকে প্রেক্ষাপটকে
ব্যাখ্যা করেন ভিন্নভাব। তারা বলেন, দুর্নীতি, আইন শৃঙ্খলা পরিস্থিতির
চরম অবনতি এবং ১৯৭৪ সালের দুর্ভিক্ষ সব মিলিয়ে ভয়াবহ সংকটের মুখে
বাধ্য হয়ে বাকশাল গঠন করা হয়েছিল। শেখ মুজিব ১৯৭৫ সালের ২৫শে
মার্চ ঢাকায় এক জনসভায় বাকশাল নিয়ে বক্তব্যে ঘুনেধরা সমাজ পাল্টানোর
কথা বলেছিলেন। তবে সব রাজনৈতিক দল বিলুপ্ত করে বাকশাল প্রতিষ্ঠার
পর তা পুরোপুরি কার্যকর হওয়ার আগেই পট পরিবর্তন হয়। উনিশ'শ পঁচাত্তর
সালের ১৫ই অগাষ্ট সেনাবাহিনীর এক দল সদস্য শেখ মুজিবকে স্বপরিবারে
হত্যা করে।

\subsection{পঞ্চম সংশোধনী: সামরিক শাসনের বৈধতা}
\label{sec:orgc20c4a1}
শেখ মুজিবুর রহমানকে হত্যার পর সামরিক শাসন যে জারি করা হয়েছিল,
তার বৈধতা দেয়া হয়েছিল এই সংশোধনীর মাধ্যমে। উনিশ'শ পঁচাত্তর
সালের ১৫ই অগাষ্ট হত্যাকাণ্ডের ঘটনার পর খন্দকার মোশতাক আহমেদ
নিজেকে রাষ্ট্রপতি ঘোষণা করে সামরিক শাসন জারি করেছিলেন। শেখ
মুজিবের হত্যাকারীদের বিচার যাতে করা না যায়, সে ব্যাপারে তিনি
ইনডেমনিটি অধ্যাদেশও জারি করেছিলেন। তিনি মাত্র ৮৩ দিন ক্ষমতায়
ছিলেন। সে সময় সেনাবাহিনীতে একের পর এক অভ্যূত্থান পাল্টা অভ্যূত্থানে
ভেঙে পড়েছিল চেইন অব কমাণ্ড। এমন প্রেক্ষাপটে এক অভ্যূত্থানের মধ্যে
(সিপাহী অভ্যূত্থান হিসাবে পরিচিত) তৎকালীন সেনা প্রধান জেনারেল
জিয়াউর রহমান ক্ষমতায় আসেন ১৯৭৫ সালের ৭ই নভেম্বর। তিনি এক দলীয়
শাসন ব্যবস্থার বদলে বহুদলীয় ব্যবস্থা আবার চালু করেন। তবে সংসদীয়
পদ্ধতিতে ফিরে না গিয়ে তিনি রাষ্ট্রপতি শাসিত সরকার পদ্ধতি বহাল
রাখেন। তিনি জাতীয়তাবাদের ক্ষেত্রেও পরিবর্তন আনেন। বাঙালির বদলে
করা হয় বাংলাদেশি জাতীয়তাবাদ। উনিশ'শ পঁচাত্তর সালের ১৫ই অগাষ্টের
পর থেকে ১৯৭৯ সালের ৫ই এপ্রিল পর্যন্ত সামরিক শাসনের সব কর্মকাণ্ডকে
পঞ্চম সংশোধনীর মাধ্যমে বৈধতা দেয়া হয়। বাংলাদেশের সংবিধানের
অন্যতম একটি মৌলিক বিষয় ছিল ধর্ম নিরপেক্ষতা। জিয়াউর রহমানের
সরকার সেখানে পঞ্চম সংশোধনীর মাধ্যমেই 'বিসমিল্লহির রাহমানির
রাহিম' যুক্ত করে। উনিশ'শ উনআশি সালের ৬ই এপ্রিল বাংলাদেশের
দ্বিতীয় সংসদে তখনকার সংসদ নেতা শাহ আজিজুর রহমান পঞ্চম সংশোধনী
বিলটি উত্থাপন করেছিলেন। এটি পাস হয়েছিল ২৪১-০ ভোটে। তবে দীর্ঘ
সময় পর ২০১০ সালে উচ্চ আদালতের এক রায়ে এই পঞ্চম সংশোধনী অবৈধ
ঘোষণা করা হয়েছে।



\subsection{ষষ্ঠ সংশোধনী: বিচারপতি সাত্তারের বৈধতা}
\label{sec:org4c00eaa}
জিয়াউর রহমানের হত্যাকাণ্ডের পর সংসদে সংবিধানের এই সংশোধনীর
মাধ্যমে বিচারপতি আব্দুস সাত্তারের রাষ্ট্রপতি হওয়ার পথ নিশ্চিত করা
হয়েছিল। উনিশ'শ একাশি সালের ৩০ শে মে চট্টগ্রাম সার্কিট হাউজে
সেনাবাহিনীর এক দল সদস্য রাষ্ট্রপতি এবং বিএনপির প্রতিষ্ঠাতা জিয়াউর
রহমানকে হত্যা করে। এক ব্যর্থ সামরিক অভ্যূত্থানে তার নিহত হওয়ার
ঘটনার পর তৎকালীন উপরাষ্ট্রপতি বিচারপতি আব্দুস সাত্তার অস্থায়ী
রাষ্ট্রপতির দায়িত্ব নেন। উনিশ'শ একাশি সালের ৮ই জুলাই দ্বিতীয়
সংসদের অধিবেশন ডেকে সংবিধানের ষষ্ঠ সংশোধনী বিল পাস করা হয়। এর
মাধ্যমে বিচারপতি সাত্তারের উপরাষ্ট্রপতি পদে বহাল থেকে রাষ্ট্রপতি
পদে নির্বাচনের বিধান নিশ্চিত করা হয়।

\subsection{সপ্তম সংশোধনী: বৈধতা পায় এরশাদের সামরিক শাসন}
\label{sec:orgfcd1317}
বিচারপতি আব্দুস সাত্তারের নেতৃত্বাধীন বিএনপি সরকারকে হটিয়ে ১৯৮২
সালের ২৪শে মার্চ ক্ষমতা দখল করে সামরিক শাসন দেন তৎকালীন
সেনাপ্রধান জেনারেল হুসেইন মুহাম্মদ এরশাদ। তিনি ক্ষমতা দখলের পর
থেকে ১৯৮৬ সালের ১০ই নভেম্বর পর্যন্ত সামরিক শাসন বহাল রেখেছিলেন।
তৃতীয় সংসদে সপ্তম সংশোধনীর মাধ্যমে জেনারেল এরশাদের সেই সামরিক
শাসন এবং সে সময়ের সব কর্মকাণ্ডের বৈধতা দেয়া হয়েছিল। সামরিক
শাসনের বৈধতা দেয়ার এই সপ্তম সংশোধনীকেও উচ্চ আদালত অবৈধ ঘোষণা
করে ২০১০ সালে।

\subsection{অষ্টম সংশোধনী: রাষ্ট্রধর্ম ইসলাম}
\label{sec:orgc052ac6}
এই সংশোধনীর মাধ্যমে রাষ্ট্রের ধর্ম নিরপেক্ষতার নীতি পুরোপুরি পাল্টে
দেয়া হয়। জিয়াউর রহমানের শাসনে পঞ্চম সংশোধনীর মাধ্যমে
'বিসমিল্লাহির রাহমানির রাহিম যুক্ত করা হয়েছিল। আর জেনারেল
এরশাদের শাসনামলে সংবিধানে এই অষ্টম সংশোধনীর মাধ্যমে ইসলামকে
রাষ্ট্রধর্ম হিসাবে ঘোষণা করা হয়। জেনারেল এরশাদ ক্ষমতায় থেকে ছোট
ছোট কয়েকটি দলকে নিয়ে ১৯৮৮ সালে সংসদ নির্বাচন করেছিলেন। সেই
নির্বাচনের মাধ্যমে গঠিত চতুর্থ সংসদে অষ্টম সংশোধনী বিলটি পাস করা
হয়েছিল। তার শাসনের বিরুদ্ধে আন্দোলনকারী আওয়ামী লীগ এবং বিএনপিসহ
বেশির ভাগ দল ১৯৮৮ সালের ঐ সংসদ নির্বাচন বর্জন করেছিল। বিএনপির
সিনিয়র নেতা মওদুদ আহমেদ সে সময় জেনারেল এরশাদের জাতীয় পার্টির
নেতা হিসাবে সংসদ নেতা হয়েছিলেন। তিনিই অষ্টম সংশোধনী বিলটি
সংসদে তুলেছিলেন। বিলটি পাস হয়েছিল ২৫৪-০ ভোটে। জেনারেল এরশাদের
শাসনের বিরুদ্ধে আন্দোলনকারী বামপন্থী দলগুলো ও শেখ হাসিনার
নেতৃত্বাধীন আওয়ামী লীগ এবং এমনকি খালেদা জিয়ার নেতৃত্বাধীন
বিএনপিও ইসলামকে রাষ্ট্রধর্ম করার ঐ সংশোধনীর বিরুদ্ধে রাজপথে
আন্দোলন করেছিল। বিশ্লেষকরা বলেছেন, সব দলের তীব্র আন্দোলনের মুখে
ক্ষমতা টিকিয়ে রাখতে জেনারেল এরশাদ তখন সংবিধানে এই অষ্টম
সংশোধনী এনেছিলেন। অষ্টম সংশোধনীতে আরেকটি বড় বিষয় আনা হয়েছিল।
সেটি হচ্ছে ঢাকার বাইরে হাইকোর্টের ছয়টি স্থায়ী বেঞ্চ স্থাপন করা।
তবে সে সময়ই সর্বোচ্চ আদালত ঢাকার বাইরে হাইকোর্টের বেঞ্চ গঠনের
বিষয়টি বাতিল করে দিয়েছে।

\subsection{নবম সংশোধনী: একজন কতবার রাষ্ট্রপতি হতে পারবেন}
\label{sec:orgf9cecb9}
উনিশ'শ উননব্বই সালের ১১ই জুলাই এই সংশোধনী পাসের মাধ্যমে
রাষ্ট্রপতি এবং উপরাষ্ট্রপতির ব্যাপারে কিছু বিধান যুক্ত করা হয়। এর
ফলে রাষ্ট্রপতি পদে কোন ব্যক্তি পর পর দুই মেয়াদের বেশি থাকতে
পারবেন না। এছাড়া রাষ্ট্রপতি এবং উপরাষ্ট্রপতির নির্বাচন একই সাথে
করার বিষয়টিও ছিল।

\subsection{দশম সংশোধনী: রাষ্ট্রপতি নির্বাচন}
\label{sec:org9375ff7}
জেনারেল এরশাদের শাসনের শেষদিকে ১৯৯০ সালের ১২ই জুন সংশোধনীটি
সংসদে পাস করা হয়েছিল। রাষ্ট্রপতির কার্যকালের মেয়াদ শেষ হওয়ার
১৮০ দিনের মধ্যে নির্বাচন করার বিধান আনা হয়েছিল এই সংশোধনীতে।

\subsection{একাদশ সংশোধনী: বিচারপতি সাহাবুদ্দীন আহমদের দায়িত্ব}
\label{sec:org33af6f9}
গণঅভ্যূত্থানে ১৯৯০ সালের ৬ই ডিসেম্বর জেনারেল এরশাদ সরকারের পতনের
পর পঞ্চম সংসদের নির্বাচন পরিচালনার জন্য তিন মাসের অন্তবর্তীকালীন
বা অস্থায়ী একটি সরকার গঠন করা হয়েছিল। সে সময় আওয়ামী লীগ এবং
বিএনপিসহ আন্দোলনকারী সব দলের ঐকমত্যের ভিত্তিতে তৎকালীন প্রধান
বিচারপতি সাহাবুদ্দীন আহমদের নেতৃত্বে সেই অস্থায়ী সরকার গঠন করা
হয়েছিল। সেজন্য বিচারপতি আহমদকে প্রথমে উপরাষ্ট্রপতি হিসাবে দায়িত্ব
দেয়া হয় এবং এরপর তিনি অস্থায়ী রাষ্ট্রপতির দায়িত্ব পালন করেন।
এছাড়া তিনি অস্থায়ী রাষ্ট্রপতি হিসাবে দায়িত্ব পালনের পর নির্বাচন
শেষে প্রধান বিচারপতির পদে ফিরে গিয়েছিলেন। এই দু'টি বিষয়ে বৈধতা
দেয়া হয়েছিল সংবিধানের একাদশ সংশোধনীর মাধ্যমে। উনিশ'শ একানব্বই
সালে সব দলের অংশ গ্রহণে নির্বাচনের মাধ্যমে গঠিত পঞ্চম সংসদে এই
সংশোধনী পাস করা হয়।

\subsection{দ্বাদশ সংশোধনী: সংসদীয় পদ্ধতিতে ফেরত}
\label{sec:orgbf3a6b1}
দীর্ঘ ১৬ বছর পর রাষ্ট্রপতি শাসিত সরকার ব্যবস্থা পাল্টিয়ে সংসদীয়
সরকার পদ্ধতি পুনরায় প্রবর্তন করা হয় দ্বাদশ সংশোধনীর মাধ্যমে।
জেনারেল এরশাদের পতনের পর ১৯৯১ সালে সংসদ নির্বাচনে জয়ী বিএনপি
সরকার গঠন করলে দলটির নেত্রী খালেদা জিয়া প্রধানমন্ত্রী এবং সংসদ
নেতা হয়েছিলেন। বিএনপিকে জামায়াতে ইসলামীর সমর্থন নিতে হয়েছিল।
তবে সরকার গঠনে সমর্থন দিলেও জামাত সেই সরকারে অংশীদার ছিল না,
সেটি ছিল বিএনপির সরকার। আর আওয়ামী লীগ নেত্রী শেখ হাসিনা
বসেছিলেন বিরোধীদলীয় নেতার আসনে। এরশাদ বিরোধী আন্দোলনের সময়
আওয়ামী লীগ এবং বিএনপির নেতৃত্বে দু'টি জোট ও বামপন্থী পাঁচটি দলের
জোট-এই তিনটি জোটের রুপরেখায় সংসদীয় পদ্ধতিতে ফিরে যাওয়ার কথা
বলা হয়েছিল। ফলে ১৯৯১ এর নির্বাচনের মাধ্যমে গঠিত সংসদের শুরুতেই
সেই উদ্যোগ নেয়া হয়। সংসদীয় পদ্ধতি প্রবর্তনের জন্য দ্বাদশ সংশোধনী
বিল সংসদে উত্থাপন করেছিলেন তৎকালীন সংসদ নেতা খালেদা জিয়া। এই
সংশোধনী বিল পাস হয়েছিল সরকারি এবং বিরোধী দলের সদস্যদের
ঐকমত্যের ভিত্তিতে। ১৯৯১ সালের ৬ই অগাষ্ট সংসদে সংশোধনীটি পাসের
ক্ষেত্রে ৩০৭-০ ভোট পড়েছিল।

\subsection{ত্রয়োদশ সংশোধনী: তত্ত্বাবধায়ক সরকার}
\label{sec:org8c61b17}
এই সংশোধনীর মাধ্যমে সাধারণ নির্বাচন ব্যবস্থায় বড় ধরনের পরিবর্তন
আনা হয়েছিল। নির্বাচন অনুষ্ঠানের জন্য তিন মাস মেয়াদের
'নির্দলীয়'-'নিরপেক্ষ' তত্ত্বাবধায়ক সরকার ব্যবস্থা প্রবর্তন করা
হয়েছিল। একজন প্রধান উপদেষ্টা এবং দশ জন উপদেষ্টা নিয়ে এই সরকার
গঠিত হতো। এরশাদ সরকারের পতনের পর একটি অস্থায়ী সরকারের অধীনে
নির্বাচনে বিএনপি খালেদা জিয়ার নেতৃত্বে ক্ষমতায় এসেছিল। সেই
সরকারের সময়ে শেখ হাসিনার নেতৃত্বে আওয়ামী লীগ সহ বেশিরভাগ দল
এবং অন্যদিকে জামায়াতে ইসলামী তত্ত্বাবধায়ক সরকারের দাবিতে সংসদে
এবং রাজপথে আন্দোলন গড়ে তুলেছিল। আন্দোলনের এক পর্যায়ে আওয়ামী
লীগসহ সংসদে প্রতিনিধিত্বকারী বিরোধী দলগুলোর ১৪৭ জন সংসদ সদস্য
একযোগে সংসদ থেকে পদত্যাগ করেছিলেন ১৯৯৪ সালের ২৮শে ডিসেম্বর।
বিরোধী সদস্যদের আসন শূণ্য করা না করার প্রশ্নে সিদ্ধান্ত অনেকদিন
ঝুলিয়ে রাখা হয়েছিল। শেষপর্যন্ত ১৯৯৫ সালের অগাষ্ট মাসে আসনগুলো শূণ্য
ঘোষণা করে উপনির্বাচন করার উদ্যোগও নেয়া হয়েছিল। উপনির্বাচন করতে
না পেরে বিরোধীদলগুলোর আন্দোলনের মুখে বিএনপি সরকার ১৯৯৫ সালের
২৪শে নভেম্বর পঞ্চম জাতীয় সংসদ ভেঙে দিয়েছিল। আওয়ামী লীগ,
বামপন্থী দলগুলো এবং জামায়াতে ইসলামীসহ আন্দোলনকারী দলগুলোর বর্জনের
মুখে বিএনপি ষষ্ঠ সংসদ নির্বাচন করেছিল ১৯৯৬ সালের ১৫ই ফেব্রুয়ারি।
বিএনপি ছাড়া ঐ নির্বাচনে শুধু একটি দল ফ্রিডম পার্টির নেতা চাকরিচ্যুত
সেনা কর্মকর্তা এবং ১০জন স্বতন্ত্র প্রার্থী হয়ে সংসদে এসেছিলেন। ষষ্ঠ
সংসদে বিএনপির বাইরে একটি দল থেকে একজন সংসদ সদস্য হয়েছিলেন।
ফলে এই সংসদে কোন বিরোধীদল ছিল না। তবে বাংলাদেশের প্রতিষ্ঠাতা
রাষ্ট্রপতি শেখ মুজিবুর রহমান হত্যাকাণ্ডে জড়িত (পরে পলাতক অবস্থায়
মৃত্যুদণ্ডের সাজাপ্রাপ্ত) এবং চাকরিচ্যুত সেনা কর্মকর্তা খন্দকার আবদুর
রশিদকে ঐ সংসদে বিরোধীদলীয় নেতার চেয়ারে বসানো হয়েছিল। সংসদ
নেতা এবং প্রধানমন্ত্রী ছিলেন খালেদা জিয়া। আওয়ামী লীগ সহ
বিরোধীদলগুলোর তীব্র আন্দোলনের মুখে তত্ত্বাবধায়ক সরকার ব্যবস্থা
প্রবর্তনের জন্য ষষ্ঠ সংসদে সংবিধানের ত্রয়োদশ সংশোধনী বিল পাস করা
হয় ১৯৯৬ সালের ২৭শে মার্চ। বিলটি এনেছিলেন তৎকালীন আইন মন্ত্রী
জমির উদ্দিন সরকার। এরপর ৩০শে মার্চ সংসদ ভেঙে দিয়ে সংশোধনী
অনুযায়ী তত্ত্বাবধায়ক সরকারের কাছে ক্ষমতা হস্তান্তর করেছিল বিএনপি
সরকার। উনিশ'শ ছিয়ানব্বই সালের ১৯শে মার্চ থেকে ৩০শে মার্চ পর্যন্ত
মাত্র ১২ দিন টিকেছিল সেই ষষ্ঠ সংসদ।

\subsection{চতুর্দশ সংশোধনী: সংরক্ষিত মহিলা আসন}
\label{sec:org954a8b0}
খালেদা জিয়ার নেতৃত্বে ২০০১ সালে বিএনপি আবার ক্ষমতায় আসার পর এই
সংশোধনীর মাধ্যমে সংরক্ষিত মহিলা আসন ৩০ থেকে বাড়িয়ে ৪৫টি করা
হয়েছিল। তবে এই সংশোধনীর মাধ্যমে সুপ্রিমকোর্টের বিচারপতিদের অবসর
নেয়ার বয়স ৬৫ থেকে বাড়িয়ে ৬৭ বছর করা হয়েছিল। সরকারি এবং
স্বায়ত্তশাসিত সব প্রতিষ্ঠানে প্রধানমন্ত্রীর ছবি প্রদর্শনের বিধানও করা
হয়েছিল এর মাধ্যমে। দু'হাজার চার সালের ১৬ই মে সংসদে এই সংশোধনী
বিল সংসদে উত্থাপন করেছিলেন তৎকালীন আইনমন্ত্রী মওদুদ আহমেদ। কিন্তু
বিচারপতিদের অবসরের বয়স বাড়ানোর ব্যাপারে বিএনপি সরকারের উদ্দেশ্য
নিয়ে প্রশ্ন তুলেছিল তখনকার বিরোধী দল আওয়ামী লীগ। নির্বাচন
পরিচালনার তত্ত্বাবধায়ক সরকার গঠনের বিধানে প্রধান উপদেষ্টা
নিয়োগের শর্তগুলোর মধ্যে এক নম্বরেরই ছিল যে, সর্বশেষ অবসর নেয়া
প্রধান বিচারপতি প্রধান উপদেষ্টা হবেন। তিনি অসম্মতি জানালে তখন
আরও চারটি উপায় নির্ধারণ করা ছিল। ফলে বিচারপতিদের অবসরের বয়স
বাড়ানোর পর তত্ত্বাবধায়ক সরকার গঠন নিয়ে জটিলতা দেখা দিলে
রাজনৈতিক দলগুলো বিরোধ সংঘাতে রুপ নিয়েছিল। শেষপর্যন্ত ২০০৭ সারে
জরুরি অবস্থা জারি করে সেনা বাহিনীর সমর্থনে তত্ত্বাবধায়ক সরকার গঠন
করা হয়েছিল।

\subsection{পঞ্চদশ সংশোধনী: ধর্মনিরপেক্ষতা}
\label{sec:orgdda33db}
শেখ হাসিনার নেতৃত্বে আওয়ামী লীগ দ্বিতীয় দফায় সরকার গঠনের দুই বছর
পর ২০১১ সালে এই সংশোধনীর মাধ্যমে ১৯৭২ সালের সংবিধানের মৌলিক
কিছু বিষয় ফিরিয়ে আনা হয়। রাষ্ট্রের মূলনীতি হিসাবে জাতীয়তাবাদ,
সমাজতন্ত্র, গণতন্ত্র এবং ধর্মনিরপেক্ষতার নীতি সংযোজন করা হয়।
সংবিধানে ধর্ম নিরপেক্ষতা এবং ধর্মীয় স্বাধীনতা পুনর্বহাল করা হয়।
তবে রাষ্ট্রধর্ম ইসলাম বহাল রাখা হয়। এই সংশোধনীর মাধ্যমেই
মুক্তিযুদ্ধের নেতা শেখ মুজিবুর রহমানকে জাতির পিতা হিসাবে স্বীকৃতি
দেয়া হয়। তত্ত্বাবধায়ক সরকার বাতিল দু'হাজার সাত সালে সেনা সমর্থিত
তত্ত্বাবধায়ক সরকার দুই বছর ক্ষমতায় থেকে তারপর নির্বাচন দিয়েছিল।
সেই নির্বাচনের মাধ্যমে ক্ষমতায় আসার পর আওয়ামী লীগের পক্ষ থেকে
তত্ত্বাবধায়ক সরকার ব্যবস্থা নিয়েই প্রশ্ন তোলা হয় । বিষয়টি আদালত
পর্যন্ত গড়ায় এবং ২০১১ সালের মে মাসে সর্বোচ্চ আদালত তত্ত্বাবধায়ক
সরকার ব্যবস্থা সর্ম্পকিত ত্রয়োদশ সংশোধনী বাতিল করে দেয়। আদালত
অবশ্য বলেছিল, তত্ত্বাবধায়ক সরকারের অধীনে আরও দু'টি সংসদ নির্বাচন
হতে পারে। কিন্তু আওয়ামী লীগ সরকার সেই ব্যবস্থায় আর কোন নির্বাচন
করার সুযোগ না রেখে পঞ্চদশ সংশোধনীর মাধ্যমে তত্ত্বাবধায়ক সরকার
ব্যবস্থা বাতিল করে দিয়েছে। সেখানে ফিরিয়ে আনা হয়েছে রাজনৈতিক
সরকারের অধীনে নির্বাচন ব্যবস্থা। এই সংশোধনীতে নির্বাচিত সরকারের
মেয়াদ শেষ হওয়ার আগের ৯০ দিনের মধ্যে নির্বাচন করার কথা বলা হয়।
সে সময় সংসদ ভেঙে না গেলেও কোন অধিবেশন বসবে না। রাজনৈতিক সেই
সরকার শুধু রাষ্ট্রের রুটিন কাজ করবে বলে বিধান রাখা হয়েছে। তবে
বিএনপি এবং জামায়াতে ইসলামী সহ বিভিন্ন দল তত্ত্বাবধায়ক সরকার
ব্যবস্থা বাতিলের বিরোধীতা করেছিলো। নবম জাতীয় সংসদে সংশোধনীটি
পাসের সময় বিএনপি-জামায়াতের সংসদ সদস্যরা অধিবেশনে অনুপস্থিত
ছিলেন। তখন সংসদে একমাত্র স্বতন্ত্র সদস্য ফজলুল আজিম বিপক্ষে ভোট
দিয়েছিলেন। আর পক্ষে ভোট পড়েছিল ২৯১টি। পঞ্চদশ সংশোধনী বিলটি
তৎকালীন আইনমন্ত্রী শফিক আহমেদ সংসদে এনেছিলেন ২০১১ সালের জুন
মাসে। শুধু একটি সংসদে ১৯৯৬ সাল থেকে ২০০১ সাল পর্যন্ত মেয়াদের
সপ্তম সংসদে কোন সংশোধনী আনা হয়নি।


\subsection{ষোড়শ সংশোধনী: বিচারপতি অপসারণ}
\label{sec:org1772224}
এই সংশোধনীর মাধ্যমে বিচারপতিদের অপসারণের ক্ষমতা সংসদের হাতে
ফিরিয়ে আনা হয়। দু'হাজার চৌদ্দ সালের ১৭ই সেপ্টেম্বর বিলটি সংসদে
পাস হয়। আইনমন্ত্রী আনিসুল হক বিলটি সংসদে উত্থাপন করেছিলেন।তবে
আপীল বিভাগ ষোড়শ সংশোধনীকে বাতিল ঘোষণা করে অপসারণের ক্ষমতা
সুপ্রিম জুডিডিশিয়াল কাউন্সিলের হাতে ফিরিয়ে নিয়েছিল।পরে তৎকালীন
প্রধান বিচারপতি এস কে সিনহার সাথে আওয়ামী লীগ সরকারের দূরত্ব
সৃষ্টি হয়েছিল।আওয়ামী লীগ সরকার আপিল বিভাগের সিদ্ধান্তের বিরুদ্ধে
রিভিউ আবেদন করে। দীর্ঘ সময়ে রিভিউ আবেদনের নিস্পত্তি হয়নি। আর এই
রিভিউ আবেদন আদালতে নিস্পত্তির অপেক্ষায় থাকলেও বিচারপতিদের
অপসারণের ক্ষমতা সংসদের হাতে থাকার সেই সংশোধনী বহাল রয়েছে।

\subsection{সপ্তদশ সংশোধনী: নারী আসন}
\label{sec:org4b22364}
এই সংশোধনীর মাধ্যমে ৫০টি সংরক্ষিত নারী আসন আরও ২৫ বছর বহাল
রাখার বিধান আনা হয়েছে।

\section{SDG 17 Goals}
\label{sec:orgf9eee3e}

\begin{enumerate}
\item No Poverty
\item Zero Hunger
\item Good Health and Well-being
\item Quality Education
\item Gender Equality
\item Clean Water and Sanitation
\item Affordable and Clean Energy
\item Decent Work and Economic Growth
\item Industry, Innovation and Infrastructure
\item Reduced Inequality
\item Sustainable Cities and Communities
\item Responsible Consumption and Production
\item Climate Action
\item Life Below Water
\item Life on Land
\item Peace, Justice and Strong Institutions
\item Partnerships to achieve the Goal
\end{enumerate}

\section{সংবিধানের তফসিল সমূহ}
\label{sec:org43da6e0}

\subsection{প্রথম তফসিল}
\label{sec:org83f3bf2}
\subsubsection{অন্যান্য বিধান সত্ত্বেও কার্যকর আইন}
\label{sec:org97d3087}
অর্থাৎ, সংবিধানে যে বিধানই থাকুক না কেন,
এই তফসিলে বর্ণিত আইনগুলো  অন্যান্য বিধান থাকা সত্ত্বেও কার্যকর হবে। এই আইনগুলো
অনুসারে গৃহীত কোন কর্মকান্ড অবৈধ ঘোষণা করা যাবে না। প্রথম তফসিলে সংবিধানের
\textbf{৪৭ নং অনুচ্ছেদের} বিস্তারিত ব্যাখ্যা প্রদান করা হয়েছে।

\subsection{দ্বিতীয় তফসিল}
\label{sec:org2347aa7}
\subsubsection{রাষ্ট্রপতি নির্বাচন}
\label{sec:orge4c8984}
সংবিধানের চতুর্থ সংশােধন আইন, ১৯৭৫ এর ৩০ নং ধারাবলে মূল
সংবিধানের এই দ্বিতীয় তফসিলটি বিলুপ্ত ঘোষণা করা হয়েছে। অর্থাৎ, দ্বিতীয় তফসিল
এখন আর কার্যকর নেই।


\subsection{তৃতীয় তফসিল}
\label{sec:orgf5912e4}
\subsubsection{শপথ ও ঘোষণা}
\label{sec:org372db73}
রাষ্ট্রে সাংবিধানিক পদে দায়িত্বপালনকারী বিভিন্ন ব্যক্তি যেসব শপথ ঘোষণার
মাধ্যমে তাদের দায়িত্ব শুরু করবেন, সেগুলোই ৩ তফসিলে বর্ণিত আছে। এই তফসিলে
\textbf{সংবিধানের ১৪৮ নং অনুচ্ছেদের} বিস্তারিত ব্যাখ্যা প্রদান করা হয়েছে।

\subsection{চতুর্থ তফসিল}
\label{sec:org9c63d39}
\subsubsection{ক্রান্তিকালীন ও অস্থায়ী বিধানাবলি}
\label{sec:orge7b85c7}
অর্থাৎ ১৯৭১ সালের ৭ মার্চ থেকে ১৯৭২ সালের ১৬ ডিসেম্বর পর্যন্ত বাংলাদেশের
নতুন সংবিধান প্রণয়ন ও কার্যকর করার আগ পর্যন্ত সময়কালকে বাংলাদেশের সংবিধানে
ক্রান্তিকাল বলে অভিহিত করা হয়েছে। এই ক্রান্তিকালীন সময়ে বাংলাদেশের যথাযথ
কর্তৃপক্ষ কর্তৃক প্রণীত ও গৃহীত এবং বাংলাদেশের প্রথম অস্থায়ী সরকার কর্তৃক প্রণীত ও
গৃহীত আইন ও বিধানের বৈধতা প্রদান করা হয়েছে। চতুর্থ তফসিলের মাধ্যমে
\textbf{সংবিধানের ১৫০(১) নং অনুচ্ছেদের} বিস্তারিত ব্যাখ্যা প্রদান করা হয়েছে।


\subsection{পঞ্চম তফসিল}
\label{sec:org59617eb}
\subsubsection{৭ই মার্চের ঐতিহাসিক ভাষণ}
\label{sec:orgfc85bf5}
১৯৭১ সালের ৭ই মার্চ, ঢাকার রেসকোর্স ময়দানে
জাতির পিতা বঙ্গবন্ধু শেখ মুজিবুর রহমানের দেওয়া ঐতিহাসিক ভাষণ। এই তফসিলে
\textbf{সংবিধানের ১৫০(২) নং অনুচ্ছেদের} বর্ণনা প্রদান করা হয়েছে।

\subsection{ষষ্ঠ তফসিল}
\label{sec:orga2579e7}
\subsubsection{স্বাধীনতার ঘোষণা}
\label{sec:org450b7b3}
জাতির পিতা বঙ্গবন্ধু শেখ মুজিবুর রহমান কর্তৃক প্রদত্ত
বাংলাদেশের স্বাধীনতার ঘোষণা। ১৯৭১ সালের ২৫ মার্চ মধ্য রাত শেষে অর্থাৎ ২৬
মার্চ প্রথম প্রহরে জাতির পিতা বঙ্গবন্ধু শেখ মুজিবুর রহমান কর্তৃক প্রদত্ত স্বাধীনতার
ঘােষণা। এই তফসিলে \textbf{সংবিধানের ১৫০(২) নং অনুচ্ছেদের} বিস্তারিত বর্ণনা প্রদান
করা হয়েছে।

\subsection{সপ্তম তফসিল}
\label{sec:orge24ad9c}
\subsubsection{স্বাধীনতার ঘোষণাপত্র}
\label{sec:orgc3b36c1}
১৯৭১ সালের ১০ই এপ্রিল, মুজিবনগর সরকার কর্তৃক জারিকৃত
স্বাধীনতার ঘোষণাপত্র। এই তফসিলে \textbf{সংবিধানের ১৫০(২) নং} অনুচ্ছেদের বিস্তারিত
বর্ণনা প্রদান করা হয়েছে

\section{স্বাধীনতার ঘোষণাপত্র}
\label{sec:org7254ed6}
যেহেতু ১৯৭০ সালের ৭ ডিসেম্বর থেকে ১৯৭১ সালের ১৭ জানুয়ারি পর্যন্ত বাংলাদেশে
অবাধ নির্বাচনের মাধ্যমে শাসনতন্ত্র রচনার উদ্দেশ্যে প্রতিনিধি নির্বাচিত করা
হয়েছিল; এবং

যেহেতু এই নির্বাচনে বাংলাদেশের জনগণ ১৬৯টি আসনের মধ্যে আওয়ামী লীগ দলীয় ১৬৭
জন প্রতিনিধি নির্বাচিত করেছিল; এবং

যেহেতু জেনারেল ইয়াহিয়া খান ১৯৭১ সালের ৩ মার্চ তারিখে শাসনতন্ত্র রচনার
উদ্দেশ্যে নির্বাচিত প্রতিনিধিদের অধিবেশন আহবান করেন; এবং

যেহেতু তিনি আহূত এই অধিবেশন স্বেচ্ছাচার এবং বেআইনিভাবে অনির্দিষ্টকালের জন্য
বন্ধ ঘোষণা করেন; এবং

যেহেতু পাকিস্তান কর্তৃপক্ষ তাদের প্রতিশ্রুতি পালন করার পরিবর্তে বাংলাদেশের
জনপ্রতিনিধিদের সঙ্গে পারস্পরিক আলোচনাকালে ন্যায়নীতি বহির্ভূত এবং
বিশ্বাসঘাতকতামূলক যুদ্ধ ঘোষণা করেন; এবং

যেহেতু উল্লিখিত বিশ্বাসঘাতকতামূলক কাজের জন্য উদ্ভূত পরিস্থিতির পরিপ্রেক্ষিতে
বাংলাদেশের সাড়ে সাত কোটি মানুষের অবিসংবাদিত নেতা বঙ্গবন্ধু শেখ মুজিবুর রহমান
জনগণের আত্মনিয়ন্ত্রণাধিকার অর্জনের আইনানুগ অধিকার প্রতিষ্ঠার জন্য ১৯৭১ সালের
২৬ মার্চ ঢাকায় যথাযথভাবে স্বাধীনতা ঘোষণা করেন, এবং বাংলাদেশের অখন্ডতা ও
মর্যাদা রক্ষার জন্য বাংলাদেশের জনগণের প্রতি উদাত্ত আহবান জানান; এবং

যেহেতু পাকিস্তান কর্তৃপক্ষ বর্বর ও নৃশংস যুদ্ধ পরিচালনা করেছে এবং এখনও
বাংলাদেশের বেসামরিক ও নিরস্ত্র জনগণের বিরুদ্ধে নজিরবিহীন গণহত্যা ও নির্যাতন
চালাচ্ছে; এবং

যেহেতু পাকিস্তান সরকার অন্যায় যুদ্ধ ও গণহত্যা এবং নানাবিধ নৃশংস অত্যাচার
পরিচালনা দ্বারা বাংলাদেশের গণপ্রতিনিধিদের পক্ষে একত্রিত হয়ে শাসনতন্ত্র প্রণয়ন
করে জনগণের সরকার প্রতিষ্ঠা করা অসম্ভব করে তুলেছে; এবং

যেহেতু বাংলাদেশের জনগণ তাদের বীরত্ব, সাহসিকতা ও বিপ্লবী কার্যক্রমের মাধ্যমে
বাংলাদেশের উপর তাদের কার্যকর কর্তৃত্ব প্রতিষ্ঠা করেছে;

সার্বভৌম ক্ষমতার অধিকারী বাংলাদেশের জনগণ নির্বাচিত প্রতিনিধিদের প্রতি যে
ম্যান্ডেট দিয়েছেন সে ম্যান্ডেট মোতাবেক আমরা, নির্বাচিত প্রতিনিধিরা, আমাদের
সমবায়ে গণপরিষদ গঠন করে পারস্পরিক আলাপ-আলোচনার মাধ্যমে বাংলাদেশের জনগণের
জন্য সাম্য, মানবিক মর্যাদা ও সামাজিক ন্যায়বিচার প্রতিষ্ঠা করার উদ্দেশ্যে
বাংলাদেশকে একটি সার্বভৌম গণপ্রজাতন্ত্র ঘোষণা করছি

এবং এর দ্বারা পূর্বাহ্ণে বঙ্গবন্ধু শেখ মুজিবুর রহমানের স্বাধীনতা ঘোষণা অনুমোদন
করছি; এবং

এতদ্বারা আমরা আরও সিদ্ধান্ত ঘোষণা করছি যে, শাসনতন্ত্র প্রণীত না হওয়া পর্যন্ত
বঙ্গবন্ধু শেখ মুজিবুর রহমান প্রজাতন্ত্রের রাষ্ট্রপ্রধান এবং সৈয়দ নজরুল ইসলাম
উপ-রাষ্ট্রপ্রধান পদে অধিষ্ঠিত থাকবেন; এবং

রাষ্ট্রপ্রধান প্রজাতন্ত্রের সশস্ত্র বাহিনীসমূহের সর্বাধিনায়ক পদে অধিষ্ঠিত থাকবেন;
ক্ষমা প্রদর্শনের ক্ষমতাসহ সর্বপ্রকার প্রশাসনিক ও আইন প্রণয়নের ক্ষমতার অধিকারী
থাকবেন; এবং

তাঁর কর ধার্য ও অর্থব্যয়ের ক্ষমতা থাকবে; এবং

বাংলাদেশের জনসাধারণের জন্য আইনানুগ ও নিয়মতান্ত্রিক সরকার প্রতিষ্ঠার জন্য
অন্যান্য প্রয়োজনীয় সকল ক্ষমতারও তিনি অধিকারী হবেন।

বাংলাদেশের জনগণের দ্বারা নির্বাচিত প্রতিনিধি হিসাবে আমরা আরও সিদ্ধান্ত
ঘোষণা করছি যে, কোনো কারণে যদি রাষ্ট্রপ্রধান না থাকেন অথবা যদি রাষ্ট্রপ্রধান
কাজে যোগদান করতে না পারেন অথবা তাঁর দায়িত্ব ও কর্তব্য পালনে যদি অক্ষম হন,
তবে রাষ্ট্রপ্রধান প্রদত্ত সকল দায়িত্ব উপ-রাষ্ট্রপ্রধান পালন করবেন। আমরা আরও
সিদ্ধান্ত ঘোষণা করছি যে, বিশ্বের একটি জাতি হিসাবে এবং জাতিসংঘের সনদ
মোতাবেক আমাদের উপর যে দায়িত্ব ও কর্তব্য বর্তেছে তা যথাযথভাবে আমরা পালন
করব।

আমরা আরও সিদ্ধান্ত ঘোষণা করছি যে, আমাদের এই স্বাধীনতার ঘোষণা ১৯৭১ সালের
২৬ মার্চ থেকে কার্যকর বলে গণ্য হবে। আমরা আরও সিদ্ধান্ত ঘোষণা করছি যে,
আমাদের এই সিদ্ধান্ত কার্যকর করার জন্য আমরা অধ্যাপক এম. ইউসুফ আলীকে যথাযথভাবে
রাষ্ট্রপ্রধান ও উপ-রাষ্ট্রপ্রধানের শপথ গ্রহণ অনুষ্ঠান পরিচালনার জন্য দায়িত্ব অর্পণ
ও নিযুক্ত করলাম।

স্বাক্ষর: অধ্যাপক এম. ইউসুফ আলী
বাংলাদেশ গণপরিষদের ক্ষমতা দ্বারা
এবং ক্ষমতাবলে যথাবিধি সর্বাধিক ক্ষমতাধিকারী।

\section{সংবিধান}
\label{sec:org087eb10}
\subsection{১৷ প্রজাতন্ত্র}
\label{sec:orgd27acc2}
বাংলাদেশ একটি একক, স্বাধীন ও সার্বভৌম প্রজাতন্ত্র, যাহা “গণপ্রজাতন্ত্রী বাংলাদেশ” নামে পরিচিত হইবে৷

\subsection{২৷ প্রজাতন্ত্রের রাষ্ট্রীয় সীমানা}
\label{sec:org29b9f30}
প্রজাতন্ত্রের রাষ্ট্রীয় সীমানার অন্তর্ভুক্ত হইবে

(ক) ১৯৭১ খ্রীষ্টাব্দের মার্চ মাসের ২৬ তারিখে স্বাধীনতা-ঘোষণার অব্যবহিত পূর্বে
    যে সকল এলাকা লইয়া পূর্ব পাকিস্তান গঠিত ছিল ৪ এবং সংবিধান (তৃতীয় সংশোধন)
    আইন, ১৯৭৪-এ অন্তর্ভুক্ত এলাকা বলিয়া উল্লিখিত এলাকা, কিন্তু উক্ত আইনে বহির্ভূত
    এলাকা বলিয়া উল্লিখিত এলাকা তদ্‌বহির্ভূত; এবং।

(খ) যে সকল এলাকা পরবর্তীকালে বাংলাদেশের সীমানাভুক্ত হইতে পারে৷

\subsection{২ক৷ রাষ্ট্রধর্ম}
\label{sec:orgdc64386}
ক৷ প্রজাতন্ত্রের রাষ্ট্রধর্ম ইসলাম, তবে অন্যান্য ধর্মও প্রজাতন্ত্রে শান্তিতে পালন করা যাইবে৷

\subsection{৩৷ রাষ্ট্রভাষা}
\label{sec:orgec7a104}
প্রজাতন্ত্রের রাষ্ট্রভাষা বাংলা৷

\subsection{৪৷ জাতীয় সঙ্গীত, পতাকা ও প্রতীক}
\label{sec:org76a213c}
(১) প্রজাতন্ত্রের জাতীয় সঙ্গীত “আমার সোনার বাংলা”র প্রথম দশ চরণ৷

(২) প্রজাতন্ত্রের জাতীয় পতাকা হইতেছে সবুজ ক্ষেত্রের উপর স্থাপিত রক্তবর্ণের একটি
    ভরাট বৃত্ত৷

(৩) প্রজাতন্ত্রের জাতীয় প্রতীক হইতেছে উভয় পার্শ্বে ধান্যশীর্ষবেষ্টিত, পানিতে
    ভাসমান জাতীয় পুষ্প শাপলা, তাহার শীর্ষদেশে পাটগাছের তিনটি পরস্পর-সংযুক্ত পত্র,
    তাহার উভয় পার্শ্বে দুইটি করিয়া তারকা৷

(৪) উপরি-উক্ত দফাসমূহ-সাপেক্ষে জাতীয় সঙ্গীত, পতাকা ও প্রতীক সম্পর্কিত
    বিধানাবলী আইনের দ্বারা নির্ধারিত হইবে৷

\subsection{৪ক। জাতির পিতার প্রতিকৃতি}
\label{sec:org0ba9a35}
(১) রাষ্ট্রপতির প্রতিকৃতি রাষ্ট্রপতি, প্রধানমন্ত্রী ও স্পীকারের কার্যালয় এবং
   বিদেশে অবস্থিত বাংলাদেশের দূতাবাস ও মিশনসমূহে সংরক্ষণ ও প্রদর্শন করিতে হইবে৷

(২) (১) দফার অতিরিক্ত কেবলমাত্র প্রধানমন্ত্রীর প্রতিকৃতি রাষ্ট্রপতি, প্রধানমন্ত্রী
    ও স্পীকারের কার্যালয় এবং সকল সরকারী ও আধা-সরকারী অফিস, স্বায়ত্তশাসিত
    প্রতিষ্ঠান, সংবিধিবদ্ধ সরকারী কর্তৃপক্ষের প্রধান ও শাখা কার্যালয়, সরকারী শিক্ষা
    প্রতিষ্ঠান, বিদেশে অবস্থিত বাংলাদেশের দূতাবাস ও মিশনসমূহে সংরক্ষণ ও প্রদর্শন
    করিতে হইবে৷

\subsection{৫৷ রাজধানী}
\label{sec:org9bfd39c}
(১) প্রজাতন্ত্রের রাজধানী ঢাকা৷

(২) রাজধানীর সীমানা আইনের দ্বারা নির্ধারিত হইবে৷

\subsection{৬৷ নাগরিকত্ব}
\label{sec:orgb4e8316}
(১) বাংলাদেশের নাগরিকত্ব আইনের দ্বারা নির্ধারিত ও নিয়ন্ত্রিত হইবে৷

(২) বাংলাদেশের নাগরিকগণ বাংলাদেশী বলিয়া পরিচিত হইবেন

\subsection{৭৷ সংবিধানের প্রাধান্য}
\label{sec:org9b5fbb2}
(১) প্রজাতন্ত্রের সকল ক্ষমতার মালিক জনগণ; এবং জনগণের পক্ষে সেই ক্ষমতার প্রয়োগ
    কেবল এই সংবিধানের অধীন ও কর্তৃত্বে কার্যকর হইবে৷

(২) জনগণের অভিপ্রায়ের পরম অভিব্যক্তিরূপে এই সংবিধান প্রজাতন্ত্রের সর্বোচ্চ আইন
    এবং অন্য কোন আইন যদি এই সংবিধানের সহিত অসমঞ্জস হয়, তাহা হইলে সেই আইনের
    যতখানি অসামঞ্জস্যপূর্ণ, ততখানি বাতিল হইবে৷

\subsection{৭ক। সংবিধান বাতিল, স্থগিতকরণ, ইত্যাদি অপরাধ}
\label{sec:orgaf3df3d}
(১) কোন ব্যক্তি শক্তি প্রদর্শন বা শক্তি প্রয়োগের মাধ্যমে বা অন্য কোন
   অসাংবিধানিক পন্থায় -

(ক) এই সংবিধান বা ইহার কোন অনুচ্ছেদ রদ, রহিত বা বাতিল বা স্থগিত করিলে
    কিংবা উহা করিবার জন্য উদ্যোগ গ্রহণ বা ষড়যন্ত্র করিলে ; কিংবা

(খ) এই সংবিধান বা ইহার কোন বিধানের প্রতি নাগরিকের আস্থা, বিশ্বাস বা প্রত্যয়
    পরাহত করিলে কিংবা উহা করিবার জন্য উদ্যোগ গ্রহণ বা ষড়যন্ত্র করিলে-

তাহার এই কার্য রাষ্ট্রদ্রোহিতা হইবে এবং ঐ ব্যক্তি রাষ্ট্রদ্রোহিতার অপরাধে দোষী
হইবে।

(২) কোন ব্যক্তি (১) দফায় বর্ণিত-

(ক) কোন কার্য করিতে সহযোগিতা বা উস্কানি প্রদান করিলে; কিংবা

(খ) কার্য অনুমোদন, মার্জনা, সমর্থন বা অনুসমর্থন করিলে-

তাহার এইরুপ কার্যও একই অপরাধ হইবে।

(৩) এই অনুচ্ছেদে বর্ণিত অপরাধে দোষী ব্যক্তি প্রচলিত আইনে অন্যান্য অপরাধের জন্য
    নির্ধারিত দণ্ডের মধ্যে সর্বোচ্চ দণ্ডে দণ্ডিত হইবে।

\subsection{৭খ। সংবিধানের মৌলিক বিধানাবলী সংশোধন অযোগ্য}
\label{sec:org584979b}
সংবিধানের ১৪২ অনুচ্ছেদে যাহা কিছুই থাকুক না কেন, সংবিধানের প্রস্তাবনা, প্রথম
ভাগের সকল অনুচ্ছেদ, দ্বিতীয় ভাগের সকল অনুচ্ছেদ, নবম-ক ভাগে বর্ণিত
অনুচ্ছেদসমূহের বিধানাবলী সাপেক্ষে তৃতীয় ভাগের সকল অনুচ্ছেদ এবং একাদশ ভাগের
১৫০ অনুচ্ছেদসহ সংবিধানের অন্যান্য মৌলিক কাঠামো সংক্রান্ত অনুচ্ছেদসমুহের
বিধানাবলী সংযোজন, পরিবর্তন, প্রতিস্থাপন, রহিতকরণ কিংবা অন্য কোন পন্থায়
সংশোধনের অযোগ্য হইবে।

\subsection{৮৷ মূলনীতিসমূহ}
\label{sec:orgee61473}
(১) সর্বশক্তিমান আল্লাহের উপর পূর্ণ আস্থা ও বিশ্বাস, জাতীয়তাবাদ, গণতন্ত্র এবং
    সমাজতন্ত্র অর্থাৎ অর্থনৈতিক ও সামাজিক সুবিচার-এই নীতিসমূহ এবং তৎসহ এই
    নীতিসমূহ হইতে উদ্ভূত এই ভাগে বর্ণিত অন্য সকল নীতি রাষ্ট্র পরিচালনার মূলনীতি
    বলিয়া পরিগণিত হইবে৷

(১ক) সর্বশক্তিমান আল্লাহের উপর পূর্ণ আস্থা ও বিশ্বাসই হইবে যাবতীয় কার্যাবলীর
ভিত্তি৷

(২) এই ভাগে বর্ণিত নীতিসমূহ বাংলাদেশ-পরিচালনার মূলসূত্র হইবে, আইন-প্রণয়নকালে
    রাষ্ট্র তাহা প্রয়োগ করিবেন, এই সংবিধান ও বাংলাদেশের অন্যান্য আইনের
    ব্যাখ্যাদানের ক্ষেত্রে তাহা নির্দেশক হইবে এবং তাহা রাষ্ট্র ও নাগরিকদের কার্যের
    ভিত্তি হইবে, তবে এই সকল নীতি আদালতের মাধ্যমে বলবৎযোগ্য হইবে না৷

\subsection{৯। জাতীয়তাবাদ}
\label{sec:org129570f}
ভাষাগত ও সংস্কৃতিগত একক সত্তাবিশিষ্ট যে বাঙালী জাতি ঐক্যবদ্ধ ও সংকল্পবদ্ধ
সংগ্রাম করিয়া জাতীয় মুক্তিযুদ্ধের মাধ্যমে বাংলাদেশের স্বাধীনতা ও সার্বভৌমত্ব
অর্জন করিয়াছেন, সেই বাঙালী জাতির ঐক্য ও সংহতি হইবে বাঙালী জাতীয়তাবাদের
ভিত্তি।

\subsection{১০। সমাজতন্ত্র ও শোষণমুক্তি}
\label{sec:orgd1686b5}
মানুষের উপর মানুষের শোষণ হইতে মুক্ত ন্যায়ানুগ ও সাম্যবাদী সমাজলাভ নিশ্চিত
করিবার উদ্দেশ্যে সমাজতান্ত্রিক অর্থনৈতিক ব্যবস্থা প্রতিষ্ঠা করা হইবে।

\subsection{১১৷ গণতন্ত্র ও মানবাধিকার}
\label{sec:orgbdef7cc}
 প্রজাতন্ত্র হইবে একটি গণতন্ত্র, যেখানে মৌলিক মানবাধিকার ও স্বাধীনতার
নিশ্চয়তা থাকিবে, মানবসত্তার মর্যাদা ও মূল্যের প্রতি শ্রদ্ধাবোধ নিশ্চিত হইবে
১০ * * * ১১ এবং প্রশাসনের সকল পর্যায়ে নির্বাচিত প্রতিনিধদের মাধ্যমে জনগণের
কার্যকর অংশগ্রহণ নিশ্চিত হইবে৷

\subsection{১২। ধর্ম নিরপেক্ষতা ও ধর্মীয় স্বাধীনতা}
\label{sec:org63510c2}
ধর্মনিরপেক্ষতা ও ধর্মীয় স্বাধীনতা – ধর্মনিরপেক্ষতার নীতিটি নির্মূলের মাধ্যমে উপলব্ধি করা হবে –
(ক) তার সমস্ত রূপে সাম্প্রদায়িকতা;
(খ) যে কোন ধর্মের পক্ষে রাষ্ট্রীয় মর্যাদা রাষ্ট্রকে প্রদত্ত;
(গ) রাজনৈতিক উদ্দেশ্যে ধর্মের অপব্যবহার;
(ঘ) নির্দিষ্ট ধর্ম অনুশীলনকারী ব্যক্তিদের বিরুদ্ধে কোনও বৈষম্য বা নিপীড়ন

\subsection{১৩৷ মালিকানার নীতি}
\label{sec:org6cb1cf7}
উৎপাদনযন্ত্র, উৎপাদনব্যবস্থা ও বন্টনপ্রণালীসমূহের মালিক বা নিয়ন্ত্রক হইবেন
জনগণ এবং এই উদ্দেশ্যে মালিকানা-ব্যবস্থা নিম্নরূপ হইবে:

(ক) রাষ্ট্রীয় মালিকানা, অর্থাৎ অর্থনৈতিক জীবনের প্রধান প্রধান ক্ষেএ লইয়া সুষ্ঠু
    ও গতিশীল রাষ্ট্রায়ত্ত সরকারী খাত সৃষ্টির মাধ্যমে জনগণের পক্ষে রাষ্ট্রের
    মালিকানা;

(খ) সমবায়ী মালিকানা, অর্থাৎ আইনের দ্বারা নির্ধারিত সীমার মধ্যে সমবায়সমূহের
    সদস্যদের পক্ষে সমবায়সমূহের মালিকানা; এবং

(গ) ব্যক্তিগত মালিকানা, অর্থাৎ আইনের দ্বারা নির্ধারিত সীমার মধ্যে ব্যক্তির
    মালিকানা৷

\subsection{১৪৷ কৃষক ও শ্রমিকের মুক্তি}
\label{sec:orgedc58dc}
রাষ্ট্রের অন্যতম মৌলিক দায়িত্ব হইবে মেহনতী মানুষকে-কৃষক ও শ্রমিকের-এবং জনগণের
অনগ্রসর অংশসমূহকে সকল প্রকার শোষণ হইতে মুক্তি দান করা৷

\subsection{১৫৷ মৌলিক প্রয়োজনের ব্যবস্থা}
\label{sec:orgf9a54c9}
রাষ্ট্রের অন্যতম মৌলিক দায়িত্ব হইবে পরিকল্পিত অর্থনৈতিক বিকাশের মাধ্যমে
উৎপাদনশক্তির ক্রমবৃদ্ধিসাধন এবং জনগণের জীবনযাত্রার বস্তুগত ও সংস্কৃতিগত মানের
দৃঢ় উন্নতিসাধন, যাহাতে নাগরিকদের জন্য নিম্নলিখিত বিষয়সমূহ অর্জন নিশ্চিত করা
যায়:

(ক) অন্ন, বস্ত্র, আশ্রয়, শিক্ষা ও চিকিৎসহ জীবনধারণের মৌলিক উপকরণের ব্যবস্থা;

(খ) কর্মের অধিকার, অর্থাৎ কর্মের গুণ ও পরিমাণ বিবেচনা করিয়া যুক্তিসঙ্গত মজুরীর
    বিনিময়ে কর্মসংস্থানের নিশ্চয়তার অধিকার;

(গ) যুক্তিসঙ্গত বিশ্রাম, বিনোদন ও অবকাশের অধিকার; এবং

(ঘ) সামাজিক নিরাপত্তার অধিকার, অর্থাৎ বেকারত্ব, ব্যাধি বা পঙ্গুত্বজনিত কিংবা
    বৈধব্য, মাতাপিতৃহীনতা বা বার্ধক্যজনিত কিংবা অনুরূপ অন্যান্য পরিস্থিতিজনিত
    আয়ত্তাতীত কারণে অভাবগ্রস্ত্মতার ক্ষেত্রে সরকারী সাহায্যলাভের অধিকার৷

\subsection{১৬৷ গ্রামীণ উন্নয়ন ও কৃষি বিপ্লব}
\label{sec:orgfd7c082}
নগর ও গ্রামাঞ্চলের জীবন যাত্রার মানের বৈষম্য ক্রমাগতভাবে দূর করিবার উদ্দেশ্যে
কৃষিবিপ্লবের বিকাশ, গ্রামাঞ্চলে বৈদ্যুতীকরণের ব্যবস্থা, কুটিরশিল্প ও অন্যান্য
শিল্পের বিকাশ এবং শিক্ষা, যোগাযোগ ব্যবস্থা ও জনস্বাস্থ্যের উন্নয়নের মাধ্যমে
গ্রামাঞ্চলের আমূল রূপান্তরসাধনের জন্য রাষ্ট্র কার্যকর ব্যবস্থা গ্রহণ করিবেন৷

\subsection{১৭৷ অবৈতনিক ও বাধ্যতামূলক শিক্ষা}
\label{sec:org23a9d92}
রাষ্ট্র
(ক) একই পদ্ধতির গণমুখী ও সার্বজনীন শিক্ষাব্যবস্থা প্রতিষ্ঠার জন্য এবং আইনের
    দ্বারা নির্ধারিত স্তর পর্যন্ত সকল বালক-বালিকাকে অবৈতনিক ও বাধ্যতামূলক
    শিক্ষাদানের জন্য;

(খ) সমাজের প্রয়োজনের সহিত শিক্ষাকে সঙ্গতিপূর্ণ করিবার জন্য এবং সেই প্রয়োজন
    সিদ্ধ করিবার উদ্দেশ্যে যথাযথ প্রশিক্ষণপ্রাপ্ত ও সদিচ্ছাপ্রণোদিত নাগরিক সৃষ্টির
    জন্য;

(গ) আইনের দ্বারা নির্ধারিত সময়ের মধ্যে নিরক্ষরতা দূর করিবার জন্য;
    কার্যকর ব্যবস্থা গ্রহণ করিবেন৷

\subsection{১৮৷ জনস্বাস্থ্য ও নৈতিকতা}
\label{sec:org00abb7f}
(১) জনগণের পুষ্টির স্তর-উন্নয়ন ও জনস্বাস্থ্যের উন্নতিসাধনকে রাষ্ট্র অন্যতম
    প্রাথমিক কর্তব্য বলিয়া গণ্য করিবেন এবং বিশেষতঃ আরোগ্যের প্রয়োজন কিংবা আইনের
    দ্বারা নির্দিষ্ট অন্যবিধ প্রয়োজন ব্যতীত মদ্য ও অন্যান্য মাদক পানীয় এবং
    স্বাস্থ্যহানিকর ভেষজের ব্যবহার নিষিদ্ধকরণের জন্য রাষ্ট্র কার্যকর ব্যবস্থা গ্রহণ
    করিবেন৷

(২) গণিকাবৃত্তি ও জুয়াখেলা নিরোধের জন্য রাষ্ট্র কার্যকর ব্যবস্থা গ্রহণ করিবেন৷

\subsection{১৮ক। পরিবেশ ও জীব-বৈচিত্র্য সংরক্ষণ ও উন্নয়ন}
\label{sec:org60939fe}
রাষ্ট্র বর্তমান ও ভবিষ্যৎ নাগরিকদের জন্য পরিবেশ সংরক্ষণ ও উন্নয়ন করিবেন এবং
প্রাকৃতিক সম্পদ, জীব-বৈচিত্র্য, জলাভূমি, বন ও বন্যপ্রাণির সংরক্ষণ ও নিরাপত্তা
বিধান করিবেন।

\subsection{১৯৷ সুযোগের সমতা}
\label{sec:orgfc819d7}
(১) সকল নাগরিকের জন্য সুযোগের সমতা নিশ্চিত করিতে রাষ্ট্র সচেষ্ট হইবেন৷

(২) মানুষে মানুষে সামাজিক ও অর্থনৈতিক অসাম্য বিলোপ করিবার জন্য, নাগরিকদের
    মধ্যে সম্পদের সুষম বন্টন নিশ্চিত করিবার জন্য এবং প্রজাতন্ত্রের সর্বত্র অর্থনৈতিক
    উন্নয়নের সমান স্তর অর্জনের উদ্দেশ্যে সুষম সুযোগ-সুবিধাদান নিশ্চিত করিবার জন্য
    রাষ্ট্র কার্যকর ব্যবস্থা গ্রহণ করিবেন৷

\subsection{২০৷ অধিকার ও কর্তব্যরূপে কর্ম}
\label{sec:orgf0c8bc5}
(১) কর্ম হইতেছে কর্মক্ষম প্রত্যেক নাগরিকের পক্ষে অধিকার, কর্তব্য ও সম্মানের
    বিষয়, এবং “প্রত্যেকের নিকট হইতে যোগ্যতানুসারে ও প্রত্যেককে কর্মানুযায়ী”-এই
    নীতির ভিত্তিতে প্রত্যেকে স্বীয় কর্মের জন্য পারিশ্রমিক লাভ করিবেন৷

(২) রাষ্ট্র এমন অবস্থা সৃষ্টির চেষ্টা করিবেন, যেখানে সাধারণ নীতি হিসাবে কোন
    ব্যক্তি অনুপার্জিত আয় ভোগ করিতে সমর্থ হইবেন না এবং যেখানে বুদ্ধিবৃত্তিমূলক ও
    কায়িক-সকল প্রকার শ্র্রম সৃষ্টিধর্মী প্রয়াসের ও মানবিক ব্যক্তিত্বের পূর্ণতর
    অভিব্যক্তিতে পরিণত হইবে৷

\subsection{২১৷ নাগরিক ও সরকারী কর্মচারীদের কর্তব্য}
\label{sec:org126f144}
(১) সংবিধান ও আইন মান্য করা, শৃঙ্খলা রক্ষা করা, নাগরিক দায়িত্ব পালন করা এবং
    জাতীয় সম্পত্তি রক্ষা করা প্রত্যেক নাগরিকের কর্তব্য৷

(২) সকল সময়ে জনগণের সেবা করিবার চেষ্টা করা প্রজাতন্ত্রের কর্মে নিযুক্ত প্রত্যেক
    ব্যক্তির কর্তব্য৷

\subsection{২২৷ নির্বাহী বিভাগ হইতে বিচার বিভাগের পৃথকীকরণ}
\label{sec:orgbff9e65}
রাষ্ট্রের নির্বাহী অঙ্গসমূহ হইতে বিচার বিভাগের পৃথকীকরণ রাষ্ট্র নিশ্চিত করিবেন৷

\subsection{২৩৷ জাতীয় সংস্কৃতি}
\label{sec:org03e9835}
রাষ্ট্র জনগণের সাংস্কৃতিক ঐতিহ্য ও উত্তরাধিকার রক্ষণের জন্য ব্যবস্থা গ্রহণ
করিবেন এবং জাতীয় ভাষা, সাহিত্য ও শিল্পকলাসমূহের এমন পরিপোষণ ও উন্নয়নের
ব্যবস্থা গ্রহণ করিবেন, যাহাতে সর্বস্তরের জনগণ জাতীয় সংস্কৃতির সমৃদ্ধিতে অবদান
রাখিবার ও অংশগ্রহণ করিবার সুযোগ লাভ করিতে পারেন৷

\subsection{২৩ক। উপজাতি, ক্ষুদ্র জাতিসত্তা, নৃ-গোষ্ঠী ও সম্প্রদায়ের সংস্কৃতি}
\label{sec:org8ce87dc}
রাষ্ট্র বিভিন্ন উপজাতি, ক্ষুদ্র জাতিসত্তা, নৃ-গোষ্ঠী ও সম্প্রদায়ের অনন্য
বৈশিষ্ট্যপূর্ণ আঞ্চলিক সংস্কৃতি এবং ঐতিহ্য সংরক্ষণ, উন্নয়ন ও বিকাশের ব্যবস্থা গ্রহণ
করিবেন।

\subsection{২৪৷ জাতীয় স্মৃতিনিদর্শন, প্রভৃতি}
\label{sec:org59f4505}
বিশেষ শৈল্পিক কিংবা ঐতিহাসিক গুরুত্বসম্পন্ন বা তাৎর্যমণ্ডিত স্মৃতিনিদর্শন, বস্তু
বা স্থানসমূহকে বিকৃতি, বিনাশ বা অপসারণ হইতে রক্ষা করিবার জন্য রাষ্ট্র ব্যবস্থা
গ্রহণ করিবেন৷

\subsection{২৫৷ আন্তর্জাতিক শান্তি, নিরাপত্তা ও সংহতির উন্নয়ন}
\label{sec:org9515b35}
(১) জাতীয় সার্বভৌমত্ব ও সমতার প্রতি শ্রদ্ধা, অন্যান্য রাষ্ট্রের অভ্যন্তরীণ বিষয়ে
    হস্তক্ষেপ না করা, আন্তর্জাতিক বিরোধের শান্তিপূর্ণ সমাধান এবং আন্তর্জাতিক আইনের
    ও জাতিসংঘের সনদে বর্ণিত নীতিসমূহের প্রতি শ্রদ্ধা-এই সকল নীতি হইবে রাষ্ট্রের
    আন্তর্জাতিক সম্পর্কের ভিত্তি এবং এই সকল নীতির ভিত্তিতে রাষ্ট্র

(ক) আন্তর্জাতিক সম্পর্কের ক্ষেত্রে শক্তিপ্রয়োগ পরিহার এবং সাধারণ ও সম্পূর্ণ
    নিরস্ত্রীকরণের জন্য চেষ্টা করিবেন;

(খ) প্রত্যেক জাতির স্বাধীন অভিপ্রায় অনুযায়ী পথ ও পন্থার মাধ্যমে অবাধে নিজস্ব
    সামাজিক, অর্থনৈতিক ও রাজনৈতিক ব্যবস্থা নির্ধারণ ও গঠনের অধিকার সমর্থন
    করিবেন; এবং

(গ) সাম্রাজ্যবাদ, ঔপনিবেশিকতাবাদ বা বর্ণবৈষম্যবাদের বিরুদ্ধে বিশ্বের সর্বত্র
    নিপীড়িত জনগণের ন্যায়সঙ্গত সংগ্রামকে সমর্থন করিবেন৷

(২) রাষ্ট্র ইসলামী সংহতির ভিত্তিতে মুসলিম দেশসমূহের মধ্যে ভ্রাতৃত্ব সম্পর্ক
    সংহত, সংরক্ষণ এবং জোরদার করিতে সচেষ্ট হইবেন৷

\subsection{২৬। মৌলিক অধিকারের সহিত অসমঞ্জস আইন বাতিল}
\label{sec:org24744ff}
(১) এই ভাগের বিধানাবলীর সহিত অসমঞ্জস সকল প্রচলিত আইন যতখানি অসামঞ্জস্যপূর্ণ,
    এই সংবিধান-প্রবর্তন হইতে সেই সকল আইনের ততখানি বাতিল হইয়া যাইবে।

(২) রাষ্ট্র এই ভাগের কোন বিধানের সহিত অসমঞ্জস কোন আইন প্রণয়ন করিবেন না এবং
    অনুরূপ কোন আইন প্রণীত হইলে তাহা এই ভাগের কোন বিধানের সহিত যতখানি
    অসামঞ্জস্যপূর্ণ, ততখানি বাতিল হইয়া যাইবে।

(৩) সংবিধানের ১৪২ অনুচ্ছেদের অধীন প্রণীত সংশোধনের ক্ষেত্রে এই অনুচ্ছেদের
   কোন কিছুই প্রযোজ্য হইবে না।

\subsection{২৭। আইনের দৃষ্টিতে সমতা}
\label{sec:org55966b1}
সকল নাগরিক আইনের দৃষ্টিতে সমান এবং আইনের সমান আশ্রয় লাভের অধিকারী।

\subsection{২৮। ধর্ম, প্রভৃতি কারণে বৈষম্য}
\label{sec:org58505dd}
(১) কেবল ধর্ম, গোষ্ঠী, বর্ণ, নারী-পুরুষভেদ বা জন্মস্থানের কারণে কোন নাগরিকের
    প্রতি রাষ্ট্র বৈষম্য প্রদর্শন করিবেন না।

২) রাষ্ট্র ও গণজীবনের সর্বস্তরে নারী পুরুষের সমান অধিকার লাভ করিবেন।

(৩) কেবল ধর্ম, গোষ্ঠী, বর্ণ, নারী-পুরুষভেদ বা জন্মস্থানের কারণে জনসাধারণের
    কোন বিনোদন বা বিশ্রামের স্থানে প্রবেশের কিংবা কোন শিক্ষা-প্রতিষ্ঠানে ভর্তির
    বিষয়ে কোন নাগরিককে কোনরূপ অক্ষমতা, বাধ্যবাধকতা, বাধা বা শর্তের অধীন করা
    যাইবে না।

(৪) নারী বা শিশুদের অনুকূলে কিংবা নাগরিকদের যে কোন অনগ্রসর অংশের অগ্রগতির
    জন্য বিশেষ বিধান-প্রণয়ন হইতে এই অনুচ্ছেদের কোন কিছুই রাষ্ট্রকে নিবৃত্ত করিবে
    না।

\subsection{২৯। সরকারী নিয়োগ-লাভে সুযোগের সমতা}
\label{sec:orgc97d53c}
(১) প্রজাতন্ত্রের কর্মে নিয়োগ বা পদ-লাভের ক্ষেত্রে সকল নাগরিকের জন্য সুযোগের
    সমতা থাকিবে।

(২) কেবল ধর্ম, গোষ্ঠী, বর্ণ, নারী-পুরুষভেদ বা জন্মস্থানের কারণে কোন নাগরিক
    প্রজাতন্ত্রের কর্মে নিয়োগ বা পদ-লাভের অযোগ্য হইবেন না কিংবা সেই ক্ষেত্রে
    তাঁহার প্রতি বৈষম্য প্রদর্শন করা যাইবে না।

(৩) এই অনুচ্ছেদের কোন কিছুই-

(ক) নাগরিকদের যে কোন অনগ্রসর অংশ যাহাতে প্রজাতন্ত্রের কর্মে উপযুক্ত
    প্রতিনিধিত্ব লাভ করিতে পারেন, সেই উদ্দেশ্যে তাঁহাদের অনুকূলে বিশেষ
    বিধান-প্রণয়ন করা হইতে,

(খ) কোন ধর্মীয় বা উপ-সমপ্রদায়গত প্রতিষ্ঠানে উক্ত ধর্মাবলম্বী বা উপ-সমপ্রদায়ভুক্ত
    ব্যক্তিদের জন্য নিয়োগ সংরক্ষণের বিধান-সংবলিত যে কোন আইন কার্যকর করা হইতে,

(গ) যে শ্রেণীর কর্মের বিশেষ প্রকৃতির জন্য তাহা নারী বা পুরুষের পক্ষে অনুপযোগী
    বিবেচিত হয়, সেইরূপ যে কোন শ্রেণীর নিয়োগ বা পদ যথাক্রমে পুরুষ বা নারীর জন্য
    সংরক্ষণ করা হইতে, রাষ্ট্রকে নিবৃত্ত করিবে না।

\subsection{৩০। বিদেশী, খেতাব, প্রভৃতি গ্রহণ নিষিদ্ধকরণ}
\label{sec:orgcd9580b}
রাষ্ট্রপতির পূর্বানুমোদন ব্যতীত কোন নাগরিক কোন বিদেশী রাষ্ট্রের নিকট হইতে কোন
উপাধি, খেতাব, সম্মান, পুরস্কার বা ভূষণ গ্রহণ করিবেন না।

\subsection{৩১। আইনের আশ্রয়-লাভের অধিকার}
\label{sec:org1ae36bd}
আইনের আশ্রয়লাভ এবং আইনানুযায়ী ও কেবল আইনানুযায়ী ব্যবহারলাভ যে কোন স্থানে
অবস্থানরত প্রত্যেক নাগরিকের এবং সাময়িকভাবে বাংলাদেশে অবস্থানরত অপরাপর
ব্যক্তির অবিচ্ছেদ্য অধিকার এবং বিশেষতঃ আইনানুযায়ী ব্যতীত এমন কোন ব্যবস্থা
গ্রহণ করা যাইবে না, যাহাতে কোন ব্যক্তির জীবন, স্বাধীনতা, দেহ, সুনাম বা
সম্পত্তির হানি ঘটে।

\subsection{৩২। জীবন ও ব্যক্তি-স্বাধীনতার অধিকাররক্ষণ}
\label{sec:org01859a4}
আইনানুযায়ী ব্যতীত জীবন ও ব্যক্তি-স্বাধীনতা হইতে কোন ব্যক্তিকে বঞ্চিত করা যাইবে না।

\subsection{৩৩। গ্রেপ্তার ও আটক সম্পর্কে রক্ষাকবচ}
\label{sec:orge62b46c}
(১) গ্রেপ্তারকৃত কোন ব্যক্তিকে যথাসম্ভব শীঘ্র গ্রেপ্তারের কারণ জ্ঞাপন না করিয়া
    প্রহরায় আটক রাখা যাইবে না এবং উক্ত ব্যক্তিকে তাঁহার মনোনীত আইনজীবীর সহিত
    পরামর্শের ও তাঁহার দ্বারা আত্মপক্ষ সমর্থনের অধিকার হইতে বঞ্চিত করা যাইবে না।

(২) গ্রেপ্তারকৃত ও প্রহরায় আটক প্রত্যেক ব্যক্তিকে নিকটতম ম্যাজিস্ট্রেটের সম্মুখে
    গ্রেপ্তারের চব্বিশ ঘন্টার মধ্যে গ্রেপ্তারের স্থান হইতে ম্যাজিস্ট্রেটের আদালতে
    আনয়নের জন্য প্রয়োজনীয় সময় ব্যতিরেকে) হাজির করা হইবে এবং ম্যাজিস্ট্রেটের আদেশ
    ব্যতীত তাঁহাকে তদতিরিক্তকাল প্রহরায় আটক রাখা যাইবে না।

(৩) এই অনুচ্ছেদের (১) ও (২) দফার কোন কিছুই সেই ব্যক্তির ক্ষেত্রে প্রযোজ্য হইবে
    না,

(ক) যিনি বর্তমান সময়ের জন্য বিদেশী শত্রু, অথবা

(খ) যাঁহাকে নিবর্তনমূলক আটকের বিধান-সংবলিত কোন আইনের অধীন গ্রেপ্তার করা
    হইয়াছে বা আটক করা হইয়াছে।

(৪) নিবর্তনমূলক আটকের বিধান-সংবলিত কোন আইন কোন ব্যক্তিকে ছয় মাসের অধিককাল
    আটক রাখিবার ক্ষমতা প্রদান করিবে না যদি সুপ্রীম কোর্টের বিচারক রহিয়াছেন বা
    ছিলেন কিংবা সুপ্রীম কোর্টের বিচারকপদে নিয়োগলাভের যোগ্যতা রাখেন, এইরূপ দুইজন
    এবং প্রজাতন্ত্রের কর্মে নিযুক্ত একজন প্রবীণ কর্মচারীর সমন্বয়ে গঠিত কোন
    উপদেষ্টা-পর্ষদ্ উক্ত ছয় মাস অতিবাহিত হইবার পূর্বে তাঁহাকে উপস্থিত হইয়া বক্তব্য
    পেশ করিবার সুযোগদানের পর রিপোর্ট প্রদান না করিয়া থাকেন যে, পর্ষদের মতে উক্ত
    ব্যক্তিকে তদতিরিক্তকাল আটক রাখিবার পর্যাপ্ত কারণ রহিয়াছে।

(৫) নির্বতনমূলক আটকের বিধান-সংবলিত কোন আইনের অধীন প্রদত্ত আদেশ অনুযায়ী কোন
    ব্যক্তিকে আটক করা হইলে আদেশদানকারী কর্তৃপক্ষ তাঁহাকে যথাসম্ভব শীঘ্র আদেশদানের
    কারণ জ্ঞাপন করিবেন এবং উক্ত আদেশের বিরুদ্ধে বক্তব্য-প্রকাশের জন্য তাঁহাকে যত
    সত্বর সম্ভব সুযোগদান করিবেন:

তবে শর্ত থাকে যে, আদেশদানকারী কর্তৃপক্ষের বিবেচনায় তথ্যাদি-প্রকাশ
জনস্বার্থবিরোধী বলিয়া মনে হইলে অনুরূপ কর্তৃপক্ষ তাহা প্রকাশে অস্বীকৃতি জ্ঞাপন
করিতে পারিবেন।

(৬) উপদেষ্টা-পর্ষদ কর্তৃক এই অনুচ্ছেদের (৪) দফার অধীন তদন্তের জন্য অনুসরণীয়
    পদ্ধতি সংসদ আইনের দ্বারা নির্ধারণ করিতে পারিবেন।

\subsection{৩৪। জবরদস্তি-শ্রম নিষিদ্ধকরণ}
\label{sec:org54ca948}
(১) সকল প্রকার জবরদস্তি-শ্রম নিষিদ্ধ; এবং এই বিধান কোনভাবে লংঘিত হইলে
    তাহা আইনতঃ দণ্ডনীয় অপরাধ বলিয়া গণ্য হইবে।

(২) এই অনুচ্ছেদের কোন কিছুই সেই সকল বাধ্যতামূলক শ্রমের ক্ষেত্রে প্রযোজ্য হইবে
    না, যেখানে

(ক) ফৌজদারী অপরাধের জন্য কোন ব্যক্তি আইনতঃ দণ্ডভোগ করিতেছেন; অথবা

(খ) জনগণের উদ্দেশ্যসাধনকল্পে আইনের দ্বারা তাহা আবশ্যক হইতেছে।

\subsection{৩৫। বিচার ও দন্ড সম্পর্কে রক্ষণ}
\label{sec:org5f8e137}
(১) অপরাধের দায়যুক্ত কার্যসংঘটনকালে বলবৎ ছিল, এইরূপ আইন ভঙ্গ করিবার অপরাধ
    ব্যতীত কোন ব্যক্তিকে দোষী সাব্যস্ত করা যাইবে না এবং অপরাধ-সংঘটনকালে বলবৎ
    সেই আইনবলে যে দণ্ড দেওয়া যাইতে পারিত, তাঁহাকে তাহার অধিক বা তাহা হইতে
    ভিন্ন দণ্ড দেওয়া যাইবে না।

(২) এক অপরাধের জন্য কোন ব্যক্তিকে একাধিকবার ফৌজদারীতে সোপর্দ ও দণ্ডিত করা
    যাইবে না।

(৩) ফৌজদারী অপরাধের দায়ে অভিযুক্ত প্রত্যেক ব্যক্তি আইনের দ্বারা প্রতিষ্ঠিত
    স্বাধীন ও নিরপেক্ষ আদালত বা ট্রাইব্যুনালে দ্রুত ও প্রকাশ্য বিচারলাভের অধিকারী
    হইবেন।

(৪) কোন অপরাধের দায়ে অভিযুক্ত ব্যক্তিকে নিজের বিরুদ্ধে সাক্ষ্য দিতে বাধ্য করা
    যাইবে না।

(৫) কোন ব্যক্তিকে যন্ত্রণা দেওয়া যাইবে না কিংবা নিষ্ঠুর, অমানুষিক বা লাঞ্ছনাকর
    দণ্ড দেওয়া যাইবে না কিংবা কাহারও সহিত অনুরূপ ব্যবহার করা যাইবে না।

(৬) প্রচলিত আইনে নির্দিষ্ট কোন দণ্ড বা বিচারপদ্ধতি সম্পর্কিত কোন বিধানের
    প্রয়োগকে এই অনুচ্ছেদের (৩) বা (৫) দফার কোন কিছুই প্রভাবিত করিবে না।

\subsection{৩৬। চলাফেরার স্বাধীনতা}
\label{sec:orgbcc33e5}
জনস্বার্থে আইনের দ্বারা আরোপিত যুক্তিসঙ্গত বাধা-নিষেধ সাপেক্ষে বাংলাদেশের
সর্বত্র অবাধ চলাফেরা, ইহার যে কোন স্থানে বসবাস ও বসতিস্থাপন এবং বাংলাদেশ
ত্যাগ ও বাংলাদেশে পুনঃপ্রবেশ করিবার অধিকার প্রত্যেক নাগরিকের থাকিবে।

\subsection{৩৭। সমাবেশের স্বাধীনতা}
\label{sec:org1e6c721}
জনশৃঙ্খলা বা জনস্বাস্থ্যের স্বার্থে আইনের দ্বারা আরোপিত যুক্তিসঙ্গত বাধা-নিষেধ
সাপেক্ষে শান্তিপূর্ণভাবে ও নিরস্ত্র অবস্থায় সমবেত হইবার এবং জনসভা ও
শোভাযাত্রায় যোগদান করিবার অধিকার প্রত্যেক নাগরিকের থাকিবে।

\subsection{৩৮। সংগঠনের স্বাধীনতা}
\label{sec:orgfd69128}
জনশৃঙ্খলা ও নৈতিকতার স্বার্থে আইনের দ্বারা আরোপিত যুক্তিসঙ্গত বাধা-নিষেধ
সাপেক্ষে সমিতি বা সংঘ গঠন করিবার অধিকার প্রত্যেক নাগরিকের থাকিবে:

\subsection{৩৯। চিন্তা ও বিবেকের স্বাধীনতা এবং বাক্-স্বাধীনতা}
\label{sec:orgb47f827}
(১) চিন্তা ও বিবেকের স্বাধীনতার নিশ্চয়তাদান করা হইল।

(২) রাষ্ট্রের নিরাপত্তা, বিদেশী রাষ্ট্রসমূহের সহিত বন্ধুত্বপূর্ণ সম্পর্ক, জনশৃঙ্খলা,
    শালীনতা ও নৈতিকতার স্বার্থে কিংবা আদালত-অবমাননা, মানহানি বা অপরাধ সংঘটনে
    প্ররোচনা সম্পর্কে আইনের দ্বারা আরোপিত যুক্তিসঙ্গত বাধা-নিষেধ সাপেক্ষে

(ক) প্রত্যেক নাগরিকের বাক্ ও ভাব প্রকাশের স্বাধীনতার অধিকারের, এবং

(খ) সংবাদ ক্ষেত্রের স্বাধীনতার,

নিশ্চয়তা দান করা হইল।

\subsection{৪০। পেশা বা বৃত্তির স্বাধীনতা}
\label{sec:orgc15efd4}
আইনের দ্বারা আরোপিত বাধা-নিষেধ সাপেক্ষে কোন পেশা বা বৃত্তি-গ্রহণের কিংবা
কারবার বা ব্যবসায়-পরিচালনার জন্য আইনের দ্বারা কোন যোগ্যতা নির্ধারিত হইয়া
থাকিলে অনুরূপ যোগ্যতাসম্পন্ন প্রত্যেক নাগরিকের যে কোন আইনসঙ্গত পেশা বা
বৃত্তি-গ্রহণের এবং যে কোন আইনসঙ্গত কারবার বা ব্যবসায়-পরিচালনার অধিকার
থাকিবে।

\subsection{৪১। ধর্মীয় স্বাধীনতা}
\label{sec:orga63066c}
(১) আইন, জনশৃঙ্খলা ও নৈতিকতা-সাপেক্ষে

(ক) প্রত্যেক নাগরিকের যে কোন ধর্ম অবলম্বন, পালন বা প্রচারের অধিকার রহিয়াছে;

(খ) প্রত্যেক ধর্মীয় সমপ্রদায় ও উপ-সমপ্রদায়ের নিজস্ব ধর্মীয় প্রতিষ্ঠানের স্থাপন,
    রক্ষণ ও ব্যবস্থাপনার অধিকার রহিয়াছে।

(২) কোন শিক্ষা-প্রতিষ্ঠানে যোগদানকারী কোন ব্যক্তির নিজস্ব ধর্ম-সংক্রান্ত না
    হইলে তাঁহাকে কোন ধর্মীয় শিক্ষাগ্রহণ কিংবা কোন ধর্মীয় অনুষ্ঠান বা উপাসনায়
    অংশগ্রহণ বা যোগদান করিতে হইবে না।

\subsection{৪২। সম্পত্তির অধিকার}
\label{sec:org97f0da0}
(১) আইনের দ্বারা আরোপিত বাধা-নিষেধ সাপেক্ষে প্রত্যেক নাগরিকের সম্পত্তি অর্জন,
    ধারণ, হস্তান্তর বা অন্যভাবে বিলি-ব্যবস্থা করিবার অধিকার থাকিবে এবং আইনের
    কর্তৃত্ব ব্যতীত কোন সম্পত্তি বাধ্যতামূলকভাবে গ্রহণ, রাষ্ট্রায়ত্ত বা দখল করা যাইবে
    না।

(২) এই অনুচ্ছেদের (১) দফার অধীন প্রণীত আইনে ক্ষতিপূরণসহ বাধ্যতামূলকভাবে গ্রহণ,
    রাষ্ট্রায়ত্তকরণ বা দখলের বিধান করা হইবে এবং ক্ষতিপূরণের পরিমাণ নির্ধারণ,
    কিংবা ক্ষতিপূরণ নির্ণয় বা প্রদানের নীতি ও পদ্ধতি নির্দিষ্ট করা হইবে, তবে অনুরূপ
    কোন আইনে ক্ষতিপূরণের বিধান অপর্যাপ্ত হইয়াছে বলিয়া সেই আইন সম্পর্কে কোন
    আদালতে কোন প্রশ্ন উত্থাপন করা যাইবে না।

(৩) ১৯৭৭ সালের ফরমানসমূহ (সংশোধন) আদেশ, ১৯৭৭ (১৯৭৭ সালের ১ নং ফরমানসমূহ
    আদেশ) প্রবর্তনের পূর্বে প্রণীত কোন আইনের প্রয়োগকে, যতদূর তাহা ক্ষতিপূরণ ব্যতীত
    কোন সম্পত্তি বাধ্যতামূলকভাবে গ্রহণ, রাষ্ট্রায়ত্তকরণ বা দখলের সহিত সম্পর্কিত, এই
    অনুচ্ছেদের কোন কিছুই প্রভাবিত করিবে না।

\subsection{৪৩। গৃহ ও যোগাযোগের রক্ষণ}
\label{sec:orgfca8946}
রাষ্ট্রের নিরাপত্তা, জনশৃঙ্খলা, জনসাধারণের নৈতিকতা বা জনস্বাস্থ্যের স্বার্থে
আইনের দ্বারা আরোপিত যুক্তিসঙ্গত বাধা-নিষেধ সাপেক্ষে প্রত্যেক নাগরিকের-

(ক) প্রবেশ, তল্লাশী ও আটক হইতে স্বীয় গৃহে নিরাপত্তালাভের অধিকার থাকিবে; এবং

(খ) চিঠিপত্রের ও যোগাযোগের অন্যান্য উপায়ের গোপনীয়তা রক্ষার অধিকার থাকিবে।
\subsection{৪৪। মৌলিক অধিকার বলবৎকরণ}
\label{sec:org4e31240}
(১) এই ভাগে প্রদত্ত অধিকারসমূহ বলবৎ করিবার জন্য এই সংবিধানের ১০২ অনুচ্ছেদের
    (১) দফা অনুযায়ী হাইকোর্ট বিভাগের নিকট মামলা রুজু করিবার অধিকারের নিশ্চয়তা
    দান করা হইল।

(২) এই সংবিধানের ১০২ অনুচ্ছেদের অধীন হাইকোর্ট বিভাগের ক্ষমতার হানি না
    ঘটাইয়া সংসদ আইনের দ্বারা অন্য কোন আদালতকে তাহার এখতিয়ারের স্থানীয় সীমার
    মধ্যে ঐ সকল বা উহার যে কোন ক্ষমতা প্রয়োগের ক্ষমতা দান করিতে পারিবেন।

\subsection{৪৫। শৃঙ্খলামূলক আইনের ক্ষেত্রে অধিকারের পরিবর্তন}
\label{sec:orgdf371ac}
কোন শৃঙ্খলা-বাহিনীর সদস্য-সম্পর্কিত কোন শৃঙ্খলামূলক আইনের যে কোন বিধান উক্ত
সদস্যদের যথাযথ কর্তব্যপালন বা উক্ত বাহিনীতে শৃঙ্খলারক্ষা নিশ্চিত করিবার
উদ্দেশ্যে প্রণীত বিধান বলিয়া অনুরূপ বিধানের ক্ষেত্রে এই ভাগের কোন কিছুই প্রযোজ্য
হইবে না।

\subsection{৪৬। দায়মুক্তি-বিধানের ক্ষমতা}
\label{sec:org355bfe0}
এই ভাগের পূর্ববর্ণিত বিধানাবলীতে যাহা বলা হইয়াছে, তাহা সত্ত্বেও প্রজাতন্ত্রের
কর্মে নিযুক্ত কোন ব্যক্তি বা অন্য কোন ব্যক্তি জাতীয় মুক্তি-সংগ্রামের প্রয়োজনে
কিংবা বাংলাদেশের রাষ্ট্রীয় সীমানার মধ্যে যে কোন অঞ্চলে শৃঙ্খলা-রক্ষা বা
পুনর্বহালের প্রয়োজনে কোন কার্য করিয়া থাকিলে সংসদ আইনের দ্বারা সেই ব্যক্তিকে
দায়মুক্ত করিতে পারিবেন কিংবা ঐ অঞ্চলে প্রদত্ত কোন দণ্ডাদেশ, দণ্ড বা
বাজেয়াপ্তির আদেশকে কিংবা অন্য কোন কার্যকে বৈধ করিয়া লইতে পারিবেন।

\subsection{৪৭। কতিপয় আইনের হেফাজত}
\label{sec:orge3478a7}
(১) নিম্নলিখিত যে কোন বিষয়ের বিধান-সংবলিত কোন আইনে (প্রচলিত আইনের ক্ষেত্রে
    সংশোধনীর মাধ্যমে) সংসদ যদি স্পষ্টরূপে ঘোষণা করেন যে, এই সংবিধানের দ্বিতীয়
    ভাগে বর্ণিত রাষ্ট্র পরিচালনার মূলনীতিসমূহের কোন একটিকে কার্যকর করিবার জন্য
    অনুরূপ বিধান করা হইল, তাহা হইলে অনুরূপ আইন এইভাগে নিশ্চয়কৃত কোন অধিকারের
    সহিত অসমঞ্জস কিংবা অনুরূপ অধিকার হরণ বা খর্ব করিতেছে, এই কারণে বাতিল বলিয়া
    গণ্য হইবে না:

(ক) কোন সম্পত্তি বাধ্যতামূলকভাবে গ্রহণ, রাষ্ট্রায়ত্তকরণ বা দখল কিংবা
    সাময়িকভাবে বা স্থায়ীভাবে কোন সম্পত্তির নিয়ন্ত্রণ বা ব্যবস্থাপনা;

(খ) বাণিজ্যিক বা অন্যবিধ উদ্যোগসম্পন্ন একাধিক প্রতিষ্ঠানের বাধ্যতামূলক
    সংযুক্তকরণ;

(গ) অনুরূপ যে কোন প্রতিষ্ঠানের পরিচালক, ব্যবস্থাপক, এজেন্ট ও কর্মচারীদের
    অধিকার এবং (যে কোন প্রকারের) শেয়ার ও স্টকের মালিকদের ভোটাধিকার বিলোপ,
    পরিবর্তন, সীমিতকরণ বা নিয়ন্ত্রণ;

(ঘ) খনিজদ্রব্য বা খনিজ তৈল-অনুসন্ধান বা লাভের অধিকার বিলোপ, পরিবর্তন,
    সীমিতকরণ বা নিয়ন্ত্রণ;

(ঙ) অন্যান্য ব্যক্তিকে অংশতঃ বা সম্পূর্ণতঃ পরিহার করিয়া সরকার কর্তৃক বা সরকারের
    নিজস্ব, নিয়ন্ত্রণাধীন বা ব্যবস্থাপনাধীন কোন সংস্থা কর্তৃক যে কোন কারবার,
    ব্যবসায়, শিল্প বা কর্মবিভাগ-চালনা; অথবা

(চ) যে কোন সম্পত্তির স্বত্ব কিংবা পেশা, বৃত্তি, কারবার বা ব্যবসায়-সংক্রান্ত যে
    কোন অধিকার কিংবা কোন সংবিধিবদ্ধ সরকারী প্রতিষ্ঠান বা কোন বাণিজ্যিক বা
    শিল্পগত উদ্যোগের মালিক বা কর্মচারীদের অধিকার বিলোপ, পরিবর্তন, সীমিতকরণ বা
    নিয়ন্ত্রণ।

(২) এই সংবিধানে যাহা বলা হইয়াছে, তাহা সত্ত্বেও প্রথম তফসিলে বর্ণিত আইনসমূহ
    (অনুরূপ আইনের কোন সংশোধনীসহ) পূর্ণভাবে বলবৎ ও কার্যকর হইতে থাকিবে এবং অনুরূপ
    যে কোন আইনের কোন বিধান কিংবা অনুরূপ কোন আইনের কর্তত্বে যাহা করা হইয়াছে বা
    করা হয় নাই, তাহা এই সংবিধানের কোন বিধানের সহিত অসমঞ্জস বা তাহার
    পরিপন্থী, এই কারণে বাতিল বা বেআইনী বলিয়া গণ্য হইবে না:

 তবে শর্ত থাকে যে, এই অনুচ্ছেদের কোন কিছুই অনুরূপ কোন আইনকে সংশোধন,
পরিবর্তন বা বাতিল করা হইতে নিবৃত্ত করিবে না।

(৩) এই সংবিধানে যাহা বলা হইয়াছে, তাহা সত্ত্বেও গণহত্যাজনিত অপরাধ,
    মানবতাবিরোধী অপরাধ বা যুদ্ধাপরাধ এবং আন্তর্জাতিক আইনের অধীন অন্যান্য অপরাধের
    জন্য কোন সশস্ত্র বাহিনী বা প্রতিরৰা বাহিনী বা সহায়ক বাহিনীর সদস্য কিংবা
    যুদ্ধবন্দীকে আটক, ফৌজদারীতে সোপর্দ কিংবা দ-দান করিবার বিধান-সংবলিত কোন
    আইন বা আইনের বিধান এই সংবিধানের কোন বিধানের সহিত অসমঞ্জস বা তাহার
    পরিপন্থী, এই কারণে বাতিল বা বেআইনী বলিয়া গণ্য হইবে না কিংবা কখনও বাতিল
    বা বেআইনী হইয়াছে বলিয়া গণ্য হইবে না।

\subsection{৪৭ক। সংবিধানের কতিপয় বিধানের অপ্রযোজ্যতা}
\label{sec:org5f24d36}
(১) যে ব্যক্তির ক্ষেত্রে এই সংবিধানের ৪৭ অনুচ্ছেদের (৩) দফায় বর্ণিত কোন আইন
    প্রযোজ্য হয়, সেই ব্যক্তির ক্ষেত্রে এই সংবিধানের ৩১ অনুচ্ছেদ, ৩৫ অনুচ্ছেদের (১) ও
    (৩) দফা এবং ৪৪ অনুচ্ছেদের অধীন নিশ্চয়কৃত অধিকারসমূহ প্রযোজ্য হইবে না।

(২) এই সংবিধানে যাহা বলা হইয়াছে, তাহা সত্ত্বেও যে ব্যক্তির ক্ষেত্রে এই
    সংবিধানের ৪৭ অনুচ্ছেদের (৩) দফায় বর্ণিত কোন আইন প্রযোজ্য হয়, এই সংবিধানের
    অধীন কোন প্রতিকারের জন্য সুপ্রীম কোর্টে আবেদন করিবার কোন অধিকার সেই ব্যক্তির
    থাকিবে না।

\subsection{৪৮। রাষ্ট্রপতি}
\label{sec:orgf3c4238}
(১) বাংলাদেশের একজন রাষ্ট্রপতি থাকিবেন, যিনি আইন অনুযায়ী সংসদ-সদস্যগণ
     কর্তৃক নির্বাচিত হইবেন।

(২) রাষ্ট্রপ্রধানরূপে রাষ্ট্রপতি রাষ্ট্রের অন্য সকল ব্যক্তির ঊর্ধ্বে স্থান লাভ
    করিবেন এবং এই সংবিধান ও অন্য কোন আইনের দ্বারা তাঁহাকে প্রদত্ত ও তাঁহার উপর
    অর্পিত সকল ক্ষমতা প্রয়োগ ও কর্তব্য পালন করিবেন।

(৩) এই সংবিধানের ৫৬ অনুচ্ছেদের (৩) দফা অনুসারে কেবল প্রধানমন্ত্রী ও ৯৫
    অনুচ্ছেদের (১) দফা অনুসারে প্রধান বিচারপতি নিয়োগের ক্ষেত্র ব্যতীত রাষ্টপতি
    তাঁহার অন্য সকল দায়িত্ব পালনে প্রধানমন্ত্রীর পরামর্শ অনুযায়ী কার্য করিবেন:

তবে শর্ত থাকে যে, প্রধানমন্ত্রী রাষ্ট্রপতিকে আদৌ কোন পরামর্শদান করিয়াছেন কি
না এবং করিয়া থাকিলে কি পরামর্শ দান করিয়াছেন, কোন আদালত সেই সম্পর্কে কোন
প্রশ্নের তদন্ত করিতে পারিবেন না।

(৪) কোন ব্যক্তি রাষ্ট্রপতি নির্বাচিত হইবার যোগ্য হইবেন না, যদি তিনি-

(ক) পঁয়ত্রিশ বৎসরের কম বয়স্ক হন; অথবা

(খ) সংসদ-সদস্য নির্বাচিত হইবার যোগ্য না হন; অথবা

(গ) কখনও এই সংবিধানের অধীন অভিশংসন দ্বারা রাষ্ট্রপতির পদ হইতে অপসারিত
    হইয়া থাকেন।

(৫) প্রধানমন্ত্রী রাষ্ট্রীয় ও পররাষ্ট্রীয় নীতি সংক্রান্ত বিষয়াদি সম্পর্কে
    রাষ্ট্রপতিকে অবহিত রাখিবেন এবং রাষ্ট্রপতি অনুরোধ করিলে যে কোন বিষয়
    মন্ত্রিসভায় বিবেচনার জন্য পেশ করিবেন।

\subsection{৪৯। ক্ষমা প্রদর্শনের অধিকার}
\label{sec:org0e6ad71}
কোন আদালত, ট্রাইব্যুনাল বা অন্য কোন কর্তৃপক্ষ কর্তৃক প্রদত্ত যে কোন দণ্ডের
মার্জনা, বিলম্বন ও বিরাম মঞ্জুর করিবার এবং যে কোন দণ্ড মওকুফ, স্থগিত বা হ্রাস
করিবার ক্ষমতা রাষ্ট্রপতির থাকিবে।

\subsection{৫০। রাষ্ট্রপতি-পদের মেয়াদ}
\label{sec:orgb65fbc1}
(১) এই সংবিধানের বিধানাবলী সাপেক্ষে রাষ্ট্রপতি কার্যভার গ্রহণের তারিখ হইতে
    পাঁচ বৎসরের মেয়াদে তাঁহার পদে অধিষ্ঠিত থাকিবেন:

তবে শর্ত থাকে যে, রাষ্ট্রপতির পদের মেয়াদ শেষ হওয়া সত্ত্বেও তাঁহার
উত্তরাধিকারী কার্যভার গ্রহণ না করা পর্যন্ত তিনি স্বীয় পদে বহাল থাকিবেন।

(২) একাদিক্রমে হউক বা না হউক-দুই মেয়াদের অধিক রাষ্ট্রপতির পদে কোন ব্যক্তি
    অধিষ্ঠিত থাকিবেন না।

(৩) স্পীকারের উদ্দেশ্যে স্বাক্ষরযুক্ত পত্রযোগে রাষ্ট্রপতি স্বীয় পদ ত্যাগ করিতে
    পারিবেন।

(৪) রাষ্ট্রপতি তাঁহার কার্যভারকালে সংসদ-সদস্য নির্বাচিত হইবার যোগ্য হইবেন না
    এবং কোন সংসদ-সদস্য রাষ্ট্রপতি নির্বাচিত হইলে রাষ্ট্রপতিরূপে তাঁহার কার্যভার
    গ্রহণের দিনে সংসদে তাঁহার আসন শূন্য হইবে।
\subsection{৫১। রাষ্ট্রপতির দায়মুক্তি}
\label{sec:org88e1e96}

(১) এই সংবিধানের ৫২ অনুচ্ছেদের হানি না ঘটাইয়া বিধান করা হইতেছে যে,
    রাষ্ট্রপতি তাঁহার দায়িত্ব পালন করিতে গিয়া কিংবা অনুরূপ বিবেচনায় কোন কার্য
    করিয়া থাকিলে বা না করিয়া থাকিলে সেইজন্য তাঁহাকে কোন আদালতে জবাবদিহি
    করিতে হইবে না, তবে এই দফা সরকারের বিরুদ্ধে কার্যধারা গ্রহণে কোন ব্যক্তির
    অধিকার ক্ষুণ্ন করিবে না।

(২) রাষ্ট্রপতির কার্যভারকালে তাঁহার বিরুদ্ধে কোন আদালতে কোন প্রকার ফৌজদারী
    কার্যধারা দায়ের করা বা চালু রাখা যাইবে না এবং তাঁহার গ্রেফতার বা কারাবাসের
    জন্য কোন আদালত হইতে পরোয়ানা জারী করা যাইবে না।

\subsection{৫২। রাষ্ট্রপতির অভিশংসন}
\label{sec:orgde8f83a}
(১) এই সংবিধান লংঘন বা গুরুতর অসদাচরণের অভিযোগে রাষ্ট্রপতিকে অভিশংসিত করা
    যাইতে পারিবে; ইহার জন্য সংসদের মোট সদস্যের সংখ্যাগরিষ্ঠ অংশের স্বাক্ষরে
    অনুরূপ অভিযোগের বিবরণ লিপিবদ্ধ করিয়া একটি প্রস্তাবের নোটিশ স্পীকারের নিকট
    প্রদান করিতে হইবে; স্পীকারের নিকট অনুরূপ নোটিশ প্রদানের দিন হইতে চৌদ্দ
    দিনের পূর্বে বা ত্রিশ দিনের পর এই প্রস্তাব আলোচিত হইতে পারিবে না এবং সংসদ
    অধিবেশনরত না থাকিলে স্পীকার অবিলম্বে সংসদ আহবান করিবেন।

(২) এই অনুচ্ছেদের অধীন কোন অভিযোগ তদন্তের জন্য সংসদ কর্তৃক নিযুক্ত বা আখ্যায়িত
    কোন আদালত, ট্রাইব্যুনাল বা কর্তৃপক্ষের নিকট সংসদ রাষ্ট্রপতির আচরণ গোচর করিতে
    পারিবেন।

(৩) অভিযোগ বিবেচনাকালে রাষ্ট্রপতির উপস্থিত থাকিবার এবং প্রতিনিধি প্রেরণের
    অধিকার থাকিবে।

(৪) অভিযোগ বিবেচনার পর মোট সদস্য-সংখ্যার অন্যূন দুই-তৃতীয়াংশ ভোটে অভিযোগ
    যথার্থ বলিয়া ঘোষণা করিয়া সংসদ কোন প্রস্তাব গ্রহণ করিলে প্রস্তাব গৃহীত হইবার
    তারিখে রাষ্ট্রপতির পদ শূন্য হইবে।

(৫) এই সংবিধানের ৫৪ অনুচ্ছেদ অনুযায়ী স্পীকার কর্তৃক রাষ্ট্রপতির দায়িত্ব
    পালনকালে এই অনুচ্ছেদের বিধানাবলী এই পরিবর্তন সাপেক্ষে প্রযোজ্য হইবে যে, এই
    অনুচ্ছেদের (১) দফায় স্পীকারের উল্লেখ ডেপুটি স্পীকারের উল্লেখ বলিয়া গণ্য হইবে
    এবং (৪) দফায় রাষ্ট্রপতির পদ শূন্য হইবার উল্লেখ স্পীকারের পদ শূন্য হইবার উল্লেখ
    বলিয়া গণ্য হইবে; এবং (৪) দফায় বর্ণিত কোন প্রস্তাব গৃহীত হইলে স্পীকার
    রাষ্ট্রপতির দায়িত্ব পালনে বিরত হইবেন।

\subsection{৫৩। অসামর্থ্যের কারণে রাষ্ট্রপতির অপসারণ}
\label{sec:org19b33d9}
(১) শারীরিক বা মানসিক অসামর্থ্যের কারণে রাষ্ট্রপতিকে তাঁহার পদ হইতে অপসারিত
    করা যাইতে পারিবে; ইহার জন্য সংসদের মোট সদস্যের সংখ্যাগরিষ্ঠ অংশের স্বাক্ষরে
    কথিত অসামর্থ্যের বিবরণ লিপিবদ্ধ করিয়া একটি প্রস্তাবের নোটিশ স্পীকারের নিকট
    প্রদান করিতে হইবে।

(২) সংসদ অধিবেশনরত না থাকিলে নোটিশ প্রাপ্তিমাত্র স্পীকার সংসদের অধিবেশন
    আহবান করিবেন এবং একটি চিকিৎসা-পর্ষদ (অতঃপর এই অনুচ্ছেদে “পর্ষদ” বলিয়া
    অভিহিত) গঠনের প্রস্তাব আহ্বান করিবেন এবং প্রয়োজনীয় প্রস্তাব উত্থাপিত ও গৃহীত
    হইবার পর স্পীকার তৎক্ষণাৎ উক্ত নোটিশের একটি প্রতিলিপি রাষ্ট্রপতির নিকট
    প্রেরণের ব্যবস্থা করিবেন এবং তাঁহার সহিত এই মর্মে স্বাক্ষরযুক্ত অনুরোধ জ্ঞাপন
    করিবেন যে, অনুরূপ অনুরোধ জ্ঞাপনের তারিখ হইতে দশ দিনের মধ্যে রাষ্ট্রপতি যেন
    পর্ষদের নিকট পরীক্ষিত হইবার জন্য উপস্থিত হন।
(৩) অপসারণের জন্য প্রস্তাবের নোটিশ স্পীকারের নিকট প্রদানের পর হইতে চৌদ্দ
    দিনের পূর্বে বা ত্রিশ দিনের পর প্রস্তাবটি ভোটে দেওয়া যাইবে না, এবং অনুরূপ
    মেয়াদের মধ্যে প্রস্তাবটি উত্থাপনের জন্য পুনরায় সংসদ আহ্বানের প্রয়োজন হইলে
    স্পীকার সংসদ আহ্বান করিবেন।

(৪) প্রস্তাবটি বিবেচিত হইবার কালে রাষ্ট্রপতির উপস্থিত থাকিবার এবং প্রতিনিধি
    প্রেরণের অধিকার থাকিবে।

(৫) প্রস্তাবটি সংসদে উত্থাপনের পূর্বে রাষ্ট্রপতি পর্ষদের দ্বারা পরীক্ষিত হইবার
    জন্য উপস্থিত না হইয়া থাকিলে প্রস্তাবটি ভোটে দেওয়া যাইতে পারিবে এবং সংসদের
    মোট সদস্য-সংখ্যার অন্যূন দুই-তৃতীয়াংশ ভোটে তাহা গৃহীত হইলে প্রস্তাবটি গৃহীত
    হইবার তারিখে রাষ্ট্রপতির পদ শূন্য হইবে।

(৬) অপসারণের জন্য প্রস্তাবটি সংসদে উস্থাপিত হইবার পূর্বে রাষ্ট্রপতি পর্ষদের
    নিকট পরীক্ষিত হইবার জন্য উপস্থিত হইয়া থাকিলে সংসদের নিকট পর্ষদের মতামত পেশ
    করিবার সুযোগ না দেওয়া পর্যন্ত প্রস্তাবটি ভোটে দেওয়া যাইবে না।

(৭) সংসদ কর্তৃক প্রস্তাবটি ও পর্ষদের রিপোর্ট (যাহা এই অনুচ্ছেদের (২) দফা
    অনুসারে পরীক্ষার সাত দিনের মধ্যে দাখিল করা হইবে এবং অনুরূপভাবে দাখিল না করা
    হইলে তাহা বিবেচনার প্রয়োজন হইবে না) বিবেচিত হইবার পর সংসদের মোট
    সদস্য-সংখ্যার অন্যূন দুই-তৃতীয়াংশ ভোটে প্রস্তাবটি গৃহীত হইলে তাহা গৃহীত হইবার
    তারিখে রাষ্ট্রপতি পদ শূন্য হইবে।

\subsection{৫৪। অনুপস্থিতি প্রভৃতির-কালে রাষ্ট্রপতি-পদে স্পীকার}
\label{sec:org0da3fa4}
রাষ্ট্রপতির পদ শূন্য হইলে কিংবা অনুপস্থিতি, অসুস্থতা বা অন্য কোন কারণে রাষ্ট্রপতি
দায়িত্ব পালনে অসমর্থ হইলে ক্ষেত্রমত রাষ্ট্রপতি নির্বাচিত না হওয়া পর্যন্ত কিংবা
রাষ্ট্রপতি পুনরায় স্বীয় কার্যভার গ্রহণ না করা পর্যন্ত স্পীকার রাষ্ট্রপতির দায়িত্ব
পালন করিবেন।

\subsection{৫৫। মন্ত্রিসভা}
\label{sec:orgb5ba1d5}
(১) প্রধানমন্ত্রীর নেতৃত্বে বাংলাদেশের একটি মন্ত্রিসভা থাকিবে এবং প্রধানমন্ত্রী
    ও সময়ে সময়ে তিনি যেরূপ স্থির করিবেন, সেইরূপ অন্যান্য মন্ত্রী লইয়া এই মন্ত্রিসভা
    গঠিত হইবে।

(২) প্রধানমন্ত্রী কর্তৃক বা তাঁহার কর্তত্বে এই সংবিধান-অনুযায়ী প্রজাতন্ত্রের
    নির্বাহী ক্ষমতা প্রযুক্ত হইবে।

(৩) মন্ত্রিসভা যৌথভাবে সংসদের নিকট দায়ী থাকিবেন।

(৪) সরকারের সকল নির্বাহী ব্যবস্থা রাষ্ট্রপতির নামে গৃহীত হইয়াছে বলিয়া প্রকাশ
    করা হইবে।

(৫) রাষ্ট্রপতির নামে প্রণীত আদেশসমূহ ও অন্যান্য চুক্তিপত্র কিরূপে সত্যায়িত বা
    প্রমাণীকৃত হইবে, রাষ্ট্রপতি তাহা বিধিসমূহ-দ্বারা নির্ধারণ করিবেন এবং অনুরূপভাবে
    সত্যায়িত বা প্রমাণীকৃত কোন আদেশ বা চুক্তিপত্র যথাযথভাবে প্রণীত বা সম্পাদিত হয়
    নাই বলিয়া তাহার বৈধতা সম্পর্কে কোন আদালতে প্রশ্ন উত্থাপন করা যাইবে না।

(৬) রাষ্ট্রপতি সরকারী কার্যাবলী বন্টন ও পরিচালনার জন্য বিধিসমূহ প্রণয়ন
    করিবেন।

\subsection{৫৬। মন্ত্রিগণ}
\label{sec:org737c489}
(১) একজন প্রধানমন্ত্রী থাকিবেন এবং প্রধানমন্ত্রী যেরূপ নির্ধারণ করিবেন, সেইরূপ
    অন্যান্য মন্ত্রী, প্রতিমন্ত্রী ও উপ-মন্ত্রী থাকিবেন।

(২) প্রধানমন্ত্রী ও অন্যান্য মন্ত্রী, প্রতিমন্ত্রী ও উপ-মন্ত্রীদিগকে রাষ্ট্রপতি
    নিয়োগ দান করিবেন:

তবে শর্ত থাকে যে, তাঁহাদের সংখ্যার অন্যূন নয়-দশমাংশ সংসদ-সদস্যগণের মধ্য হইতে
নিযুক্ত হইবেন এবং অনধিক এক-দশমাংশ সংসদ-সদস্য নির্বাচিত হইবার যোগ্য
ব্যক্তিগণের মধ্য হইতে মনোনীত হইতে পারিবেন।

(৩) যে সংসদ-সদস্য সংসদের সংখ্যাগরিষ্ঠ সদস্যের আস্থাভাজন বলিয়া রাষ্ট্রপতির
    নিকট প্রতীয়মান হইবেন, রাষ্ট্রপতি তাঁহাকে প্রধানমন্ত্রী নিয়োগ করিবেন।

(৪) সংসদ ভাংগিয়া যাওয়া এবং সংসদ-সদস্যদের অব্যবহিত পরবর্তী সাধারণ নির্বাচন
    অনুষ্ঠানের মধ্যবর্তীকালে এই অনুচ্ছেদের (২) বা (৩) দফার অধীন নিয়োগ দানের
    প্রয়োজন দেখা দিলে সংসদ ভাংগিয়া যাইবার অব্যবহিত পূর্বে যাঁহারা সংসদ-সদস্য
    ছিলেন, এই দফার উদ্দেশ্যসাধনকল্পে তাঁহারা সদস্যরূপে বহাল রহিয়াছেন বলিয়া গণ্য
    হইবেন।

\subsection{৫৭। প্রধানমন্ত্রীর পদের মেয়াদ}
\label{sec:orga4725e6}
(১) প্রধানমন্ত্রীর পদ শূন্য হইবে, যদি-

(ক) তিনি কোন সময়ে রাষ্ট্রপতির নিকট পদত্যাগপত্র প্রদান করেন; অথবা

(খ) তিনি সংসদ-সদস্য না থাকেন।

(২) সংসদের সংখ্যাগরিষ্ঠ সদস্যের সমর্থন হারাইলে প্রধানমন্ত্রী পদত্যাগ করিবেন
    কিংবা সংসদ ভাংগিয়া দিবার জন্য লিখিতভাবে রাষ্ট্রপতিকে পরামর্শদান করিবেন এবং
    তিনি অনুরূপ পরামর্শদান করিলে রাষ্ট্রপতি, অন্য কোন সংসদ-সদস্য সংসদের
    সংখ্যাগরিষ্ঠ সদস্যের আস্থাভাজন নহেন এই মর্মে সন্তুষ্ট হইলে, সংসদ ভাংগিয়া
    দিবেন।

(৩) প্রধানমন্ত্রীর উত্তরাধিকারী কার্যভার গ্রহণ না করা পর্যন্ত প্রধানমন্ত্রীকে
    স্বীয় পদে বহাল থাকিতে এই অনুচ্ছেদের কোন কিছুই অযোগ্য করিবে না।

\subsection{৫৮। অন্যান্য মন্ত্রীর পদের মেয়াদ}
\label{sec:org760dad2}
(১) প্রধানমন্ত্রী ব্যতীত অন্য কোন মন্ত্রীর পদ শূন্য হইবে, যদি-

(ক) তিনি রাষ্ট্রপতির নিকট পেশ করিবার জন্য প্রধানমন্ত্রীর নিকট পদত্যাগপত্র
    প্রদান করেন;

(খ) তিনি সংসদ-সদস্য না থাকেন, তবে ৫৬ অনুচ্ছেদের (২) দফার শর্তাংশের অধীনে
    মনোনীত মন্ত্রীর ক্ষেত্রে ইহা প্রযোজ্য হইবে না;

(গ) এই অনুচ্ছেদের (২) দফা অনুসারে রাষ্ট্রপতি অনুরূপ নির্দেশ দান করেন; অথবা

(ঘ) এই অনুচ্ছেদের (৪) দফায় যেরূপ বিধান করা হইয়াছে তাহা কার্যকর হয়।

(২) প্রধানমন্ত্রী যে কোন সময়ে কোন মন্ত্রীকে পদত্যাগ করিতে অনুরোধ করিতে
    পারিবেন এবং উক্ত মন্ত্রী অনুরূপ অনুরোধ পালনে অসমর্থ হইলে তিনি রাষ্ট্রপতিকে উক্ত
    মন্ত্রীর নিয়োগের অবসান ঘটাইবার পরামর্শ দান করিতে পারিবেন।

(৩) সংসদ ভাংগিয়া যাওয়া অবস্থায় যেকোন সময়ে কোন মন্ত্রীকে স্বীয় পদে বহাল
    থাকিতে এই অনুচ্ছেদের (১) দফার (ক), (খ) ও (ঘ) উপ-দফার কোন কিছুই অযোগ্য
    করিবে না।

(৪) প্রধানমন্ত্রী পদত্যাগ করিলে বা স্বীয় পদে বহাল না থাকিলে মন্ত্রীদের
    প্রত্যেকে পদত্যাগ করিয়াছেন বলিয়া গণ্য হইবে; তবে এই পরিচ্ছেদের
    বিধানাবলী-সাপেক্ষে তাঁহাদের উত্তরাধিকারীগণ কার্যভার গ্রহণ না করা পর্যন্ত
    তাঁহারা স্ব স্ব পদে বহাল থাকিবেন।

(৫) এই অনুচ্ছেদে “মন্ত্রী” বলিতে প্রতিমন্ত্রী ও উপ-মন্ত্রী অন্তভর্ুক্ত।

পরিচ্ছেদের প্রয়োগ ২৩ ৫৮ক। এই পরিচ্ছেদের কোন কিছু ৫৫(৪), (৫) ও (৬) অনুচ্ছেদের
বিধানাবলী ব্যতীত, যে মেয়াদে সংসদ ভাংগিয়া দেওয়া হয় বা ভংগ অবস্থায় থাকে
সেই মেয়াদে প্রযুক্ত হইবে না:

তবে শর্ত থাকে যে, ২ক পরিচ্ছেদে যাহা কিছু থাকুক না কেন, যেক্ষেত্রে ৭২(৪)
অনুচ্ছেদের অধীন কোন ভংগ হইয়া যাওয়া সংসদকে পুনরাহ্বান করা হয় সেক্ষেত্রে এই
পরিচ্ছেদ প্রযোজ্য হইবে।

\subsection{৫৮ক। [বিলুপ্ত]}
\label{sec:org6357b38}

\subsection{৫৯। স্থানীয় শাসন}
\label{sec:org8ea106d}
(১) আইনানুযায়ী নির্বাচিত ব্যক্তিদের সমন্বয়ে গঠিত প্রতিষ্ঠানসমূহের উপর
    প্রজাতন্ত্রের প্রত্যেক প্রশাসনিক একাংশের স্থানীয় শাসনের ভার প্রদান করা হইবে।

(২) এই সংবিধান ও অন্য কোন আইন-সাপেক্ষে সংসদ আইনের দ্বারা যেরূপ নির্দিষ্ট
    করিবেন, এই অনুচ্ছেদের (১) দফায় উল্লিখিত প্রত্যেক প্রতিষ্ঠান যথোপযুক্ত প্রশাসনিক
    একাংশের মধ্যে সেইরূপ দায়িত্ব পালন করিবেন এবং অনুরূপ আইনে নিম্নলিখিত বিষয়
    সংক্রান্ত দায়িত্ব অন্তর্ভুক্ত হইতে পারিবে:

(ক) প্রশাসন ও সরকারী কর্মচারীদের কার্য;

(খ) জনশৃংখলা রক্ষা;

(ক) জনসাধারণের কার্য ও অর্থনৈতিক উন্নয়ন সম্পর্কিত পরিকল্পনা প্রণয়ন ও
    বাস্তবায়ন।

\subsection{৬০। স্থানীয় শাসন সংক্রান্ত প্রতিষ্ঠানের ক্ষমতা}
\label{sec:orgdd0dd83}
এই সংবিধানের ৫৯ অনুচ্ছেদের বিধানাবলীকে পূর্ণ কার্যকরতাদানের উদ্দেশ্যে সংসদ
আইনের দ্বারা উক্ত অনুচ্ছেদে উল্লিখিত স্থানীয় শাসন সংক্রান্ত প্রতিষ্ঠানসমূহকে
স্থানীয় প্রয়োজনে কর আরোপ করিবার ক্ষমতাসহ বাজেট প্রস্তুতকরণ ও নিজস্ব তহবিল
রক্ষণাবেক্ষণের ক্ষমতা প্রদান করিবেন।

\subsection{৬১। সর্বাধিনায়কতা}
\label{sec:org87c19ad}
বাংলাদেশের প্রতিরক্ষা কর্মবিভাগসমূহের সর্বাধিনায়কতা রাষ্ট্রপতির উপর ন্যস্ত
হইবে এবং আইনের দ্বারা তাহার প্রয়োগ ২৪ নিয়ন্ত্রিত হইবে এবং যে মেয়াদে ৫৮খ
অনুচ্ছেদের অধীন নির্দলীয় তত্ত্বাবধায়ক সরকার থাকিবে সেই মেয়াদে উক্ত আইন
রাষ্ট্রপতি কর্তৃক পরিচালিত হইবে।

\subsection{৬২। প্রতিরক্ষা কর্মবিভাগে ভর্তি প্রভৃতি}
\label{sec:orge253f09}
(১) সংসদ আইনের দ্বারা নিম্নলিখিত বিষয়সমূহ নিয়ন্ত্রণ করিবেন:

(ক) বাংলাদেশের প্রতিরক্ষা কর্মবিভাগসমূহ ও উক্ত কর্মবিভাগসমূহের সংরক্ষিত অংশসমূহ
    গঠন ও রক্ষণাবেক্ষণ;

(খ) উক্ত কর্মবিভাগসমূহে কমিশন মঞ্জুরী;

(গ) প্রতিরক্ষা-বাহিনীসমূহের প্রধানদের নিয়োগদান ও তাঁহাদের বেতন ও
    ভাতা-নির্ধারণ; এবং

(ঘ) উক্ত কর্মবিভাগসমূহ ও সংরক্ষিত অংশসমূহ-সংক্রান্ত শৃঙ্খলামূলক ও অন্যান্য বিষয়।

(২) সংসদ আইনের দ্বারা এই অনুচ্ছেদের (১) দফায় বর্ণিত বিষয়সমূহের জন্য বিধান না
    করা পর্যন্ত অনুরূপ যে সকল বিষয় প্রচলিত আইনের অধীন নহে, রাষ্ট্রপতি আদেশের
    দ্বারা সেই সকল বিষয়ের জন্য বিধান করিতে পারিবেন।

\subsection{৬৩। যুদ্ধ}
\label{sec:org76f286a}
(১) সংসদের সম্মতি ব্যতীত যুদ্ধ ঘোষণা করা যাইবে না কিংবা প্রজাতন্ত্র কোন যুদ্ধে অংশ গ্রহণ করিবেন না।

\subsection{৬৪। অ্যাটর্ণি-জেনারেল}
\label{sec:orgfcb364b}
(১) সুপ্রীম কোর্টের বিচারক হইবার যোগ্য কোন ব্যক্তিকে রাষ্ট্রপতি বাংলাদেশের
    অ্যাটর্ণি-জেনারেল-পদে নিয়োগদান করিবেন।

(২) অ্যাটর্ণি-জেনারেল রাষ্ট্রপতি কর্তৃক প্রদত্ত সকল দায়িত্ব পালন করিবেন।

(৩) অ্যাটর্ণি-জেনারেলের দায়িত্ব পালনের জন্য বাংলাদেশের সকল আদালতে তাঁহার
    বক্তব্য পেশ করিবার অধিকার থাকিবে।

(৪) রাষ্ট্রপতির সন্তোষানুযায়ী সময়সীমা পর্যন্ত অ্যাটর্ণি-জেনারেল স্বীয় পদে বহাল
    থাকিবেন এবং রাষ্ট্রপতি কর্তৃক নির্ধারিত পারিশ্রমিক লাভ করিবেন।

\subsection{৬৫। সংসদ-প্রতিষ্ঠা}
\label{sec:orgfa42e0e}
(১) “জাতীয় সংসদ” নামে বাংলাদেশের একটি সংসদ থাকিবে এবং এই সংবিধানের
    বিধানাবলী-সাপেক্ষে প্রজাতন্ত্রের আইনপ্রণয়ন-ক্ষমতা সংসদের উপর ন্যস্ত হইবে:

তবে শর্ত থাকে যে, সংসদের আইন দ্বারা যে কোন ব্যক্তি বা কর্তৃপক্ষকে আদেশ, বিধি,
প্রবিধান, উপ-আইন বা আইনগত কার্যকরতাসম্পন্ন অন্যান্য চুক্তিপত্র প্রণয়নের ক্ষমতার্পণ
হইতে এই দফার কোন কিছুই সংসদকে নিবৃত্ত করিবে না।

(২) একক আঞ্চলিক নির্বাচনী এলাকাসমূহ হইতে প্রত্যক্ষ নির্বাচনের মাধ্যমে
    আইনানুযায়ী নির্বাচিত তিন শত সদস্য লইয়া এবং এই অনুচ্ছেদের (৩) দফার
    কার্যকরতাকালে উক্ত দফায় বর্ণিত সদস্যদিগকে লইয়া সংসদ গঠিত হইবে; সদস্যগণ
    সংসদ-সদস্য বলিয়া অভিহিত হইবেন।

২৬ (৩) সংবিধান (চতুর্দশ সংশোধন) আইন, ২০০৪ প্রবর্তনকালে বিদ্যমান সংসদের
অব্যবহিত পরবর্তী সংসদের প্রথম বৈঠকের তারিখ হইতে শুরু করিয়া দশ বৎসর কাল
অতিবাহিত হইবার অব্যবহিত পরবর্তীকালে সংসদ ভাংগিয়া না যাওয়া পর্যন্ত
পঁয়তাল্লিশটি আসন কেবল মহিলা-সদস্যদের জন্য সংরক্ষিত থাকিবে এবং তাঁহারা
আইনানুযায়ী পূর্বোক্ত সদস্যদের দ্বারা সংসদে আনুপাতিক প্রতিনিধিত্ব পদ্ধতির ভিত্তিতে
একক হস্তান্তরযোগ্য ভোটের মাধ্যমে নির্বাচিত হইবেন:

তবে শর্ত থাকে যে, এই দফার কোন কিছুই এই অনুচ্ছেদের (২) দফার অধীন কোন আসনে
কোন মহিলার নির্বাচন নিবৃত্ত করিবে না।

(৪) রাজধানীতে সংসদের আসন থাকিবে।

\subsection{৬৬। সংসদে নির্বাচিত হইবার যোগ্যতা ও অযোগ্যতা}
\label{sec:org0424709}
(১) কোন ব্যক্তি বাংলাদেশের নাগরিক হইলে এবং তাঁহার বয়স পঁচিশ বৎসর পূর্ণ হইলে
    এই অনুচ্ছেদের (২) দফায় বর্ণিত বিধান-সাপেক্ষে তিনি সংসদের সদস্য নির্বাচিত
    হইবার এবং সংসদ-সদস্য থাকিবার যোগ্য হইবেন।

(২) কোন ব্যক্তি সংসদের সদস্য নির্বাচিত হইবার এবং সংসদ-সদস্য থাকিবার যোগ্য
    হইবেন না, যদি

(ক) কোন উপযুক্ত আদালত তাঁহাকে অপ্রকৃতিস্থ বলিয়া ঘোষণা করেন;

(খ) তিনি দেউলিয়া ঘোষিত হইবার পর দায় হইতে অব্যাহতি লাভ না করিয়া থাকেন;

(গ) তিনি কোন বিদেশী রাষ্ট্রের নাগরিকত্ব অর্জন করেন কিংবা কোন বিদেশী
    রাষ্ট্রের প্রতি আনুগত্য ঘোষণা বা স্বীকার করেন;

(ঘ) তিনি নৈতিক স্খলনজনিত কোন ফৌজদারী অপরাধে দোষী সাব্যস্ত হইয়া অনূ্যন দুই
    বৎসরের কারাদণ্ডে দণ্ডিত হন এবং তাঁহার মুক্তিলাভের পর পাঁচ বৎসরকাল অতিবাহিত
    না হইয়া থাকে;

২৭ (ঘঘ) আইনের দ্বারা পদাধিকারীকে অযোগ্য ঘোষণা করিতেছে না এমন পদ ব্যতীত
তিনি প্রজাতন্ত্রের কর্মে কোন লাভজনক পদে অধিষ্ঠিত থাকেন; অথবা


(ছ) তিনি কোন আইনের দ্বারা বা অধীন অনুরূপ নির্বাচনের জন্য অযোগ্য হন।

৩০ (২ক) এই অনুচ্ছেদের উদ্দেশ্য সাধনকল্পে কোন ব্যক্তি ৩১ কেবল রাষ্ট্রপতি,
৩২ * * * প্রধানমন্ত্রী, ৩৩ * * * মন্ত্রী, প্রতিমন্ত্রী বা উপ-মন্ত্রী হইবার কারণে
প্রজাতন্ত্রের কর্মে কোন লাভজনক পদে অধিষ্ঠিত বলিয়া গণ্য হইবেন না।

(৪) কোন সংসদ-সদস্য তাঁহার নির্বাচনের পর এই অনুচ্ছেদের (২) দফায় বর্ণিত
    অযোগ্যতার অধীন হইয়াছেন কিনা কিংবা এই সংবিধানের ৭০ অনুচ্ছেদ অনুসারে কোন
    সংসদ-সদস্যের আসন শূন্য হইবে কিনা, সে সম্পর্কে কোন বিতর্ক দেখা দিলে শুনানী ও
    নিষ্পত্তির জন্য প্রশ্নটি নির্বাচন কমিশনের নিকট প্রেরিত হইবে এবং অনুরূপ ক্ষেত্রে
    কমিশনের সিদ্ধান্ত চূড়ান্ত হইবে।

(৫) এই অনুচ্ছেদের (৪) দফার বিধানাবলী যাহাতে পূর্ণ কার্যকরতা লাভ করিতে পারে,
    সেই উদ্দেশ্যে নির্বাচন কমিশনকে ক্ষমতাদানের জন্য সংসদ যেরূপ প্রয়োজন বোধ
    করিবেন, আইনের দ্বারা সেইরূপ বিধান করিতে পারিবেন।

\subsection{৬৭। সদস্যদের আসন শূন্য হওয়া}
\label{sec:org1746242}
(১) কোন সংসদ-সদস্যের আসন শূন্য হইবে, যদি

(ক) তাঁহার নির্বাচনের পর সংসদের প্রথম বৈঠকের তারিখ হইতে নব্বই দিনের মধ্যে
    তিনি তৃতীয় তফসিলে নির্ধারিত শপথ গ্রহণ বা ঘোষণা করিতে ও শপথপত্রে বা
    ঘোষণাপত্রে স্বাক্ষরদান করিতে অসমর্থ হন:

তবে শর্ত থাকে যে, অনুরূপ মেয়াদ অতিবাহিত হইবার পূর্বে স্পীকার যথার্থ কারণে
তাহা বর্ধিত করিতে পারিবেন;

(খ) সংসদের অনুমতি না লইয়া তিনি একাদিক্রমে নব্বই বৈঠক-দিবস অনুপস্থিত থাকেন;

(গ) সংসদ ভাঙ্গিয়া যায়;

(ঘ) তিনি এই সংবিধানের ৬৬ অনুচ্ছেদের (২) দফার অধীন অযোগ্য হইয়া যান; অথবা

(ঙ) এই সংবিধানের ৭০ অনুচ্ছেদে বর্ণিত পরিস্থিতির উদ্ভব হয়।

(২) কোন সংসদ-সদস্য স্পীকারের নিকট স্বাক্ষরযুক্ত পত্রযোগে স্বীয় পদ ত্যাগ করিতে
    পারিবেন, এবং স্পীকার- কিংবা স্পীকারের পদ শূন্য থাকিলে বা অন্য কোন কারণে
    স্পীকার স্বীয় দায়িত্ব পালনে অসমর্থ হইলে ডেপুটি স্পীকার- যখন উক্ত পত্র প্রাপ্ত
    হন, তখন হইতে উক্ত সদস্যের আসন শূন্য হইবে।

\subsection{৬৮। সংসদ-সদস্যদের [পারিশ্রমিক] প্রভৃতি}
\label{sec:org91632ad}
সংসদের আইন দ্বারা কিংবা অনুরূপভাবে নির্ধারিত না হওয়া পর্যন্ত রাষ্ট্রপতি কর্তৃক
আদেশের দ্বারা যেরূপ নির্ধারিত হইবে, সংসদ-সদস্যগণ সেইরূপ ৩৬ পারিশ্রমিক, ভাতা
ও বিশেষ-অধিকার লাভ করিবেন।

\subsection{৬৯। শপথ গ্রহণের পূর্বে আসন গ্রহণ বা ভোট দান করিলে সদস্যের অর্থদন্ড}
\label{sec:org10ecc0f}
কোন ব্যক্তি এই সংবিধানের বিধান অনুযায়ী শপথ গ্রহণ বা ঘোষণা করিবার এবং
শপথপত্রে বা ঘোষণাপত্রে স্বাক্ষরদান করিবার পূর্বে কিংবা তিনি সংসদ-সদস্য হইবার
যোগ্য নহেন বা অযোগ্য হইয়াছেন জানিয়া সংসদ-সদস্যরূপে আসনগ্রহণ বা ভোটদান
করিলে তিনি প্রতি দিনের অনুরূপ কার্যের জন্য প্রজাতন্ত্রের নিকট দেনা হিসাবে
উসুলযোগ্য এক হাজার টাকা করিয়া অর্থদণ্ডে দণ্ডনীয় হইবেন।

\subsection{৭০। রাজনৈতিক দল হইতে পদত্যাগ বা দলের বিপক্ষে ভোটদানের কারণে আসন শূন্য হওয়া}
\label{sec:orgb47fb66}
(১) কোন নির্বাচনে কোন রাজনৈতিক দলের প্রার্থীরূপে মনোনীত হইয়া কোন ব্যক্তি
    সংসদ-সদস্য নির্বাচিত হইলে তিনি যদি উক্ত দল হইতে পদত্যাগ করেন, অথবা সংসদে
    উক্ত দলের বিপক্ষে ভোটদান করেন, তাহা হইলে সংসদে তাঁহার আসন শূন্য হইবে।

ব্যাখ্যা।- যদি কোন সংসদ-সদস্য, যে দল তাঁহাকে নির্বাচনে প্রার্থীরূপে মনোনীত
করিয়াছেন, সেই দলের নির্দেশ অমান্য করিয়া-

(ক) সংসদে উপস্থিত থাকিয়া ভোটদানে বিরত থাকেন, অথবা

(খ) সংসদের কোন বৈঠকে অনুপস্থিত থাকেন,

তাহা হইলে তিনি উক্ত দলের বিপক্ষে ভোটদান করিয়াছেন বলিয়া গণ্য হইবেন।

(২) যদি কোন সময় কোন রাজনৈতিক দলের সংসদীয় দলের নেতৃত্ব সম্পর্কে কোন প্রশ্ন
    উঠে তাহা হইলে সংসদে সেই দলের সংখ্যাগরিষ্ঠ সদস্যের নেতৃত্বের দাবীদার কোন
    সদস্য কর্তৃক লিখিতভাবে অবহিত হইবার সাত দিনের মধ্যে স্পীকার সংসদের
    কার্যপ্রণালী-বিধি অনুযায়ী উক্ত দলের সকল সংসদ-সদস্যের সভা আহ্বান করিয়া বিভক্তি
    ভোটের মাধ্যমে সংখ্যাগরিষ্ঠ ভোটের দ্বারা উক্ত দলের সংসদীয় নেতৃত্ব নির্ধারণ
    করিবেন এবং সংসদে ভোটদানের ব্যাপারে অনুরূপ নির্ধারিত নেতৃত্বের নির্দেশ যদি কোন
    সদস্য অমান্য করেন তাহা হইলে তিনি (১) দফার অধীন উক্ত দলের বিপক্ষে ভোটদান
    করিয়াছেন বলিয়া গণ্য হইবে এবং সংসদে তাঁহার আসন শূন্য হইবে।

(৩) যদি কোন ব্যক্তি নির্দলীয় প্রার্থীরূপে সংসদ-সদস্য নির্বাচিত হইবার পর কোন
    রাজনৈতিক দলে যোগদান করেন, তাহা হইলে তিনি এই অনুচ্ছেদের উদ্দেশ্য সাধনকল্পে
    উক্ত দলের প্রার্থীরূপে সংসদ-সদস্য নির্বাচিত হইয়াছেন বলিয়া গণ্য হইবে।

\subsection{৭১। দ্বৈত-সদস্যতায় বাধা}
\label{sec:org08ccaad}
(১) কোন ব্যক্তি একই সময়ে দুই বা ততোধিক নির্বাচনী এলাকার সংসদ-সদস্য হইবেন
    না।

(২) কোন ব্যক্তির একই সময়ে দুই বা ততোধিক নির্বাচনী এলাকা হইতে নির্বাচন
    প্রার্থী হওয়ায় এই অনুচ্ছেদের (১) দফায় বর্ণিত কোন কিছুই প্রতিবন্ধকতা সৃষ্টি করিবে
    না, তবে তিনি যদি একাধিক নির্বাচনী এলাকা হইতে নির্বাচিত হন তাহা হইলে-

(ক) তাঁহার সর্বশেষ নির্বাচনের ত্রিশ দিনের মধ্যে তিনি কোন্ নির্বাচনী এলাকার
    প্রতিনিধিত্ব করিতে ইচ্ছুক, তাহা জ্ঞাপন করিয়া নির্বাচন কমিশনকে একটি স্বাক্ষরযুক্ত
    ঘোষণা প্রদান করিবেন এবং তিনি অন্য যে সকল নির্বাচনী এলাকা হইতে নির্বাচিত
    হইয়াছিলেন, অতঃপর সেই সকল এলাকার আসনসমূহ শূন্য হইবে;

(খ) এই দফার (ক) উপ-দফা মান্য করিতে অসমর্থ হইলে তিনি যে সকল আসনে নির্বাচিত
    হইয়াছিলেন, সেই সকল আসন শূন্য হইবে; এবং

(গ) এই দফার উপরি-উক্ত বিধানসমূহ যতখানি প্রযোজ্য, ততখানি পালন না করা পর্যন্ত
    নির্বাচিত ব্যক্তি সংসদ-সদস্যের শপথ গ্রহণ বা ঘোষণা করিতে ও শপথপত্রে বা
    ঘোষণাপত্রে স্বাক্ষরদান করিতে পারিবেন না।

\subsection{৭২। সংসদের অধিবেশন}
\label{sec:org7294672}
(১) সরকারী বিজ্ঞপ্তি-দ্বারা রাষ্ট্রপতি সংসদ আহ্বান, স্থগিত ও ভঙ্গ করিবেন এবং
    সংসদ আহ্বানকালে রাষ্ট্রপতি প্রথম বৈঠকের সময় ও স্থান নির্ধারণ করিবেন:

৩৮ তবে শর্ত থাকে যে, সংসদের এক অধিবেশনের সমাপ্তি ও পরবর্তী অধিবেশনের প্রথম
বৈঠকের মধ্যে ষাট দিনের অতিরিক্ত বিরতি থাকিবে না:

তবে আরও শর্ত থাকে যে, এই দফার অধীন তাঁহার দায়িত্ব পালনে রাষ্ট্রপতি
প্রধানমন্ত্রী কর্তৃক লিখিতভাবে প্রদত্ত পরামর্শ অনুযায়ী কার্য করিবেন।

(২) এই অনুচ্ছেদের (১) দফার বিধানাবলী সত্ত্বেও সংসদ-সদস্যদের যে কোন সাধারণ
    নির্বাচনের ফলাফল ঘোষিত হইবার ত্রিশ দিনের মধ্যে বৈঠক অনুষ্ঠানের জন্য সংসদ
    আহ্বান করা হইবে।

(৩) রাষ্ট্রপতি পূর্বে ভাঙ্গিয়া না দিয়া থাকিলে প্রথম বৈঠকের তারিখ হইতে পাঁচ
    বৎসর অতিবাহিত হইলে সংসদ ভাঙ্গিয়া যাইবে:

তবে শর্ত থাকে যে, প্রজাতন্ত্র যুদ্ধে লিপ্ত থাকিবার কালে সংসদের আইন-দ্বারা অনুরূপ
মেয়াদ এককালে অনধিক এক বৎসর বর্ধিত করা যাইতে পারিবে, তবে যুদ্ধ সমাপ্ত হইলে
বর্ধিত মেয়াদ কোনক্রমে ছয় মাসের অধিক হইবে না।

(৪) সংসদ ভঙ্গ হইবার পর এবং সংসদের পরবর্তী সাধারণ নির্বাচন অনুষ্ঠানের পূর্বে
    রাষ্ট্রপতির নিকট যদি সন্তোষজনকভাবে প্রতীয়মান হয় যে, প্রজাতন্ত্র যে যুদ্ধে লিপ্ত
    রহিয়াছেন, সেই যুদ্ধাবস্থার বিদ্যমানতার জন্য সংসদ পুনরাহ্বান করা প্রয়োজন, তাহা
    হইলে যে সংসদ ভাঙ্গিয়া দেওয়া হইয়াছিল, রাষ্ট্রপতি তাহা আহবান করিবেন।

৩৯ * * *

(৫) এই অনুচ্ছেদের (১) দফার বিধানাবলী-সাপেক্ষে কার্যপ্রণালী-বিধি-দ্বারা বা
    অন্যভাবে সংসদ যেরূপ নির্ধারণ করিবেন, সংসদের বৈঠকসমূহ সেইরূপ সময়ে ও স্থানে
    অনুষ্ঠিত হইবে।

\subsection{৭৩। সংসদে রাষ্ট্রপতির ভাষণ ও বাণী}
\label{sec:org824135a}
(১) রাষ্ট্রপতি সংসদে ভাষণদান এবং বাণী প্রেরণ করিতে পারিবেন।

(২) সংসদ-সদস্যদের প্রত্যেক সাধারণ নির্বাচনের পর প্রথম অধিবেশনের সূচনায় এবং
    প্রত্যেক বৎসর প্রথম অধিবেশনের সূচনায় রাষ্ট্রপতি সংসদে ভাষণ দান করিবেন।

(৩) রাষ্ট্রপতি কর্তৃক প্রদত্ত ভাষণ শ্রবণ বা প্রেরিত বাণী প্রাপ্তির পর সংসদ উক্ত
    ভাষণ বা বাণী সম্পর্কে আলোচনা করিবেন।

\subsection{৭৩ক। সংসদ সম্পর্কে মন্ত্রীগণের অধিকার}
\label{sec:orgdbb38bf}


\subsection{৭৪। স্পীকার ও ডেপুটি স্পীকার}
\label{sec:orgeea0dc2}
(১) কোন সাধারণ নির্বাচনের পর সংসদের প্রথম বৈঠকে সংসদ-সদস্যদের মধ্য হইতে
    সংসদ একজন স্পীকার ও একজন ডেপুটি স্পীকার নির্বাচিত করিবেন, এবং এই দুই পদের
    যে কোনটি শূন্য হইলে সাত দিনের মধ্যে কিংবা ঐ সময়ে সংসদ বৈঠকরত না থাকিলে
    পরবর্তী প্রথম বৈঠকে তাহা পূর্ণ করিবার জন্য সংসদ-সদস্যদের মধ্য হইতে একজনকে
    নির্বাচিত করিবেন।

(২) স্পীকার বা ডেপুটি স্পীকারের পদ শূন্য হইবে, যদি

(ক) তিনি সংসদ-সদস্য না থাকেন;

(খ) তিনি মন্ত্রী-পদ গ্রহণ করেন;

(গ) পদ হইতে তাঁহার অপসারণ দাবী করিয়া মোট সংসদ-সদস্যের সংখ্যাগরিষ্ঠ ভোটে
    সমর্থিত কোন প্রস্তাব (প্রস্তাবটি উত্থাপনের অভিপ্রায় জ্ঞাপন করিয়া অন্যূন চৌদ্দ
    দিনের নোটিশ প্রদানের পর) সংসদে গৃহীত হয়;

(ঘ) তিনি রাষ্ট্রপতির নিকট স্বাক্ষরযুক্ত পত্রযোগে তাঁহার পদ ত্যাগ করেন;

(ঙ) কোন সাধারণ নির্বাচনের পর অন্য কোন সদস্য তাঁহার কার্যভার গ্রহণ করেন; অথবা

(চ) ডেপুটি স্পীকারের ক্ষেত্রে, তিনি স্পীকারের পদে যোগদান করেন।

(৩) স্পীকারের পদ শূন্য হইলে বা তিনি ৪৪ রাষ্ট্রপতিরূপে কার্য করিলে কিংবা অন্য
    কোন কারণে তিনি স্বীয় দায়িত্ব পালনে অসমর্থ বলিয়া সংসদ নির্ধারণ করিলে
    স্পীকারের সকল দায়িত্ব ডেপুটি স্পীকার পালন করিবেন, কিংবা ডেপুটি স্পীকারের পদও
    শূন্য হইলে সংসদের কার্যপ্রণালী-বিধি-অনুযায়ী কোন সংসদ-সদস্য তাহা পালন করিবেন;
    এবং সংসদের কোন বৈঠকে স্পীকারের অনুপস্থিতিতে ডেপুটি স্পীকার কিংবা ডেপুটি
    স্পীকারও অনুপস্থিত থাকিলে সংসদের কার্যপ্রণালী-বিধি-অনুযায়ী কোন সংসদ-সদস্য
    স্পীকারের দায়িত্ব পালন করিবেন।

(৪) সংসদের কোন বৈঠকে স্পীকারকে তাঁহার পদ হইতে অপসারণের জন্য কোন প্রস্তাব
    বিবেচনাকালে স্পীকার (কিংবা ডেপুটি স্পীকারকে তাঁহার পদ হইতে অপসারণের জন্য
    কোন প্রস্তাব বিবেচনাকালে ডেপুটি স্পীকার) উপস্থিত থাকিলেও সভাপতিত্ব করিবেন না
    এবং এই অনুচ্ছেদের (৩) দফায় বর্ণিত ক্ষেত্রমত স্পীকার বা ডেপুটি স্পীকারের
    অনুপস্থিতিকালীন বৈঠক সম্পর্কে প্রযোজ্য বিধানাবলী অনুরূপ প্রত্যেক বৈঠকের ক্ষেত্রে
    প্রযোজ্য হইবে।

(৫) স্পীকার বা ডেপুটি স্পীকারের অপসারণের জন্য কোন প্রস্তাব সংসদে বিবেচিত
    হইবার কালে ক্ষেত্রমত স্পীকার বা ডেপুটি স্পীকারের কথা বলিবার ও সংসদের
    কার্যধারায় অন্যভাবে অংশগ্রহণের অধিকার থাকিবে এবং তিনি কেবল সদস্যরূপে
    ভোটদানের অধিকারী হইবেন।

(৬) এই অনুচ্ছেদের (২) দফার বিধানাবলী সত্ত্বেও ক্ষেত্রমত স্পীকার বা ডেপুটি
    স্পীকার তাঁহার উত্তরাধিকারী কার্যভার গ্রহণ না করা পর্যন্ত স্বীয় পদে বহাল
    রহিয়াছেন বলিয়া গণ্য হইবে।

\subsection{৭৫। কার্যপ্রণালী-বিধি, কোরাম প্রভৃতি}
\label{sec:org22c89dd}
(১) এই সংবিধান-সাপেক্ষে

(ক) সংসদ কর্তৃক প্রণীত কার্যপ্রণালী-বিধি-দ্বারা এবং অনুরূপ বিধি প্রণীত না হওয়া
    পর্যন্ত রাষ্ট্রপতি কর্তৃক প্রণীত কার্যপ্রণালী-বিধি-দ্বারা সংসদের কার্যপ্রণালী
    নিয়ন্ত্রিত হইবে;

(খ) উপস্থিত ও ভোটদানকারী সদস্যদের সংখ্যাগরিষ্ঠ ভোটে সংসদে সিদ্ধান্ত গৃহীত
    হইবে, তবে সমসংখ্যক ভোটের ক্ষেত্র ব্যতীত সভাপতি ভোটদান করিবেন না এবং অনুরূপ
    ক্ষেত্রে তিনি নির্ণায়ক ভোট প্রদান করিবেন;

(গ) সংসদের কোন সদস্যপদ শূন্য রহিয়াছে, কেবল এই কারণে কিংবা সংসদে উপস্থিত
    হইবার বা ভোটদানের বা অন্য কোন উপায়ে কার্যধারায় অংশগ্রহণের অধিকার না থাকা
    সত্ত্বেও কোন ব্যক্তি অনুরূপ কার্য করিয়াছেন, কেবল এই কারণে সংসদের কোন কার্যধারা
    অবৈধ হইবে না।

(২) সংসদের বৈঠক চলাকালে কোন সময়ে উপস্থিত সদস্য-সংখ্যা ষাটের কম বলিয়া যদি
    সভাপতির দৃষ্টি আকর্ষণ করা হয়, তাহা হইলে তিনি অনূ্যন ষাট জন সদস্য উপস্থিত না
    হওয়া বৈঠক স্থগিত রাখিবেন কিংবা মুলতবী করিবেন।

\subsection{৭৬। সংসদের স্থায়ী কমিটিসমূহ}
\label{sec:orgb6dc9fe}
(১) ৪৫ * * * সংসদ-সদস্যদের মধ্য হইতে সদস্য লইয়া সংসদ নিম্নলিখিত স্থায়ী
    কমিটিসমূহ নিয়োগ করিবেন:

(ক) সরকারী হিসাব কমিটি;

(খ) বিশেষ-অধিকার কমিটি; এবং

(গ) সংসদের কার্যপ্রণালী-বিধিতে নির্দিষ্ট অন্যান্য স্থায়ী কমিটি।

(২) সংসদ এই অনুচ্ছেদের (১) দফায় উলি্লখিত কমিটিসমূহের অতিরিক্ত অন্যান্য স্থায়ী
    কমিটি নিয়োগ করিবেন এবং অনুরূপভাবে নিযুক্ত কোন কমিটি এই সংবিধান ও অন্য কোন
    আইন-সাপেক্ষে

(ক) খসড়া বিল ও অন্যান্য আইনগত প্রস্তাব পরীক্ষা করিতে পারিবেন;

(খ) আইনের বলবৎকরণ পর্যালোচনা এবং অনুরূপ বলবৎকরণের জন্য ব্যবস্থাদি গ্রহণের
    প্রস্তাব করিতে পারিবেন;

(গ) জনগুরুত্বসম্পন্ন বলিয়া সংসদ কোন বিষয় সম্পর্কে কমিটিকে অবহিত করিলে সেই
    বিষয়ে কোন মন্ত্রণালয়ের কার্য বা প্রশাসন সম্বন্ধে অনুসন্ধান বা তদন্ত করিতে
    পারিবেন এবং কোন মন্ত্রণালয়ের নিকট হইতে ক্ষমতাপ্রাপ্ত প্রতিনিধির মাধ্যমে
    প্রাসঙ্গিক তথ্যাদি সংগ্রহের এবং প্রশ্নাদির মৌখিক বা লিখিত উত্তরলাভের ব্যবস্থা
    করিতে পারিবেন;

(ঘ) সংসদ কর্তৃক অর্পিত যে কোন দায়িত্ব পালন করিতে পারিবেন।

(৩) সংসদ আইনের দ্বারা এই অনুচ্ছেদের অধীন নিযুক্ত কমিটিসমূহকে

(ক) সাক্ষীদের হাজিরা বলবৎ করিবার এবং শপথ, ঘোষণা বা অন্য কোন উপায়ের অধীন
    করিয়া তাঁহাদের সাক্ষ্যগ্রহণের;

(খ) দলিলপত্র দাখিল করিতে বাধ্য করিবার;

ক্ষমতা প্রদান করিতে পারিবেন।

\subsection{৭৭। ন্যায়পাল}
\label{sec:org2aade2c}
(১) সংসদ আইনের দ্বারা ন্যায়পালের পদ-প্রতিষ্ঠার জন্য বিধান করিতে পারিবেন।

(২) সংসদ আইনের দ্বারা ন্যায়পালকে কোন মন্ত্রণালয়, সরকারী কর্মচারী বা
    সংবিধিবদ্ধ সরকারী কর্তৃপক্ষের যে কোন কার্য সম্পর্কে তদন্ত পরিচালনার ক্ষমতাসহ
    যেরূপ ক্ষমতা কিংবা যেরূপ দায়িত্ব প্রদান করিবেন, ন্যায়পাল সেইরূপ ক্ষমতা প্রয়োগ ও
    দায়িত্ব পালন করিবেন।

(৩) ন্যায়পাল তাঁহার দায়িত্বপালন সম্পর্কে বাৎসরিক রিপোর্ট প্রণয়ন করিবেন এবং
    অনুরূপ রিপোর্ট সংসদে উপস্থাপিত হইবে।

\subsection{৭৮। সংসদ ও সদস্যদের বিশেষ অধিকার ও দায়মুক্তি}
\label{sec:orge9cf788}
(১) সংসদের কার্যধারার বৈধতা সম্পর্কে কোন আদালতে প্রশ্ন উত্থাপন করা যাইবে না।

(২) সংসদের যে সদস্য বা কর্মচারীর উপর সংসদের কার্যপ্রণালী নিয়ন্ত্রণ,
    কার্যপরিচালনা বা শৃঙ্খলা রক্ষার ক্ষমতা ন্যস্ত থাকিবে, তিনি সকল ক্ষমতা-প্রয়োগ
    সম্পর্কিত কোন ব্যাপারে কোন আদালতের এখতিয়ারের অধীন হইবেন না।

(৩) সংসদে বা সংসদের কোন কমিটিতে কিছু বলা বা ভোটদানের জন্য কোন
    সংসদ-সদস্যের বিরুদ্ধে কোন আদালতে কার্যধারা গ্রহণ করা যাইবে না।

(৪) সংসদ কর্তৃক বা সংসদের কর্তৃত্বে কোন রিপোর্ট, কাগজপত্র, ভোট বা কার্যধারা
    প্রকাশের জন্য কোন ব্যক্তির বিরুদ্ধে কোন আদালতে কোন কার্যধারা গ্রহণ করা যাইবে
    না।

(৫) এই অনুচ্ছেদ-সাপেক্ষে সংসদের আইন-দ্বারা সংসদের, সংসদের কমিটিসমূহের এবং
    সংসদ-সদস্যদের বিশেষ-অধিকার নির্ধারণ করা যাইতে পারিবে।

\subsection{৭৯। সংসদ-সচিবালয়}
\label{sec:orgecd6613}
(১) সংসদের নিজস্ব সচিবালয় থাকিবে।

(২) সংসদের সচিবালয়ে কর্মচারীদের নিয়োগ ও কর্মের শর্তসমূহ সংসদ আইনের দ্বারা
    নির্ধারণ করিতে পারিবেন।

(৩) সংসদ কর্তৃক বিধান প্রণীত না হওয়া পর্যন্ত স্পীকারের সহিত পরামর্শক্রমে
    রাষ্ট্রপতি সংসদের সচিবালয়ে কর্মচারীদের নিয়োগ ও কর্মের শর্তসমূহ নির্ধারণ করিয়া
    বিধি প্রণয়ন করিতে পারিবেন এবং অনুরূপভাবে প্রণীত বিধিসমূহ যে কোন আইনের
    বিধান-সাপেক্ষে কার্যকর হইবে।

\subsection{৮০। আইন প্রণয়ন পদ্ধতি}
\label{sec:org395b300}
(১) আইনপ্রণয়নের উদ্দেশ্যে সংসদে আনীত প্রত্যেকটি প্রস্তাব বিল আকারে উত্থাপিত
    হইবে।

(২) সংসদ কর্তৃক কোন বিল গৃহীত হইলে সম্মতির জন্য তাহা রাষ্ট্রপতির নিকট পেশ
    করিতে হইবে।

(৩) রাষ্ট্রপতির নিকট কোন বিল পেশ করিবার পর পনর দিনের মধ্যে তিনি তাহাতে
    সম্মতিদান করিবেন ৪৬ * * * কিংবা অর্থবিল ব্যতীত অন্য কোন বিলের ক্ষেত্রে বিলটি
    বা তাহার কোন বিশেষ বিধান পুনর্বিবেচনার কিংবা রাষ্ট্রপতি কর্তৃক নির্দেশিত কোন
    সংশোধনী বিবেচনার অনুরোধ জ্ঞাপন করিয়া একটি বার্তাসহ তিনি বিলটি সংসদে ফেরত
    দিতে পারিবেন; এবং রাষ্ট্রপতি তাহা করিতে অসমর্থ হইলে উক্ত মেয়াদের অবসানে
    তিনি বিলটিতে সম্মতিদান করিয়াছেন বলিয়া গণ্য হইবে।

(৪) রাষ্ট্রপতি যদি বিলটি অনুরূপভাবে সংসদে ফেরত পাঠান, তাহা হইলে সংসদ
    রাষ্টপতির বার্তাসহ তাহা পুনর্বিবেচনা করিবেন; এবং সংশোধনীসহ বা সংশোধনী
    ব্যতিরেকে ৪৭ মোট সংসদ-সদস্যের সংখ্যাগরিষ্ঠ ভোটের দ্বারা সংসদ পুনরায় বিলটি
    গ্রহণ করিলে সম্মতির জন্য তাহা রাষ্ট্রপতির নিকট উপস্থাপিত হইবে এবং অনুরূপ
    উপস্থাপনের সাত দিনের মধ্যে তিনি বিলটিতে সম্মতিদান করিবেন; এবং রাষ্ট্রপতি
    তাহা করিতে অসমর্থ হইলে উক্ত মেয়াদের অবসানে তিনি বিলটিতে সম্মতিদান
    করিয়াছেন বলিয়া গণ্য হইবে।

(৫) সংসদ কর্তৃক গৃহীত বিলটিতে রাষ্ট্রপতি সম্মতিদান করিলে বা তিনি সম্মতিদান
    করিয়াছেন বলিয়া গণ্য হইলে তাহা আইনে পরিণত হইবে এবং সংসদের আইন বলিয়া
    অভিহিত হইবে।

\subsection{৮১। অর্থবিল}
\label{sec:orge646edd}
(১) এই ভাগে “অর্থবিল” বলিতে কেবল নিম্নলিখিত বিষয়সমূহের সকল বা যে কোন একটি
    সম্পর্কিত বিধানাবলী-সংবলিত বিল বুঝাইবে:

(ক) কোন কর আরোপ, নিয়ন্ত্রণ, রদবদল, মওকুফ বা রহিতকরণ;

(খ) সরকার কর্তৃক ঋণগ্রহণ বা কোন গ্যারান্টিদান, কিংবা সরকারের আর্থিক
    দায়-দায়িত্ব সম্পর্কিত আইন সংশোধন;

(গ) সংযুক্ত তহবিলের রক্ষণাবেক্ষণ, অনুরূপ তহবিলে অর্থপ্রদান বা অনুরূপ তহবিল হইতে
    অর্থ দান বা নির্দিষ্টকরণ;

(ঘ) সংযুক্ত তহবিলের উপর দায় আরোপ কিংবা অনুরূপ কোন দায় রদবদল বা বিলোপ;

(ঙ) সংযুক্ত তহবিল বা প্রজাতন্ত্রের সরকারী হিসাব বাবদ অর্থপ্রাপ্তি, কিংবা অনুরূপ
    অর্থ রক্ষণাবেক্ষণ বা দান, কিংবা সরকারের হিসাব-নিরীক্ষা;

(চ) উপরি-উক্ত উপ-দফাসমূহে নির্ধারিত যে কোন বিষয়ের অধীন কোন আনুষঙ্গিক বিষয়।

(২) কোন জরিমানা বা অন্য অর্থদণ্ড আরোপ বা রদবদল, কিংবা লাইসেন্স-ফি বা কোন
    কার্যের জন্য ফি বা উসুল আরোপ বা প্রদান কিংবা স্থানীয় উদ্দেশ্যসাধনকল্পে কোন
    স্থানীয় কর্তৃপক্ষ বা প্রতিষ্ঠান কর্তৃক কোন কর আরোপ, নিয়ন্ত্রণ, রদবদল, মওকুফ বা
    রহিতকরণের বিধান করা হইয়াছে, কেবল এই কারণে কোন বিল অর্থবিল বলিয়া গণ্য
    হইবে না।

(৩) রাষ্ট্রপতির সম্মতির জন্য তাঁহার নিকট পেশ করিবার সময়ে প্রত্যেক অর্থবিলে
    স্পীকারের স্বাক্ষরে এই মর্মে একটি সার্টিফিকেটে থাকিবে যে, তাহা একটি অর্থবিল,
    এবং অনুরূপ সার্টিফিকেট সকল বিষয়ে চূড়ান্ত হইবে এবং সেই সম্পর্কে কোন আদালতে
    প্রশ্ন উত্থাপন করা যাইবে না।

\subsection{৮২। আর্থিক ব্যবস্থাবলীর সুপারিশ}
\label{sec:orgd655a67}
কোন অর্থ বিল, অথবা সরকারী অর্থ ব্যয়ের প্রশ্ন জড়িত রহিয়াছে এমন কোন বিল
রাষ্ট্রপতির সুপারিশ ব্যতীত সংসদে উত্থাপন করা যাইবে না:

তবে শর্ত থাকে যে, কোন কর হ্রাস বা বিলোপের বিধান-সংবলিত কোন সংশোধনী
উত্থাপনের জন্য এই অনুচ্ছেদের অধীন সুপারিশের প্রয়োজন হইবে না।

\subsection{৮৩। সংসদের আইন ব্যতীত করারোপে বাধা}
\label{sec:org24defe1}
সংসদের কোন আইনের দ্বারা বা কর্তৃত্ব ব্যতীত কোন কর আরোপ বা সংগ্রহ করা যাইবে
না।

\subsection{৮৪। সংযুক্ত তহবিল ও প্রজাতন্ত্রের সরকারী হিসাব}
\label{sec:org72ea152}
(১) সরকার কর্তৃক প্রাপ্ত সকল রাজস্ব, সরকার কর্তৃক সংগৃহীত সকল ঋণ এবং কোন ঋণ
    পরিশোধ হইতে সরকার কর্তৃক প্রাপ্ত সকল অর্থ একটি মাত্র তহবিলের অংশে পরিণত হইবে
    এবং তাহা “সংযুক্ত তহবিল” নামে অভিহিত হইবে।

(২) সরকার কর্তৃক বা সরকারের পক্ষে প্রাপ্ত অন্য সকল সরকারী অর্থ প্রজাতন্ত্রের
    সরকারী হিসাবে জমা হইবে।

\subsection{৮৫। সরকারী অর্থের নিয়ন্ত্রণ}
\label{sec:org24fae7b}
সরকারী অর্থের রক্ষণাবেক্ষণ, ক্ষেত্রমত সংযুক্ত তহবিলে অর্থ প্রদান বা তাহা হইতে
অর্থ প্রত্যাহার কিংবা প্রজাতন্ত্রের সরকারী হিসাবে অর্থ প্রদান বা তাহা হইতে অর্থ
প্রত্যাহার এবং উপরি-উক্ত বিষয়সমূহের সহিত সংশ্লিষ্ট বা আনুষঙ্গিক সকল বিষয়
সংসদের আইন-দ্বারা এবং অনুরূপ আইনের বিধান না হওয়া পর্যন্ত রাষ্ট্রপতি কর্তৃক
প্রণীত বিধিসমূহ-দ্বারা নিয়ন্ত্রিত হইবে।

\subsection{৮৬। প্রজাতন্ত্রের সরকারী হিসাবে প্রদেয় অর্থ}
\label{sec:orga393295}
প্রজাতন্ত্রের সরকারী হিসাবে জমা হইবে-

(ক) রাজস্ব কিংবা এই সংবিধানের ৮৪ অনুচ্ছেদের (১) দফার কারণে যেরূপ অর্থ সংযুক্ত
    তহবিলের অংশে পরিণত হইবে, তাহা ব্যতীত প্রজাতন্ত্রের কর্মে নিযুক্ত কিংবা
    প্রজাতন্ত্রের বিষয়াবলীর সহিত সংশ্লিষ্ট কোন ব্যক্তি কর্তৃক প্রাপ্ত বা ব্যক্তির নিকট
    জমা রহিয়াছে, এইরূপ সকল অর্থ; অথবা

(খ) যে কোন মোকদ্দমা, বিষয়, হিসাব বা ব্যক্তি বাবদ যে কোন আদালত কর্তৃক প্রাপ্ত
    বা আদালতের নিকট জমা রহিয়াছে, এইরূপ সকল অর্থ।

\subsection{৮৭। বার্ষিক আর্থিক বিবৃতি}
\label{sec:org143b928}
(১) প্রত্যেক অর্থ-বৎসর সম্পর্কে উক্ত বৎসরের জন্য সরকারের অনুমিত আয় ও
    ব্যয়-সংবলিত একটি বিবৃতি (এই ভাগে “বার্ষিক আর্থিক বিবৃতি” নামে অভিহিত) সংসদে
    উপস্থাপিত হইবে।

(২) বার্ষিক আর্থিক বিবৃতিতে পৃথক পৃথকভাবে

(ক) এই সংবিধানের দ্বারা বা অধীন সংযুক্ত তহবিলের উপর দায়রূপে বর্ণিত ব্যয়
    নির্বাহের জন্য প্রয়োজনীয় অর্থ, এবং

(খ) সংযুক্ত তহবিল হইতে ব্যয় করা হইবে, এইরূপ প্রস্তাবিত অন্যান্য ব্যয় নির্বাহের
    জন্য প্রয়োজনীয় অর্থ,

প্রদর্শিত হইবে এবং অন্যান্য ব্যয় হইতে রাজস্ব খাতের ব্যয় পৃথক করিয়া প্রদর্শিত
হইবে।

\subsection{৮৮। সংযুক্ত তহবিলের উপর দায়}
\label{sec:org6b716be}
সংযুক্ত তহবিলের উপর দায়যুক্ত ব্যয় নিম্নরূপ হইবে:

(ক) রাষ্ট্রপতিকে দেয় পারিশ্রমিক ও তাঁহার দপ্তর-সংশ্লিষ্ট অন্যান্য ব্যয়;

৪৯ * * *

(খ) (অ) স্পীকার ও ডেপুটি স্পীকার,

(আ) সুপ্রীম কোর্টের বিচারকগণ,

(ই) মহা হিসাব-নিরীক্ষক ও নিয়ন্ত্রক,

(ঈ) নির্বাচন কমিশনারগণ,

(উ) সরকারী কর্ম কমিশনের সদস্যদিগকে,

দেয় পারিশ্রমিক;

(গ) সংসদ, সুপ্রীম কোর্ট, মহা হিসাব-নিরীক্ষক ও নিয়ন্ত্রকের দপ্তর, নির্বাচন
    কমিশন এবং সরকারী কর্ম কমিশনের কর্মচারীদিগকে দেয় পারিশ্রমিকসহ প্রশাসনিক
    ব্যয়;

(ঘ) সুদ, পরিশোধ-তহবিলের দায়, মূলধন পরিশোধ বা তাহার ক্রম-পরিশোধ এবং
    ঋণসংগ্রহ-ব্যপদেশে ও সংযুক্ত তহবিলের জামানতে গৃহীত ঋণের মোচন-সংক্রান্ত অন্যান্য
    ব্যয়সহ সরকারের ঋণ-সংক্রান্ত সকল দেনার দায়;

(ঙ) কোন আদালত বা ট্রাইব্যুনাল কর্তৃক প্রজাতন্ত্রের বিরুদ্ধে প্রদত্ত কোন রায়, ডিক্রী
    বা রোয়েদাদ কার্যকর করিবার জন্য প্রয়োজনীয় যে কোন পরিমাণ অর্থ; এবং

(চ) এই সংবিধান বা সংসদের আইন দ্বারা অনুরূপ দায়যুক্ত বলিয়া ঘোষিত অন্য যে কোন
    ব্যয়।

\subsection{৮৯। বার্ষিক আর্থিক বিবৃতি সম্পর্কিত পদ্ধতি}
\label{sec:org62bc14d}
(১) সংযুক্ত তহবিলের দায়যুক্ত ব্যয়-সম্পর্কিত বার্ষিক আর্থিক বিবৃতির অংশ সংসদে
    আলোচনা করা হইবে, কিন্তু তাহা ভোটের আওতাভুক্ত হইবে না।

(২) অন্যান্য ব্যয়-সম্পর্কিত বার্ষিক আর্থিক বিবৃতির অংশ মঞ্জুরী-দাবীর আকারে সংসদে
    উপস্থাপিত হইবে এবং কোন মঞ্জুরী-দাবীতে সম্মতিদানের বা সম্মতিদানে অস্বীকৃতির
    কিংবা মঞ্জুরী-দাবীতে নির্ধারিত অর্থ হ্রাস-সাপেক্ষে তাহাতে সম্মতিদানের ক্ষমতা
    সংসদের থাকিবে।

(৩) রাষ্ট্রপতির সুপারিশ ব্যতীত কোন মঞ্জুরী দাবী করা যাইবে না।

\subsection{৯০। নির্দিষ্টকরণ আইন}
\label{sec:org6a7c6b0}
(১) সংসদ কর্তৃক এই সংবিধানের ৮৯ অনুচ্ছেদের অধীন মঞ্জুরী-দানের পর সংযুক্ত
    তহবিল হইতে নিম্নলিখিত ব্যয় নির্বাহের জন্য প্রয়োজনীয় সকল অর্থ নির্দিষ্টকরণের
    বিধান-সংবলিত একটি বিল যথাশীঘ্র সংসদে উত্থাপন করা হইবে:

(ক) সংসদ কর্তৃক প্রদত্ত অনুরূপ মঞ্জুরী; এবং

(খ) সংসদে উপস্থাপিত বিবৃতিতে প্রদর্শিত অর্থের অনধিক সংযুক্ত তহবিলের দায়যুক্ত
    ব্যয়।

(২) অনুরূপ কোন বিল সম্পর্কে সংসদে এমন কোন সংশোধনীর প্রস্তাব করা হইবে না,
    যাহার ফলে অনুরূপভাবে প্রদত্ত কোন মঞ্জুরীর পরিমাণ বা উদ্দেশ্য কিংবা সংযুক্ত
    তহবিলের উপর দায়যুক্ত ব্যয়ের পরিমাণ পরিবর্তিত হইয়া যায়।

(৩) এই সংবিধানের বিধানাবলী-সাপেক্ষে সংযুক্ত তহবিল হইতে এই অনুচ্ছেদের
    বিধানাবলী অনুযায়ী গৃহীত আইনের দ্বারা নির্দিষ্টকরণ ব্যতীত কোন অর্থ প্রত্যাহার
    করা হইবে না।

\subsection{৯১। সম্পূরক ও অতিরিক্ত মঞ্জুরী}
\label{sec:org3732897}
কোন অর্থ-বৎসর প্রসঙ্গে যদি দেখা যায় যে,

(ক) চলিত অর্থ-বৎসরে নির্দিষ্ট কোন কর্মবিভাগের জন্য অনুমোদিত ব্যয় অপর্যাপ্ত
    হইয়াছে কিংবা ঐ বৎসরের বার্ষিক আর্থিক বিবৃতিতে অন্তর্ভুক্ত হয় নাই, এমন কোন নূতন
    কর্মবিভাগের জন্য ব্যয় নির্বাহের প্রয়োজনীয়তা দেখা দিয়াছে, অথবা

(খ) কোন অর্থ-বৎসরে কোন কর্মবিভাগের জন্য মঞ্জুরীকৃত অর্থের অধিক অর্থ ঐ বৎসরে
    উক্ত কর্মবিভাগের জন্য ব্যয়িত হইয়াছে,

তাহা হইলে এই সংবিধানের দ্বারা বা অধীন সংযুক্ত তহবিলের উপর ইহাকে দায়যুক্ত
করা হউক বা না হউক, সংযুক্ত তহবিল হইতে এই ব্যয় নির্বাহের কর্তৃত্ব প্রদান করিবার
ক্ষমতা রাষ্ট্রপতির থাকিবে এবং রাষ্ট্রপতি ক্ষেত্রমত এই ব্যয়ের অনুমিত
পরিমাণ-সংবলিত একটি সম্পূরক আর্থিক বিবৃতি কিংবা অতিরিক্ত ব্যয়ের পরিমাণ সংবলিত
একটি অতিরিক্ত আর্থিক বিবৃতি সংসদে উপস্থাপনের ব্যবস্থা করিবেন এবং বার্ষিক
আর্থিক বিবৃতির ন্যায় উপরি-উক্ত বিবৃতির ক্ষেত্রে (প্রয়োজনীয় উপযোগীকরণসহ) এই
সংবিধানের ৮৭ হইতে ৯০ পর্যন্ত অনুচ্ছেদসমূহ প্রযোজ্য হইবে।

\subsection{৯২। হিসাব, ঋণ প্রভৃতির উপর ভোট}
\label{sec:org87340ce}
(১) এই পরিচ্ছেদের উপরি-উক্ত বিধানাবলীতে যাহা বলা হইয়াছে, তাহা সত্ত্বেও

(ক) মঞ্জুরীর উপর ভোটদান সম্পর্কে এই সংবিধানের ৮৯ অনুচ্ছেদে নির্ধারিত পদ্ধতি
    সম্পন্ন না হওয়া পর্যন্ত এবং ঐ ব্যয় সম্পর্কিত ৯০ অনুচ্ছেদের বিধানাবলী অনুযায়ী আইন
    গৃহীত না হওয়া পর্যন্ত কোন অর্থ বৎসরের কোন অংশের জন্য অনুমিত ব্যয়ের অগ্রিম
    মঞ্জুরীদানের ক্ষমতা সংসদের থাকিবে;

(খ) কোন কার্যের বিশালতা বা অনির্দিষ্ট বৈশিষ্ট্যের জন্য বার্ষিক আর্থিক বিবৃতিতে
    সাধারণভাবে প্রদত্ত বিস্তারিত বৃত্তান্তের সহিত অনুরূপ কার্য-সংক্রান্ত ব্যয়দাবী
    নির্ধারিত করা সম্ভব না হইলে প্রজাতন্ত্রের সম্পদ হইতে অনুরূপ অপ্রত্যাশিত
    ব্যয়নির্বাহের জন্য মঞ্জুরীদানের ক্ষমতা সংসদের থাকিবে;

(গ) কোন অর্থ-বৎসরের চলিত ব্যয়ের অংশ নয়, এইরূপ ব্যতিক্রমী মঞ্জুরীদানের ক্ষমতা
    সংসদের থাকিবে;

এবং যে উদ্দেশ্যে অনুরূপ মঞ্জুরীদান করা হইয়াছে, তাহা সাধনকল্পে সংযুক্ত তহবিল
হইতে আইনের দ্বারা অর্থ প্রত্যাহারের কর্তৃত্ব প্রদানের ক্ষমতা সংসদের থাকিবে।

(২) বার্ষিক আর্থিক বিবৃতিতে উল্লিখিত কোন ব্যয়-সম্পর্কিত মঞ্জুরীদানের ক্ষেত্রে এবং
    অনুরূপ ব্যয় নির্বাহের উদ্দেশ্যে সংযুক্ত তহবিল হইতে অর্থ নির্দিষ্টকরণের কর্তৃত্ব
    প্রদানের জন্য প্রণীতব্য আইনের ক্ষেত্রে এই সংবিধানের ৮৯ ও ৯০ অনুচ্ছেদের
    বিধানাবলী যেরূপ সক্রিয় হইবে, বর্তমান অনুচ্ছেদের (১) দফার অধীন কোন
    মঞ্জুরীদানের ক্ষেত্রে এবং ঐ দফার অধীন প্রণীতব্য কোন আইনের ক্ষেত্রেও উক্ত
    অনুচ্ছেদদ্বয় সমভাবে কার্যকর হইবে।

৫০ (৩) এই পরিচ্ছেদের উপরি-উক্ত বিধানাবলীতে যাহা বলা হইয়াছে, তাহা সত্ত্বেও
যদি কোন অর্থ-বৎসর প্রসঙ্গের সংসদ-

(ক) উক্ত বৎসর আরম্ভ হওয়ার পূর্বে এই সংবিধানের ৮৯ অনুচ্ছেদের অধীন মঞ্জুরীদান
    এবং ৯০ অনুচ্ছেদের অধীন আইন গ্রহণে অসমর্থ হইয়া থাকে এবং এই অনুচ্ছেদের অধীন
    কোন অগ্রিম মঞ্জুরীদান না করিয়া থাকে; অথবা

(খ) কোন ক্ষেত্রে এই অনুচ্ছেদের অধীন কোন মেয়াদের জন্য কোন অগ্রিম মঞ্জুরী দেওয়া
    হইয়া থাকিলে সেই মেয়াদ উত্তীর্ণ হইবার পূর্বে ৮৯ অনুচ্ছেদের অধীন মঞ্জুরীদানে এবং
    ৯০ অনুচ্ছেদের অধীন আইন গ্রহণে অসমর্থ হইয়া থাকে,

তাহা হইলে রাষ্ট্রপতি প্রধানমন্ত্রীর পরামর্শক্রমে, আদেশের দ্বারা অনুরূপ মঞ্জুরীদান
না করা এবং আইন গৃহীত না হওয়া পর্যন্ত, ঐ বৎসরের অনধিক ষাট দিন মেয়াদ পর্যন্ত
উক্ত বৎসরের আর্থিক বিবৃতিতে উল্লিখিত ব্যয় নির্বাহের জন্য সংযুক্ত তহবিল হইতে অর্থ
প্রত্যাহারের কর্তৃত্ব প্রদান করিতে পারিবেন।

\subsection{৯২ক। [বিলুপ্ত]}
\label{sec:org9cb4b46}
\subsection{৯৩। অধ্যাদেশপ্রণয়ন-ক্ষমতা}
\label{sec:org6eda3f1}
(১) ৫৩ সংসদ ভাঙ্গিয়া যাওয়া অবস্থায় অথবা উহার অধিবেশনকাল ব্যতীত কোন সময়ে
    রাষ্ট্রপতির নিকট আশু ব্যবস্থা গ্রহণের জন্য প্রয়োজনীয় পরিস্থিতি বিদ্যমান রহিয়াছে
    বলিয়া সন্তোষজনকভাবে প্রতীয়মান হইলে তিনি উক্ত পরিস্থিতিতে যেরূপ প্রয়োজনীয়
    বলিয়া মনে করিবেন, সেইরূপ অধ্যাদেশ প্রণয়ন ও জারী করিতে পারিবেন এবং জারী
    হইবার সময় হইতে অনুরূপভাবে প্রণীত অধ্যাদেশ সংসদের আইনের ন্যায় ক্ষমতাসম্পন্ন
    হইবে:

তবে শর্ত থাকে যে, এই দফার অধীন কোন অধ্যাদেশে এমন কোন বিধান করা হইবে না,

(ক) যাহা এই সংবিধানের অধীন সংসদের আইন-দ্বারা আইনসঙ্গতভাবে করা যায় না;

(খ) যাহাতে এই সংবিধানের কোন বিধান পরিবর্তিত বা রহিত হইয়া যায়; অথবা

(খ) যাহার দ্বারা পূর্বে প্রণীত কোন অধ্যাদেশের যে কোন বিধানকে অব্যাহতভাবে
    বলবৎ করা যায়।

(২) এই অনুচ্ছেদের (১) দফার অধীন প্রণীত কোন অধ্যাদেশ জারী হইবার পর অনুষ্ঠিত
    সংসদের প্রথম বৈঠকে তাহা উপস্থাপিত হইবে এবং ইতঃপূর্বে বাতিল না হইয়া থাকিলে
    অধ্যাদেশটি অনুরূপভাবে উপস্থাপনের পর ত্রিশ দিন অতিবাহিত হইলে কিংবা অনুরূপ
    মেয়াদ উত্তীর্ণ হইবার পূর্বে তাহা অননুমোদন করিয়া সংসদে প্রস্তাব গৃহীত হইলে
    অধ্যাদেশটির কার্যকরতা লোপ পাইবে।

(৩) সংসদ ভাঙ্গিয়া যাওয়া অবস্থা কোন সময়ে রাষ্ট্রপতির নিকট ব্যবস্থা-গ্রহণের জন্য
    প্রয়োজনীয় পরিস্থিতি বিদ্যমান রহিয়াছে বলিয়া সন্তোষজনকভাবে প্রতীয়মান হইলে
    তিনি এমন অধ্যাদেশ প্রণয়ন ও জারী করিতে পারিবেন, যাহাতে সংবিধান-দ্বারা
    সংযুক্ত তহবিলের উপর কোন ব্যয় দায়যুক্ত হউক বা না হউক, উক্ত তহবিল হইতে সেইরূপ
    ব্যয়নির্বাহের কতর্ৃত্ব প্রদান করা যাইবে এবং অনুরূপভাবে প্রণীত কোন অধ্যাদেশ জারী
    হইবার সময় হইতে তাহা সংসদের আইনের ন্যায় ক্ষমতাসম্পন্ন হইবে।

(৪) এই অনুচ্ছেদের (৩) দফার অধীন জারীকৃত প্রত্যেক অধ্যাদেশ যথাশীঘ্র সংসদে
    উপস্থাপিত হইবে এবং সংসদ পুনর্গঠিত হইবার তারিখ হইতে ত্রিশ দিনের মধ্যে এই
    সংবিধানের ৮৭, ৮৯ ও ৯০ অনুচ্ছেদসমূহের বিধানাবলী প্রয়োজনীয় উপযোগীকরণসহ পালিত
    হইবে।

\subsection{৯৪। সুপ্রীম কোর্ট প্রতিষ্ঠা}
\label{sec:orgaefd4ce}
(১) “বাংলাদেশ সুপ্রীম কোর্ট” নামে বাংলাদেশের একটি সর্বোচ্চ আদালত থাকিবে
    এবং আপীল বিভাগ ও হাইকোর্ট বিভাগ লইয়া তাহা গঠিত হইবে।

(২) প্রধান বিচারপতি (যিনি “বাংলাদেশের প্রধান বিচারপতি” নামে অভিহিত
    হইবেন) এবং প্রত্যেক বিভাগে আসন গ্রহণের জন্য রাষ্ট্রপতি যেরূপ সংখ্যক বিচারক
    নিয়োগের প্রয়োজন বোধ করিবেন, সেইরূপ সংখ্যক অন্যান্য বিচারক লইয়া সুপ্রীম কোর্ট
    গঠিত হইবে।

(৩) প্রধান বিচারপতি ও আপীল বিভাগে নিযুক্ত বিচারকগণ কেবল উক্ত বিভাগে এবং
    অন্যান্য বিচারক কেবল হাইকোর্ট বিভাগে আসন গ্রহণ করিবেন।

(৪) এই সংবিধানের বিধানাবলী-সাপেক্ষে প্রধান বিচারপতি এবং অন্যান্য বিচারক
    বিচারকার্য পালনের ক্ষেত্রে স্বাধীন থাকিবেন।

\subsection{৯৫। বিচারক-নিয়োগ}
\label{sec:org72ac0b0}
(১) প্রধান বিচারপতি এবং অন্যান্য বিচারকগণ রাষ্ট্রপতি কর্তৃক নিযুক্ত হইবেন।

(২) কোন ব্যক্তি বাংলাদেশের নাগরিক না হইলে, এবং-

(ক) সুপ্রীম কোর্টে অন্যূন দশ বৎসরকাল এ্যাডভোকেট না থাকিয়া থাকিলে; অথবা

(খ) বাংলাদেশের রাষ্ট্রীয় সীমানার মধ্যে অন্যূন দশ বৎসরকাল কোন বিচার বিভাগীয়
    পদে অধিষ্ঠান না করিয়া থাকিলে; অথবা

(গ) সুপ্রীম কোর্টের বিচারক পদে নিয়োগ লাভের জন্য আইনের দ্বারা নির্ধারিত
    অন্যান্য যোগ্যতা না থাকিয়া থাকিলে;

তিনি বিচারকপদে নিয়োগ লাভের যোগ্য হইবেন না।

(৩) এই অনুচ্ছেদে “সুপ্রীম কোর্ট” বলিতে ১৯৭৭ সালের দ্বিতীয় ফরমান (দশম সংশোধন)
    আদেশ প্রবর্তনের পূর্বে যে কোন সময়ে বর্তমানে বাংলাদেশের অন্তর্ভুক্ত এলাকার মধ্যে
    যে আদালত হাইকোর্ট অথবা সুপ্রীম কোর্ট হিসাবে এখতিয়ার প্রয়োগ করিয়াছেন সেই
    আদালত অন্তর্ভুক্ত হইবে।

\subsection{৯৬। বিচারকের পদের মেয়াদ}
\label{sec:org9a25a5e}
(১) এই অনুচ্ছেদের অন্যান্য বিধানাবলী সাপেক্ষে, কোন বিচারক ৫৭ সাতষট্টি বৎসর
    বয়স পূর্ণ হওয়া পর্যন্ত স্বীয় পদে বহাল থাকিবেন।

(২) এই অনুচ্ছেদের নিম্নরূপ বিধানাবলী অনুযায়ী ব্যতীত কোন বিচারককে তাঁহার পদ
    হইতে অপসারিত করা যাইবে না।

(৩) একটি সুপ্রীম জুডিশিয়াল কাউন্সিল থাকিবে যাহা এই অনুচ্ছেদে “কাউন্সিল” বলিয়া
    উল্লিখিত হইবে এবং বাংলাদেশের প্রধান বিচারপতি এবং অন্যান্য বিচারকের মধ্যে
    পরবর্তী যে দুইজন কর্মে প্রবীন তাঁহাদের লইয়া গঠিত হইবে:

তবে শর্ত থাকে যে, কাউন্সিল যদি কোন সময়ে কাউন্সিলের সদস্য এইরূপ কোন বিচারকের
সামর্থ্য বা আচরণ সম্পর্কে তদন্ত করেন, অথবা কাউন্সিলের কোন সদস্য যদি অনুপস্থিত
থাকেন অথবা অসুস্থতা কিংবা অন্য কোন কারণে কার্য করিতে অসমর্থ হন তাহা হইলে
কাউন্সিলের যাঁহারা সদস্য আছেন তাঁহাদের পরবর্তী যে বিচারক কর্মে প্রবীণ তিনিই
অনুরূপ সদস্য হিসাবে কার্য করিবেন।

(৪) কাউন্সিলের দায়িত্ব হইবে-

(ক) বিচারকগণের জন্য পালনীয় আচরণ বিধি নির্ধারণ করা, এবং

(খ) কোন বিচারকের অথবা কোন বিচারক যেরূপ পদ্ধতিতে অপসারিত হইতে পারেন সেইরূপ
    পদ্ধতি ব্যতীত তাঁহার পদ হইতে অপসারণ যোগ্য নহেন এইরূপ অন্য কোন কর্মকর্তার
    সামর্থ্য বা আচরণ সম্পর্কে তদন্ত করা।

(৫) যে ক্ষেত্রে কাউন্সিল অথবা অন্য কোন সূত্র হইতে প্রাপ্ত তথ্যে রাষ্ট্রপতির এইরূপ
    বুঝিবার কারণ থাকে যে কোন বিচারক-

(ক) শারীরিক বা মানসিক অসামর্থ্যের কারণে তাঁহার পদের দায়িত্ব সঠিকভাবে পালন
    করিতে অযোগ্য হইয়া পড়িতে পারেন, অথবা

(খ) গুরুতর অসদাচরণের জন্য দোষী হইতে পারেন, সেইক্ষেত্রে রাষ্ট্রপতি কাউন্সিলকে
    বিষয়টি সম্পর্কে তদন্ত করিতে ও উহার তদন্তফল জ্ঞাপন করিবার জন্য নির্দেশ দিতে
    পারেন।

(৬) কাউন্সিল তদন্ত করিবার পর রাষ্ট্রপতির নিকট যদি এইরূপ রিপোর্ট করেন যে,
    উহার মতে উক্ত বিচারক তাঁহার পদের দায়িত্ব সঠিকভাবে পালনে অযোগ্য হইয়া
    পড়িয়াছেন অথবা গুরুতর অসদাচরণের জন্য দোষী হইয়াছেন তাহা হইলে রাষ্ট্রপতি
    আদেশের দ্বারা উক্ত বিচারককে তাঁহার পদ হইতে অপসারিত করিবেন।

(৭) এই অনুচ্ছেদের অধীনে তদন্তের উদ্দেশ্যে কাউন্সিল স্বীয় কার্য-পদ্ধতি নিয়ন্ত্রণ
    করিবেন এবং পরওয়ানা জারী ও নির্বাহের ব্যাপারে সুপ্রীম কোর্টের ন্যায় উহার একই
    ক্ষমতা থাকিবে।

(৮) কোন বিচারক রাষ্ট্রপতিকে উদ্দেশ করিয়া স্বাক্ষরযুক্ত পত্রযোগে স্বীয় পদ ত্যাগ
    করিতে পারিবেন।

\subsection{৯৭। অস্থায়ী প্রধান বিচারপতি নিয়োগ}
\label{sec:orgcec0b3c}
প্রধান বিচারপতির পদ শূন্য হইলে কিংবা অনুপস্থিতি, অসুস্থতা বা অন্য কোন কারণে
প্রধান বিচারপতি তাঁহার দায়িত্বপালনে অসমর্থ বলিয়া রাষ্ট্রপতির নিকট
সন্তোষজনকভাবে প্রতীয়মান হইলে ক্ষেত্রমত অন্য কোন ব্যক্তি অনুরূপ পদে যোগদান না
করা পর্যন্ত কিংবা প্রধান বিচারপতি স্বীয় কার্যভার পুনরায় গ্রহণ না করা পর্যন্ত
আপীল বিভাগের অন্যান্য বিচারকের মধ্যে যিনি কর্মে প্রবীণতম, তিনি অনুরূপ
কার্যভার পালন করিবেন।

\subsection{৯৮। সুপ্রীম কোর্টের অতিরিক্ত বিচারকগণ}
\label{sec:org94f18a8}
সংবিধানের ৯৪ অনুচ্ছেদের বিধানাবলী সত্ত্বেও ৫৮ * * * রাষ্ট্রপতির নিকট সুপ্রীম
কোর্টের কোন বিভাগের বিচারক-সংখ্যা সাময়িকভাবে বৃদ্ধি করা উচিত বলিয়া
সন্তোষজনকভাবে প্রতীয়মান হইলে তিনি যথাযথ যোগ্যতাসম্পন্ন এক বা একাধিক
ব্যক্তিকে অনধিক দুই বৎসরের জন্য অতিরিক্ত বিচারক নিযুক্ত করিতে পারিবেন, কিংবা
তিনি উপযুক্ত বিবেচনা করিলে হাইকোর্ট বিভাগের কোন বিচারককে ৫৯ একজন এ্যাডহক
বিচারক হিসাবে যে কোন অস্থায়ী মেয়াদের জন্য আপীল বিভাগে আসন গ্রহণের ব্যবস্থা
করিতে পারিবেন, এবং অনুরূপ বিচারক এইরূপ আসন গ্রহণকালে আপীল বিভাগের একজন
বিচারকের ন্যায় একই এখতিয়ার ও ক্ষমতা প্রয়োগ ও দায়িত্ব পালন করিবেন:

তবে শর্ত থাকে যে, অতিরিক্ত বিচারকরূপে নিযুক্ত (কোন ব্যক্তিকে এই সংবিধানের ৯৫
অনুচ্ছেদের অধীন বিচারকরূপে নিযুক্ত) হইতে কিংবা বর্তমান অনুচ্ছেদের অধীন আরও এক
মেয়াদের জন্য অতিরিক্ত বিচারকরূপে নিযুক্ত হইতে বর্তমান অনুচ্ছেদের কোন কিছুই
নিবৃত্ত করিবে না।

\subsection{৯৯। অবসর গ্রহণের পর বিচারগণের অক্ষমতা}
\label{sec:orgc77befa}
(১) (২) দফায় ব্যবস্থিত বিধান ব্যতিরেকে, কোন ব্যক্তি অতিরিক্ত বিচারকরূপে
দায়িত্ব পালন ব্যতীত বিচারক পদে দায়িত্ব পালন করিয়া থাকিলে উক্ত পদ হইতে অবসর
গ্রহণের কিংবা অপসারিত হইবার পর তিনি কোন আদালত বা কর্তৃপক্ষের নিকট ওকালতি
বা কার্য করিতে পারিবেন না অথবা বিচার বিভাগীয় বা ৬১ আধা-বিচার বিভাগীয় পদ
অথবা প্রধান উপদেষ্টা বা উপদেষ্টার পদ ব্যতীত প্রজাতন্ত্রের কর্মে কোন লাভজনক পদে
বহাল হইবেন না।

(২) কোন ব্যক্তি হাইকোর্ট বিভাগের বিচারক পদে বহাল থাকিলে উক্ত পদ হইতে অবসর
    গ্রহণের বা অপসারিত হইবার পর তিনি আপীল বিভাগে ওকালতি বা কার্য করিতে
    পারিবেন।

\subsection{১০০। সুপ্রীম কোর্টের আসন}
\label{sec:orgf851114}
রাজধানীতে সুপ্রীম কোর্টের স্থায়ী আসন থাকিবে, তবে রাষ্ট্রপতির অনুমোদন লইয়া
প্রধান বিচারপতি সময়ে সময়ে অন্য যে স্থান বা স্থানসমূহ নির্ধারণ করিবেন, সেই
স্থান বা স্থানসমূহে হাইকোর্ট বিভাগের অধিবেশন অনুষ্ঠিত হইতে পারিবে।

\subsection{১০১। হাইকোর্ট বিভাগের এখতিয়ার}
\label{sec:org7d10f88}
এই সংবিধান বা অন্য কোন আইনের দ্বারা হাইকোর্ট বিভাগের উপর যেরূপ আদি, আপীল
ও অন্য প্রকার এখতিয়ার, ক্ষমতা ও দায়িত্ব অর্পিত হইয়াছে বা হইতে পারে উক্ত
বিভাগের সেইরূপ এখতিয়ার, ক্ষমতা ও দায়িত্ব থাকিবে।

\subsection{১০২। কতিপয় আদেশ ও নির্দেশ প্রভৃতি দানের ক্ষেত্রে হাইকোর্ট বিভাগের ক্ষমতা}
\label{sec:org03f9888}
(১) কোন সংক্ষুব্ধ ব্যক্তির আবেদনক্রমে এই সংবিধানের তৃতীয় ভাগের দ্বারা অর্পিত
    অধিকারসমূহের যে কোন একটি বলবৎ করিবার জন্য প্রজাতন্ত্রের বিষয়াবলীর সহিত
    সম্পর্কিত কোন দায়িত্ব পালনকারী ব্যক্তিসহ যে কোন ব্যক্তি বা কর্তৃপক্ষকে হাইকোর্ট
    বিভাগ উপযুক্ত নির্দেশাবলী বা আদেশাবলী দান করিতে পারিবেন।

(২) হাইকোর্ট বিভাগের নিকট যদি সন্তোষজনকভাবে প্রতীয়মান হয় যে, আইনের দ্বারা
    অন্য কোন সমফলপ্রদ বিধান করা হয় নাই, তাহা হইলে

(ক) যে কোন সংক্ষুব্ধ ব্যক্তির আবেদনক্রমে

(অ) প্রজাতন্ত্র বা কোন স্থানীয় কর্তৃপক্ষের বিষয়াবলীর সহিত সংশ্লিষ্ট যে কোন
দায়িত্ব পালনে রত ব্যক্তিকে আইনের দ্বারা অনুমোদিত নয়, এমন কোন কার্য করা হইতে
বিরত রাখিবার জন্য কিংবা আইনের দ্বারা তাঁহার করণীয় কার্য করিবার জন্য নির্দেশ
প্রদান করিয়া; অথবা

(আ) প্রজাতন্ত্র বা কোন স্থানীয় কর্তৃপক্ষের বিষয়াবলীর সহিত সংশ্লিষ্ট যে কোন
দায়িত্ব পালনে রত ব্যক্তির কৃত কোন কার্য বা গৃহীত কোন কার্যধারা আইনসংগত কর্তৃত্ব
ব্যতিরেকে করা হইয়াছে বা গৃহীত হইয়াছে ও তাহার কোন আইনগত কার্যকরতা নাই
বলিয়া ঘোষণা করিয়া;

উক্ত বিভাগ আদেশদান করিতে পারিবেন; অথবা

(খ) যে কোন ব্যক্তির আবেদনক্রমে

(অ) আইনসংগত কর্তৃত্ব ব্যতিরেকে বা বেআইনী উপায়ে কোন ব্যক্তিকে প্রহরায় আটক রাখা
হয় নাই বলিয়া যাহাতে উক্ত বিভাগের নিকট সন্তোষজনকভাবে প্রতীয়মান হইতে পারে
সেইজন্য প্রহরায় আটক উক্ত ব্যক্তিকে উক্ত বিভাগের সম্মুখে আনয়নের নির্দেশ প্রদান
করিয়া; অথবা

(আ) কোন সরকারী পদে আসীন বা আসীন বলিয়া বিবেচিত কোন ব্যক্তিকে তিনি কোন্
কর্তৃত্ববলে অনুরূপ পদমর্যাদায় অধিষ্ঠানের দাবী করিতেছেন, তাহা প্রদর্শনের নির্দেশ
প্রদান করিয়া;

উক্ত বিভাগ আদেশদান করিতে পারিবেন।

(৩) উপরি-উক্ত দফাসমূহে যাহা বলা হইয়াছে, তাহা সত্ত্বেও এই সংবিধানের ৪৭
    অনুচ্ছেদ প্রযোজ্য হয়, এইরূপ কোন আইনের ক্ষেত্রে বর্তমান অনুচ্ছেদের অধীন
    অন্তর্বর্তীকালীন বা অন্য কোন আদেশদানের ক্ষমতা হাইকোর্ট বিভাগের থাকিবে না।

(৪) এই অনুচ্ছেদের (১) দফা কিংবা এই অনুচ্ছেদের (২) দফার (ক) উপ-দফার অধীন
    কোন আবেদনক্রমে যে ক্ষেত্রে অন্তর্বর্তী আদেশ প্রার্থনা করা হইয়াছে এবং অনুরূপ
    অন্তর্বর্তী আদেশ

(ক) যেখানে উন্নয়ন কর্মসূচী বাস্তবায়নের জন্য কোন ব্যবস্থার কিংবা কোন উন্নয়নমূলক
    কার্যের প্রতিকূলতা বা বাধা সৃষ্টি করিতে পারে; অথবা

(খ) যেখানে অন্য কোনভাবে জনস্বার্থের পক্ষে ক্ষতিকর হইতে পারে,
    সেইখানে অ্যাটর্ণি-জেনারেলকে উক্ত আবেদন সম্পর্কে যুক্তিসংগত নোটিশদান এবং
    অ্যাটর্ণি-জেনারেলের (কিংবা এই বিষয়ে তাঁহার দ্বারা ভারপ্রাপ্ত অন্য কোন
    এ্যাডভোকেটের) বক্তব্য শ্রবণ না করা পর্যন্ত এবং এই দফার (ক) বা (খ) উপ-দফায়
    উল্লেখিত প্রতিক্রিয়া সৃষ্টি করিবে না বলিয়া হাইকোর্ট বিভাগের নিকট
    সন্তোষজনকভাবে প্রতীয়মান না হওয়া পর্যন্ত উক্ত বিভাগ কোন অন্তর্বর্তী আদেশদান
    করিবেন না।

(৫) প্রসংগের প্রয়োজনে অন্যরূপ না হইলে এই অনুচ্ছেদে “ব্যক্তি” বলিতে সংবিধিদ্ধ
    সরকারী কর্তৃপক্ষ ও বাংলাদেশের প্রতিরক্ষা কর্মবিভাগসমূহ বা কোন শৃঙ্খলা বাহিনী
    সংক্রান্ত আইনের অধীন প্রতিষ্ঠিত কোন আদালত বা ট্রাইব্যুনাল ব্যতীত কিংবা এই
    সংবিধানের ১১৭ অনুচ্ছেদ প্রযোজ্য হয়, এইরূপ কোন ট্রাইব্যুনাল ব্যতীত যে কোন আদালত
    বা ট্রাইব্যুনাল অন্তর্ভুক্ত হইবে।

\subsection{১০৩। আপীল বিভাগের এখতিয়ার}
\label{sec:org36d0b72}
(১) হাইকোর্ট বিভাগের রায়, ডিক্রী, আদেশ বা দণ্ডাদেশের বিরুদ্ধে আপীল শুনানীর
    ও তাহা নিষ্পত্তির এখতিয়ার আপীল বিভাগের থাকিবে।

(২) হাইকোর্ট বিভাগের রায়, ডিক্রী, আদেশ বা দণ্ডাদেশের বিরুদ্ধে আপীল বিভাগের
    নিকট সেই ক্ষেত্রে অধিকারবলে আপীল করা যাইবে, যে ক্ষেত্রে হাইকোর্ট বিভাগ

(ক) এই মর্মে সার্টিফিকেট দান করিবেন যে, মামলাটির সহিত এই সংবিধান-ব্যাখ্যার
    বিষয়ে আইনের গুরুত্বপূর্ণ প্রশ্ন জড়িত রহিয়াছে; অথবা

৬৪ (খ) কোন ব্যক্তিকে মৃত্যুদণ্ডে বা যাবজ্জীবন কারাদণ্ডে দণ্ডিত করিয়াছেন; অথবা

(গ) উক্ত বিভাগের অবমাননার জন্য কোন ব্যক্তিকে দণ্ডদান করিয়াছেন;

এবং সংসদে আইন-দ্বারা যেরূপ বিধান করা হইবে, সেইরূপ অন্যান্য ক্ষেত্রে।

(৩) হাইকোর্ট বিভাগের রায়, ডিক্রী, আদেশ বা দণ্ডাদেশের বিরুদ্ধে যে মামলায় এই
    অনুচ্ছেদের (২) দফা প্রযোজ্য নহে, কেবল আপীল বিভাগ আপীলের অনুমতিদান করিলে সেই
    মামলায় আপীল চলিবে।

(৪) সংসদ আইনের দ্বারা ঘোষণা করিতে পারিবেন যে, এই অনুচ্ছেদের বিধানসমূহ
    হাইকোর্ট বিভাগের প্রসঙ্গে যেরূপ প্রযোজ্য, অন্য কোন আদালত বা ট্রাইব্যুনালের
    ক্ষেত্রেও তাহা সেইরূপ প্রযোজ্য হইবে।

\subsection{১০৪। আপীল বিভাগের পরোয়ানা জারী ও নির্বাহ}
\label{sec:org323ef23}
কোন ব্যক্তির হাজিরা কিংবা কোন দলিলপত্র উদ্ঘাটন বা দাখিল করিবার আদেশসহ
আপীল বিভাগের নিকট বিচারাধীন যে কোন মামলা বা বিষয়ে সম্পূর্ণ ন্যায়বিচারের
জন্য যেরূপ প্রয়োজনীয় হইতে পারে, উক্ত বিভাগ সেইরূপ নির্দেশ, আদেশ, ডিক্রী বা
রীট জারী করিতে পারিবেন।

\subsection{১০৫। আপীল বিভাগ কর্তৃক রায় বা আদেশ পুনর্বিবেচনা}
\label{sec:orgac48555}
সংসদের যে কোন আইনের বিধানাবলী-সাপেক্ষে এবং আপীল বিভাগ কর্তৃক প্রণীত যে কোন
বিধি-সাপেক্ষে আপীল বিভাগের কোন ঘোষিত রায় বা প্রদত্ত আদেশ পুনর্বিবেচনার
ক্ষমতা উক্ত বিভাগের থাকিবে।

\subsection{১০৬। সুপ্রীম কোর্টের উপদেষ্টামূলক এখতিয়ার}
\label{sec:org2b033dc}
যদি কোন সময়ে রাষ্ট্রপতির নিকট প্রতীয়মান হয় যে, আইনের এইরূপ কোন প্রশ্ন
উত্থাপিত হইয়াছে বা উত্থাপনের সম্ভাবনা দেখা দিয়াছে, যাহা এমন ধরণের ও এমন
জন-গুরুত্বসম্পন্ন যে, সেই সম্পর্কে সুপ্রীম কোর্টের মতামত গ্রহণ করা প্রয়োজন, তাহা
হইলে তিনি প্রশ্নটি আপীল বিভাগের বিবেচনার জন্য প্রেরণ করিতে পারিবেন এবং উক্ত
বিভাগ স্বীয় বিবেচনায় উপযুক্ত শুনানীর পর প্রশ্নটি সম্পর্কে রাষ্ট্রপতিকে স্বীয়
মতামত জ্ঞাপন করিতে পারিবেন।

\subsection{১০৭। সুপ্রীম কোর্টের বিধি-প্রণয়ন-ক্ষমতা}
\label{sec:org4f51dce}
(১) সংসদ কর্তৃক প্রণীত যে কোন আইন-সাপেক্ষে সুপ্রীম কোর্ট রাষ্ট্রপতির অনুমোদন
    লইয়া প্রত্যেক বিভাগের এবং অধঃস্তন যে কোন আদালতের রীতি ও পদ্ধতি-নিয়ন্ত্রণের
    জন্য বিধিসমূহ প্রণয়ন করিতে পারিবেন।

(২) সুপ্রীম কোর্ট এই অনুচ্ছেদের (১) দফা এবং এই সংবিধানের ৬৫ ১১৩ অনুচ্ছেদের
    অধীন দায়িত্বসমূহের ভার উক্ত আদালতের কোন একটি বিভাগকে কিংবা এক বা একাধিক
    বিচারককে অর্পণ করিতে পারিবেন।

(৩) এই অনুচ্ছেদের অধীন প্রণীত বিধিসমূহ-সাপেক্ষে কোন্ কোন্ বিচারককে লইয়া ৬৬
    কোন্ বিভাগের কোন্ বেঞ্চ গঠিত হইবে এবং কোন্ কোন্ বিচারক কোন্ উদ্দেশ্যে আসনগ্রহণ
    করিবেন, তাহা প্রধান বিচারপতি নির্ধারণ করিবেন।

(৪) প্রধান বিচারপতি সুপ্রীম কোর্টের যে কোন বিভাগের কর্মে প্রবীণতম বিচারককে
    সেই বিভাগে এই অনুচ্ছেদের (৩) দফা কিংবা এই অনুচ্ছেদের অধীন প্রণীত
    বিধিসমূহ-দ্বারা অর্পিত যে কোন ক্ষমতা প্রয়োগের ভার প্রদান করিতে পারিবেন।

\subsection{১০৮। "কোর্ট অব রেকর্ড" রূপে সুপ্রীম কোর্ট}
\label{sec:orgc0556fa}
সুপ্রীম কোর্ট একটি “কোর্ট অব্ রেকর্ড” হইবেন এবং ইহার অবমাননার জন্য তদন্তের
আদেশদান বা দণ্ডাদেশদানের ক্ষমতাসহ আইন-সাপেক্ষে অনুরূপ আদালতের সকল ক্ষমতার
অধিকারী থাকিবেন।

\subsection{১০৯। আদালতসমূহের উপর তত্ত্বাবধান ও নিয়ন্ত্রণ}
\label{sec:org9e63a56}
হাইকোর্ট বিভাগের অধঃস্তন সকল ৬৭ আদালত ও ট্রাইব্যুনালের উপর উক্ত বিভাগের
তত্ত্বাবধান ও নিয়ন্ত্রণ-ক্ষমতা থাকিবে।

\subsection{১১০। অধস্তন আদালত হইতে হাইকোর্ট বিভাগে মামলা স্থানান্তর}
\label{sec:org8595ce8}
হাইকোর্ট বিভাগের নিকট যদি সন্তোষজনকভাবে প্রতীয়মান হয় যে, উক্ত বিভাগের কোন
অধঃস্তন আদালতের বিচারাধীন কোন মামলায় এই সংবিধানের ব্যাখ্যা-সংক্রান্ত আইনের
এমন গুরুত্বপূর্ণ প্রশ্ন বা এমন জন-গুরুত্বসম্পন্ন বিষয় জড়িত রহিয়াছে, সংশ্লিষ্ট
মামলার মীমাংসার জন্য যাহার সম্পর্কে সিদ্ধান্ত গ্রহণ প্রয়োজন, তাহা হইলে
হাইকোর্ট বিভাগ উক্ত আদালত হইতে মামলাটি প্রত্যাহার করিয়া লইবেন এবং

(ক) স্বয়ং মামলাটির মীমাংসা করিবেন; অথবা

(খ) উক্ত আইনের প্রশ্নটির নিষ্পত্তি করিবেন এবং উক্ত প্রশ্ন সম্বন্ধে হাইকোর্ট
    বিভাগের রায়ের নকলসহ যে আদালত হইতে মামলাটি প্রত্যাহার করা হইয়াছিল, সেই
    আদালতে (বা অন্য কোন অধঃস্তন আদালতে) মামলাটি ফেরত পাঠাইবেন এবং তাহা প্রাপ্ত
    হইবার পর সেই আদালত উক্ত রায়ের সহিত সঙ্গতি রক্ষা করিয়া মামলাটির মীমাংসা
    করিতে প্রবৃত্ত হইবেন।

\subsection{১১১। সুপ্রীম কোর্টের রায়ের বাধ্যতামূলক কার্যকরতা}
\label{sec:org5d34dbb}
আপীল বিভাগ কর্তৃক ঘোষিত আইন হাইকোর্ট বিভাগের জন্য এবং সুপ্রীম কোর্টের যে কোন
বিভাগ কর্তৃক ঘোষিত আইন অধঃস্তন সকল আদালতের জন্য অবশ্যপালনীয় হইবে।

\subsection{১১২। সুপ্রীম কোর্টের সহায়তা}
\label{sec:org9da219c}
প্রজাতন্ত্রের রাষ্ট্রীয় সীমানার অন্তর্ভুক্ত সকল নির্বাহী ও বিচার বিভাগীয় কর্তৃপক্ষ
সুপ্রীম কোর্টের সহায়তা করিবেন।

\subsection{১১৩। সুপ্রীম কোর্টের কর্মচারীগণ}
\label{sec:org8bbe5ec}
(১) প্রধান বিচারপতি কিংবা তাঁহার নির্দেশক্রমে অন্য কোন বিচারক বা কর্মচারী
    সুপ্রীম কোর্টের কর্মচারীদিগকে নিযুক্ত করিবেন এবং রাষ্ট্রপতির পূর্বানুমোদনক্রমে
    সুপ্রীম কোর্ট কর্তৃক প্রণীত বিধিসমূহ-অনুযায়ী এই নিয়োগদান করা হইবে।

(২) সংসদের যে কোন আইনের বিধানাবলী-সাপেক্ষে সুপ্রীম কোর্ট কর্তৃক প্রণীত
    বিধিসমূহে যেরূপ নির্ধারিত হইবে, সুপ্রীম কোর্টের কর্মচারীদের কর্মের শর্তাবলী
    সেইরূপ হইবে।

\subsection{১১৪। অধস্তন আদালত-সমূহ প্রতিষ্ঠা}
\label{sec:orgbb09beb}
আইনের দ্বারা যেরূপ প্রতিষ্ঠিত হইবে, সুপ্রীম কোর্ট ব্যতীত সেইরূপ অন্যান্য অধস্তন
আদালত থাকিবে।

\subsection{১১৫। অধস্তন আদালতে নিয়োগ}
\label{sec:orgb872a45}
বিচারবিভাগীয় পদে বা বিচার বিভাগীয় দায়িত্ব পালনকারী ম্যাজিষ্ট্রেট পদে
রাষ্ট্রপতি কর্তৃক উক্ত উদেশ্যে প্রণীত বিধিসমূহ অনুযায়ী রাষ্ট্রপতি নিয়োগদান
করিবেন।

\subsection{১১৬। অধস্তন আদালতসমূহের নিয়ন্ত্রণ ও শৃঙ্খলা}
\label{sec:orge89a0dc}
বিচার-কর্মবিভাগে নিযুক্ত ব্যক্তিদের এবং বিচারবিভাগীয় দায়িত্বপালনে রত
ম্যাজিষ্ট্রেটদের নিয়ন্ত্রণ (কর্মস্থল-নির্ধারণ, পদোন্নতিদান ও ছুটি-মঞ্জুরীসহ) ও
শৃঙ্খলাবিধান ৬৯ রাষ্ট্রপতির উপর ন্যস্ত থাকিবে ৭০ এবং সুপ্রীম কোর্টের সহিত
পরামর্শক্রমে রাষ্ট্রপতি কর্তৃক তাহা প্রযুক্ত হইবে।

\subsection{১১৬ক। বিচারবিভাগীয় কর্মচারীগণ বিচারকার্য পালনের ক্ষেত্রে স্বাধীন}
\label{sec:org9d400ea}
এই সংবিধানের বিধানাবলী সাপেক্ষে বিচার-কর্মবিভাগে নিযুক্ত ব্যক্তিগণ এবং
ম্যাজিষ্ট্রেটগণ বিচারকার্য পালনের ক্ষেত্রে স্বাধীন থাকিবেন।

\subsection{১১৭। প্রশাসনিক ট্রাইব্যুনালসমূহ}
\label{sec:orga920569}
(১) ইতঃপূর্বে যাহা বলা হইয়াছে, তাহা সত্ত্বেও নিম্নলিখিত ক্ষেত্রসমুহ সম্পর্কে বা
    ক্ষেত্রসমুহ হইতে উদ্ভূত বিষয়াদির উপর এখতিয়ার প্রয়োগের জন্য সংসদ আইনের দ্বারা
    এক বা একাধিক প্রশাসনিক ট্রাইব্যুনাল প্রতিষ্ঠিত করিতে পারিবেন-

(ক) নবম ভাগে বর্ণিত বিষয়াদি এবং অর্থদণ্ড বা অন্য দণ্ডসহ প্রজাতন্ত্রের কর্মে
    নিযুক্ত ব্যক্তিদের কর্মের শর্তাবলী;

(খ) যে কোন রাষ্ট্রায়ত্ত উদ্যোগ বা সংবিধিবদ্ধ সরকারী কর্তৃপক্ষের চালনা ও
    ব্যবস্থাপনা এবং অনুরূপ উদ্যোগ বা সংবিধিবদ্ধ সরকারী কর্তৃপক্ষে কর্মসহ কোন আইনের
    দ্বারা বা অধীন সরকারের উপর ন্যস্ত বা সরকারের দ্বারা পরিচালিত কোন সম্পত্তির
    অর্জন, প্রশাসন, ব্যবস্থাপনা ও বিলি-ব্যবস্থা;

(গ) যে আইনের উপর এই সংবিধানের ১০২ অনুচ্ছেদের ৭২ (৩) দফা প্রযোজ্য হয়, সেইরূপ
    কোন আইন।

(২) কোন ক্ষেত্রে এই অনুচ্ছেদের অধীন কোন প্রশাসনিক ট্রাইব্যুনাল প্রতিষ্ঠিত হইলে
    অনুরূপ ট্রাইব্যুনালের এখতিয়ারের অন্তর্গত কোন বিষয়ে অন্য কোন আদালত কোনরূপ
    কার্যধারা গ্রহণ করিবেন না বা কোন আদেশ প্রদান করিবেন না:

তবে শর্ত থাকে যে, সংসদ আইনের দ্বারা কোন ট্রাইব্যুনালের সিদ্ধান্ত পুনর্বিবেচনা
বা অনুরূপ সিদ্ধান্তের বিরুদ্ধে আপীলের বিধান করিতে পারিবেন।

\subsection{১১৮। নির্বাচন কমিশন প্রতিষ্ঠা}
\label{sec:org5253303}
(১) প্রধান নির্বাচন কমিশনারকে লইয়া এবং রাষ্ট্রপতি সময়ে সময়ে যেরূপ নির্দেশ
    করিবেন, সেইরূপ সংখ্যক অন্যান্য নির্বাচন কমিশনারকে লইয়া বাংলাদেশের একটি
    নির্বাচন কমিশন থাকিবে এবং উক্ত বিষয়ে প্রণীত কোন আইনের বিধানাবলী-সাপেক্ষে
    রাষ্ট্রপতি প্রধান নির্বাচন কমিশনারকে ও অন্যান্য নির্বাচন কমিশনারকে নিয়োগদান
    করিবেন।

(২) একাধিক নির্বাচন কমিশনারকে লইয়া নির্বাচন কমিশন গঠিত হইলে প্রধান
    নির্বাচন কমিশনার তাহার সভাপতিরূপে কার্য করিবেন।

(৩) এই সংবিধানের বিধানাবলী-সাপেক্ষে কোন নির্বাচন কমিশনারের পদের মেয়াদ
    তাঁহার কার্যভার গ্রহণের তারিখ হইতে পাঁচ বৎসরকাল হইবে এবং

(ক) প্রধান নির্বাচন কমিশনার-পদে অধিষ্ঠিত ছিলেন, এমন কোন ব্যক্তি প্রজাতন্ত্রের
    কর্মে নিয়োগলাভের যোগ্য হইবেন না;

(খ) অন্য কোন নির্বাচন কমিশনার অনুরূপ পদে কর্মাবসানের পর প্রধান নির্বাচন
    কমিশনাররূপে নিয়োগলাভের যোগ্য হইবেন, তবে অন্য কোনভাবে প্রজাতন্ত্রের কর্মে
    নিয়োগলাভের যোগ্য হইবেন না।

(৪) নির্বাচন কমিশন দায়িত্ব পালনের ক্ষেত্রে স্বাধীন থাকিবেন এবং কেবল এই
    সংবিধান ও আইনের অধীন হইবেন।

(৫) সংসদ কর্তৃক প্রণীত যে কোন আইনের বিধানাবলী-সাপেক্ষে নির্বাচন কমিশনারদের
    কর্মের শর্তাবলী রাষ্ট্রপতি আদেশের দ্বারা যেরূপ নির্ধারণ করিবেন, সেইরূপ হইবে:

তবে শর্ত থাকে যে, সুপ্রীম কোর্টের বিচারক যেরূপ পদ্ধতি ও কারণে অপসারিত হইতে
পারেন, সেইরূপ পদ্ধতি ও কারণ ব্যতীত কোন নির্বাচন কমিশনার অপসারিত হইবেন না।

(৬) কোন নির্বাচন কমিশনার রাষ্ট্রপতিকে উদ্দেশ করিয়া স্বাক্ষরযুক্ত পত্রযোগে স্বীয়
    পদ ত্যাগ করিতে পারি

\subsection{১১৯। নির্বাচন কমিশনের দায়িত্ব}
\label{sec:org81f7cde}
(১) রাষ্ট্রপতি পদের ও সংসদের নির্বাচনের জন্য ভোটার-তালিকা প্রস্তুতকরণের
    তত্ত্বাবধান, নির্দেশ ও নিয়ন্ত্রণ এবং অনুরূপ নির্বাচন পরিচালনার দায়িত্ব নির্বাচন
    কমিশনের উপর ন্যস্ত থাকিবে এবং নির্বাচন কমিশন এই সংবিধান ও আইনানুযায়ী

(ক) রাষ্ট্রপতি পদের নির্বাচন অনুষ্ঠান করিবেন;

(খ) সংসদ-সদস্যদের নির্বাচন অনুষ্ঠান করিবেন;

(গ) সংসদে নির্বাচনের জন্য নির্বাচনী এলাকার সীমানা নির্ধারণ করিবেন; এবং

(ঘ) রাষ্ট্রপতির পদের এবং সংসদের নির্বাচনের জন্য ভোটার-তালিকা প্রস্তুত করিবেন।

(২) উপরি-উক্ত দফাসমূহে নির্ধারিত দায়িত্বসমূহের অতিরিক্ত যেরূপ দায়িত্ব এই
    সংবিধান বা অন্য কোন আইনের দ্বারা নির্ধারিত হইবে, নির্বাচন কমিশন সেইরূপ
    দায়িত্ব পালন করিবেন।

\subsection{১২০। নির্বাচন কমিশনের কর্মচারীগণ}
\label{sec:orgee59e11}
এই ভাগের অধীন নির্বাচন কমিশনের উপর ন্যস্ত দায়িত্ব পালনের জন্য যেরূপ কর্মচারীর
প্রয়োজন হইবে, নির্বাচন কমিশন অনুরোধ করিলে রাষ্ট্রপতি নির্বাচন কমিশনকে সেইরূপ
কর্মচারী প্রদানের ব্যবস্থা করিবেন।

\subsection{১২১। প্রতি এলাকার জন্য একটিমাত্র ভোটার তালিকা}
\label{sec:org18f58a6}
সংসদের নির্বাচনের জন্য প্রত্যেক আঞ্চলিক নির্বাচনী এলাকার একটি করিয়া
ভোটার-তালিকা থাকিবে এবং ধর্ম, জাত, বর্ণ ও নারী-পুরুষভেদের ভিত্তিতে ভোটারদের
বিন্যস্ত করিয়া কোন বিশেষ ভোটার-তালিকা প্রণয়ন করা যাইবে না।

\subsection{১২২। ভোটার-তালিকায় নামভুক্তির যোগ্যতা}
\label{sec:orgdb46c71}
(১) প্রাপ্ত বয়স্কের ভোটাধিকার-ভিত্তিতে ৭৫ * * * সংসদের নির্বাচন অনুষ্ঠিত
    হইবে।

(২) কোন ব্যক্তি সংসদের নির্বাচনের জন্য নির্ধারিত কোন নির্বাচনী এলাকায়
    ভোটার-তালিকাভু্ক্ত হইবার অধিকারী হইবেন, যদি

(ক) তিনি বাংলাদেশের নাগরিক হন;

(খ) তাঁহার বয়স আঠার বৎসরের কম না হয়;

(গ) কোন যোগ্য আদালত কর্তৃক তাঁহার সম্পর্কে অপ্রকৃতিস্থ বলিয়া ঘোষণা বহাল না
    থাকিয়া থাকে; ৭৬ এবং

(ঘ) তিনি ঐ নির্বাচনী এলাকার অধিবাসী বা আইনের দ্বারা ঐ নির্বাচনী এলাকার
    অধিবাসী বিবেচিত হন ৭৭ ।

\subsection{১২৩। নির্বাচন-অনুষ্ঠানের সময়}
\label{sec:org1210914}
(১) রাষ্ট্রপতি-পদের মেয়াদ অবসানের কারণে উক্ত পদ শূন্য হইলে মেয়াদ-সমাপ্তির
    তারিখের পূর্ববর্তী নব্বই হইতে ষাট দিনের মধ্যে শূন্য পদ পূরণের জন্য নির্বাচন
    অনুষ্ঠিত হইবে:

তবে শর্ত থাকে যে, যে সংসদের দ্বারা তিনি নির্বাচিত হইয়াছেন সেই সংসদের
মেয়াদকালে রাষ্ট্রপতির কার্যকাল শেষ হইলে সংসদের পরবর্তী সাধারণ নির্বাচন না
হওয়া পর্যন্ত অনুরূপ শন্য পদ পূর্ণ করিবার জন্য নির্বাচন অনুষ্ঠিত হইবে না, এবং অনুরূপ
সাধারণ নির্বাচনের পর সংসদের প্রথম বৈঠকের দিন হইতে ত্রিশ দিনের মধ্যে
রাষ্ট্রপতির শূন্য পদ পূর্ণ করিবার জন্য নির্বাচন অনুষ্ঠিত হইবে।

(২) মৃত্যু, পদত্যাগ বা অপসারণের ফলে রাষ্ট্রপতির পদ শূন্য হইলে পদটি শূন্য হইবার
    পর নব্বই দিনের মধ্যে, তাহা পূর্ণ করিবার জন্য নির্বাচন অনুষ্ঠিত হইবে।

৮১ (৩) মেয়াদ অবসানের কারণে অথবা মেয়াদ অবসান ব্যতীত অন্য কোন কারণে সংসদ
ভাংগিয়া যাইবার পরবর্তী নব্বই দিনের মধ্যে সংসদ-সদস্যদের সাধারণ নির্বাচন
অনুষ্ঠিত হইবে।

(৪) সংসদ ভাঙ্গিয়া যাওয়া ব্যতীত অন্য কোন কারণে সংসদের কোন সদস্যপদ শূন্য হইলে
    পদটি শূন্য হইবার নব্বই দিনের মধ্যে উক্ত শূন্যপদ পূর্ণ করিবার জন্য নির্বাচন অনুষ্ঠিত
    হইবে ৮২ :

তবে শর্ত থাকে যে, যদি প্রধান নির্বাচন কমিশনারের মতে, কোন দৈব-দূর্বিপাকের
কারণে এই দফার নির্ধারিত মেয়াদের মধ্যে উক্ত নির্বাচন অনুষ্ঠান সম্ভব না হয়,
তাহা হইলে উক্ত মেয়াদের শেষ দিনের পরবর্তী নব্বই দিনের মধ্যে উক্ত নির্বাচন
অনুষ্ঠিত হইবে।

\subsection{১২৪। নির্বাচন সম্পর্কে সংসদের বিধান প্রণয়নের ক্ষমতা}
\label{sec:org4cad168}
এই সংবিধানের বিধানাবলী সাপেক্ষে সংসদ আইনের দ্বারা নির্বাচনী এলাকার সীমা
নির্ধারণ, ভোটার-তালিকা প্রস্তুতকরণ, নির্বাচন অনুষ্ঠান এবং সংসদের যথাযথ গঠনের
জন্য প্রয়োজনীয় অন্যান্য বিষয়সহ সংসদের নির্বাচন সংক্রান্ত বা নির্বাচনের সহিত
সম্পর্কিত সকল বিষয়ে বিধান প্রণয়ন করিতে পারিবেন।

\subsection{১২৫। নির্বাচনী আইন ও নির্বাচনের বৈধতা}
\label{sec:org732c313}
এই সংবিধানে যাহা বলা হইয়াছে, তাহা সত্ত্বেও

(ক) এই সংবিধানের ১২৪ অনুচ্ছেদের অধীন প্রণীত বা প্রণীত বলিয়া বিবেচিত
    নির্বাচনী এলাকার সীমা নির্ধারণ, কিংবা অনুরূপ নির্বাচনী এলাকার জন্য আসন-বন্টন
    সম্পর্কিত যে কোন আইনের বৈধতা সম্পর্কে আদালতে প্রশ্ন উত্থাপন করা যাইবে না;

(খ) সংসদ কর্তৃক প্রণীত কোন আইনের দ্বারা বা অধীন বিধান-অনুযায়ী কর্তৃপক্ষের নিকট
    এবং অনুরূপভাবে নির্ধারিত প্রণালীতে নির্বাচনী দরখাস্ত ব্যতীত ৮৪ রাষ্ট্রপতি
    ৮৫ * * * পদে নির্বাচন বা সংসদের কোন নির্বাচন সম্পর্কে কোন প্রশ্ন উত্থাপন করা
    যাইবে না।

\subsection{১২৬। নির্বাচন কমিশনকে নির্বাহী কর্তৃপক্ষের সহায়তাদান}
\label{sec:org8309c91}
নির্বাচন কমিশনের দায়িত্বপালনে সহায়তা করা সকল নির্বাহী কর্তৃপক্ষের কর্তব্য
হইবে।

\subsection{১২৭। মহা হিসাব-নিরীক্ষক পদের প্রতিষ্ঠা}
\label{sec:org615fed7}
(১) বাংলাদেশের একজন মহা হিসাব-নিরীক্ষক ও নিয়ন্ত্রক (অতঃপর “মহা
    হিসাব-নিরীক্ষক” নামে অভিহিত) থাকিবেন এবং তাঁহাকে রাষ্ট্রপতি নিয়োগদান
    করিবেন।

(২) এই সংবিধান ও সংসদ কর্তৃক প্রণীত যে কোন আইনের বিধানাবলী সাপেক্ষে মহা
    হিসাব-নিরীক্ষকের কর্মের শর্তাবলী রাষ্ট্রপতি আদেশের দ্বারা যেরূপ নির্ধারণ
    করিবেন, সেইরূপ হইবে।

\subsection{১২৮। মহা-হিসাব নিরীক্ষকের দায়িত্ব}
\label{sec:orgd097fe5}
(১) মহা হিসাব-নিরীক্ষক প্রজাতন্ত্রের সরকারী হিসাব এবং সকল আদালত, সরকারী
    কর্তৃপক্ষ ও কর্মচারীর সরকারী হিসাব নিরীক্ষা করিবেন ও অনুরূপ হিসাব সম্পর্কে
    রিপোর্টদান করিবেন এবং সেই উদ্দেশ্যে তিনি কিংবা সেই প্রয়োজনে তাঁহার দ্বারা
    ক্ষমতাপ্রাপ্ত কোন ব্যক্তি প্রজাতন্ত্রের কর্মে নিযুক্ত যে কোন ব্যক্তির দখলভুক্ত সকল
    নথি, বহি, রসিদ, দলিল, নগদ অর্থ, ষ্ট্যাম্প, জামিন, ভাণ্ডার বা অন্য প্রকার
    সরকারী সম্পত্তি পরীক্ষার অধিকারী হইবেন।

(২) এই অনুচ্ছেদের (১) দফায় বর্ণিত বিধানাবলীর হানি না করিয়া বিধান করা
    হইতেছে যে, আইনের দ্বারা প্রত্যক্ষভাবে প্রতিষ্ঠিত কোন যৌথ সংস্থার ক্ষেত্রে
    আইনের দ্বারা যেরূপ ব্যক্তি কর্তৃক উক্ত সংস্থার হিসাব নিরীক্ষার ও অনুরূপ হিসাব
    সম্পর্কে রিপোর্ট দানের ব্যবস্থা করা হইয়া থাকে, সেইরূপ ব্যক্তি কর্তৃক অনুরূপ হিসাব
    নিরীক্ষা ও অনুরূপ হিসাব সম্পর্কে রিপোর্ট দান করা যাইবে।

(৩) এই অনুচ্ছেদের (১) দফায় নির্ধারিত দায়িত্বসমূহ ব্যতীত সংসদ আইনের দ্বারা
    যেরূপ নির্ধারণ করিবেন, মহা হিসাব-নিরীক্ষককে সেইরূপ দায়িত্বভার অর্পণ করিতে
    পারিবেন এবং এই দফার অধীন বিধানাবলী প্রণীত না হওয়া পর্যন্ত রাষ্ট্রপতি আদেশের
    দ্বারা অনুরূপ বিধানাবলী প্রণয়ন করিতে পারিবেন।

(৪) এই অনুচ্ছেদের (১) দফার অধীন দায়িত্বপালনের ক্ষেত্রে মহা হিসাব-নিরীক্ষককে
    অন্য কোন ব্যক্তি বা কর্তৃপক্ষের পরিচালনা বা নিয়ন্ত্রণের অধীন করা হইবে না।

\subsection{১২৯। মহা হিসাব-নিরীক্ষকের কর্মের মেয়াদ}
\label{sec:orgf3ea98c}
(১) এই অনুচ্ছেদের বিধানাবলী-সাপেক্ষে মহা হিসাব-নিরীক্ষক তাঁহার দায়িত্ব
    গ্রহণের তারিখ হইতে পাঁচ বৎসর বা তাঁহার পঁয়ষট্টি বৎসর বয়স পূর্ণ হওয়া ইহার মধ্যে
    যাহা অগ্রে ঘটে, সেই কাল পর্যন্ত স্বীয় পদে বহাল থাকিবেন।

(২) সুপ্রীম কোর্টের কোন বিচারক যেরূপ পদ্ধতি ও কারণে অপসারিত হইতে পারেন,
    সেইরূপ পদ্ধতি ও কারণ ব্যতীত মহা হিসাব-নিরীক্ষক অপসারিত হইবেন না।

(৩) মহা হিসাব-নিরীক্ষক রাষ্ট্রপতিকে উদ্দেশ করিয়া স্বাক্ষরযুক্ত পত্রযোগে স্বীয় পদ
    ত্যাগ করিতে পারিবেন।

(৪) কর্মাবসানের পর মহা হিসাব-নিরীক্ষক প্রজাতন্ত্রের কর্মে অন্য কোন পদে নিযুক্ত
    হইবার যোগ্য হইবেন না।

\subsection{১৩০। অস্থায়ী মহা হিসাব-নিরীক্ষক}
\label{sec:orgebad711}
কোন সময়ে মহা হিসাব-নিরীক্ষকের পদ শূন্য থাকিলে কিংবা অনুপস্থিতি, অসুস্থতা বা
অন্য কোন কারণে তিনি কার্যভার পালনে অক্ষম বলিয়া রাষ্ট্রপতির নিকট
সন্তোষজনকভাবে প্রতীয়মান হইলে ক্ষেত্রমত এই সংবিধানের ১২৭ অনুচ্ছেদের অধীন কোন
নিয়োগদান না করা পর্যন্ত কিংবা মহা হিসাব-নিরীক্ষক পুনরায় স্বীয় দায়িত্ব গ্রহণ
না করা পর্যন্ত রাষ্ট্রপতি কোন ব্যক্তিকে মহা হিসাব-নিরীক্ষকরূপে কার্য করিবার জন্য
এবং উক্ত পদের দায়িত্বভার পালনের জন্য নিয়োগদান করিতে পারিবেন।
\subsection{১৩১। প্রজাতন্ত্রের হিসাব-রক্ষার আকার ও পদ্ধতি}
\label{sec:org9dc37a5}
রাষ্ট্রপতির অনুমোদনক্রমে মহা হিসাব-নিরীক্ষক যেরূপ নির্ধারণ করিবেন, সেইরূপ আকার
ও পদ্ধতিতে প্রজাতন্ত্রের হিসাব রক্ষিত হইবে।

\subsection{১৩২। সংসদে মহা হিসাব-নিরীক্ষকের রিপোর্ট উপস্থাপন}
\label{sec:org219e2fd}
প্রজাতন্ত্রের হিসাব সম্পর্কিত মহা হিসাব-নিরীক্ষকের রিপোর্টসমূহ রাষ্ট্রপতির নিকট
পেশ করা হইবে এবং রাষ্ট্রপতি তাহা সংসদে পেশ করিবার ব্যবস্থা করিবেন।

\subsection{১৩৩। নিয়োগ ও কর্মের শর্তাবলী}
\label{sec:org91cd770}
এই সংবিধানের বিধানাবলী-সাপেক্ষে সংসদ আইনের দ্বারা প্রজাতন্ত্রের কর্মে
কর্মচারীদের নিয়োগ ও কর্মের শর্তাবলী নিয়ন্ত্রণ করিতে পারিবেন:

তবে শর্ত থাকে যে, এই উদ্দেশ্যে আইনের দ্বারা বা অধীন বিধান প্রণীত না হওয়া
পর্যন্ত অনুরূপ কর্মচারীদের নিয়োগ ও কর্মের শর্তাবলী নিয়ন্ত্রণ করিয়া
বিধিসমূহ-প্রণয়নের ক্ষমতা রাষ্ট্রপতির থাকিবে এবং অনুরূপ যে কোন আইনের
বিধানাবলী-সাপেক্ষে অনুরূপ বিধিসমূহ কার্যকর হইবে।

\subsection{১৩৪। কর্মের মেয়াদ}
\label{sec:org8bb223a}
এই সংবিধানের দ্বারা অন্যরূপ বিধান না করা হইয়া থাকিলে প্রজাতন্ত্রের কর্মে
নিযুক্ত প্রত্যেক ব্যক্তি রাষ্ট্রপতির সন্তোষানুযায়ী সময়সীমা পর্যন্ত স্বীয় পদে বহাল
থাকিবেন।

\subsection{১৩৫। অসামরিক সরকারী কর্মচারীদের বরখাস্ত প্রভৃতি}
\label{sec:orge07444c}
(১) প্রজাতন্ত্রের কর্মে অসামরিক পদে নিযুক্ত কোন ব্যক্তি তাঁহার নিয়োগকারী
    কর্তৃপক্ষ-অপেক্ষা অধঃস্তন কোন কর্তৃপক্ষের দ্বারা বরখাস্ত বা অপসারিত বা পদাবনমিত
    হইবেন না।

(২) অনুরূপ পদে নিযুক্ত কোন ব্যক্তিকে তাঁহার সম্পর্কে প্রস্তাবিত ব্যবস্থা গ্রহণের
    বিরুদ্ধে কারণ দর্শাইবার যুক্তিসঙ্গত সুযোগদান না করা পর্যন্ত তাঁহাকে বরখাস্ত বা
    অপসারিত বা পদাবনমিত করা যাইবে না:

তবে শর্ত থাকে যে, এই দফা সেই সকল ক্ষেত্রে প্রযোজ্য হইবে না, যেখানে

(অ) কোন ব্যক্তি যে আচরণের ফলে ফৌজদারী অপরাধে দণ্ডিত হইয়াছেন, সেই আচরণের
    জন্য তাঁহাকে বরখাস্ত, অপসারিত বা পদাবনমিত করা হইয়াছে; অথবা

(আ) কোন ব্যক্তিকে বরখাস্ত, অপসারিত বা পদাবনমিত করিবার ক্ষমতাসম্পন্ন কর্তৃপক্ষের
    নিকট সন্তোষজনকভাবে প্রতীয়মান হয় যে, কোন কারণে- যাহা উক্ত কর্তৃপক্ষ লিপিবদ্ধ
    করিবেন- উক্ত ব্যক্তিকে কারণ দর্শাইবার সুযোগদান করা যুক্তিসঙ্গতভাবে সম্ভব নহে;
    অথবা

(ই) রাষ্ট্রপতির নিকট সন্তোষজনকভাবে প্রতীয়মান হয় যে, রাষ্ট্রের নিরাপত্তার
    স্বার্থে উক্ত ব্যক্তিকে অনুরূপ সুযোগদান সমীচীন নহে।

(৩) অনুরূপ কোন ব্যক্তিকে এই অনুচ্ছেদের (২) দফায় বর্ণিত কারণ দর্শাইবার সুযোগদান
    করা যুক্তিসঙ্গতভাবে সম্ভব কি না, এইরূপ প্রশ্ন উত্থাপিত হইলে সেই সম্পর্কে তাঁহাকে
    বরখাস্ত, অপসারিত বা পদাবনমিত করিবার ক্ষমতাসম্পন্ন কর্তৃপক্ষের সিদ্ধান্ত চূড়ান্ত
    হইবে।

(৪) যে ক্ষেত্রে কোন ব্যক্তি কোন লিখিত চুক্তির অধীন প্রজাতন্ত্রের কর্মে নিযুক্ত
    হইয়াছেন এবং উক্ত চুক্তির শর্তাবলী-অনুযায়ী যথাযথ নোটিশের দ্বারা চুক্তিটির অবসান
    ঘটান হইয়াছে, সেই ক্ষেত্রে চুক্তিটির অনুরূপ অবসানের জন্য তিনি এই অনুচ্ছেদের
    উদ্দেশ্যসাধনকল্পে পদ হইতে অপসারিত হইয়াছেন বলিয়া গণ্য হইবে না।

\subsection{১৩৬। কর্মবিভাগ-পুনর্গঠন}
\label{sec:org8ce6071}
আইনের দ্বারা প্রজাতন্ত্রের কর্মবিভাগসমূহের সৃষ্টি, সংযুক্তকরণ ও একত্রীকরণসহ
পুনর্গঠনের বিধান করা যাইবে এবং অনুরূপ আইন প্রজাতন্ত্রের কর্মে নিযুক্ত কোন ব্যক্তির
কর্মের শর্তাবলীর তারতম্য করিতে ও তাহা রদ করিতে পারিবে।

\subsection{১৩৭। কমিশন-প্রতিষ্ঠা}
\label{sec:org63d3256}
আইনের দ্বারা বাংলাদেশের জন্য এক বা একাধিক সরকারী কর্ম কমিশন প্রতিষ্ঠার
বিধান করা যাইবে এবং একজন সভাপতিকে ও আইনের দ্বারা যেরূপ নির্ধারিত হইবে,
সেইরূপ অন্যান্য সদস্যকে লইয়া প্রত্যেক কমিশন গঠিত হইবে।

\subsection{১৩৮। সদস্য-নিয়োগ}
\label{sec:org170a6c6}
(১) প্রত্যেক সরকারী কর্ম কমিশনের সভাপতি ও অন্যান্য সদস্য রাষ্ট্রপতি কর্তৃক নিযুক্ত
    হইবেন:

তবে শর্ত থাকে যে, প্রত্যেক কমিশনের যতদূর সম্ভব অর্ধেক (তবে অর্ধেকের কম নহে)
সংখ্যক সদস্য এমন ব্যক্তিগণ হইবেন, যাঁহারা কুড়ি বৎসর বা ততোধিককাল বাংলাদেশের
রাষ্ট্রীয় সীমানার মধ্যে যে কোন সময়ে কার্যরত কোন সরকারের কর্মে কোন পদে
অধিষ্ঠিত ছিলেন।

(২) সংসদ কর্তৃক প্রণীত যে কোন আইন-সাপেক্ষে কোন সরকারী কর্ম কমিশনের সভাপতি ও
    অন্যান্য সদস্যের কর্মের শর্তাবলী রাষ্ট্রপতি আদেশের দ্বারা যেরূপ নির্ধারণ করিবেন,
    সেইরূপ হইবে।

\subsection{১৩৯। পদের মেয়াদ}
\label{sec:orgbd8a719}
(১) এই অনুচ্ছেদের বিধানাবলী-সাপেক্ষে কোন সরকারী কর্ম কমিশনের সভাপতি বা অন্য
    কোন সদস্য তাঁহার দায়িত্ব গ্রহণের তারিখ হইতে পাঁচ বৎসর বা তাঁহার ৮৭ পয়ষট্টি
    বৎসর বয়স পূর্ণ হওয়া ইহার মধ্যে যাহা অগ্রে ঘটে, সেই কাল পর্যন্ত স্বীয় পদে বহাল
    থাকিবেন।

(২) সুপ্রীম কোর্টের কোন বিচারক যেরূপ পদ্ধতি ও কারণে অপসারিত হইতে পারেন,
    সেইরূপ পদ্ধতি ও কারণ ব্যতীত কোন সরকারী কর্ম কমিশনের সভাপতি বা অন্য কোন সদস্য
    অপসারিত হইবেন না।

(৩) কোন সরকারী কর্ম কমিশনের সভাপতি বা অন্য কোন সদস্য রাষ্ট্রপতিকে উদ্দেশ
    করিয়া স্বাক্ষরযুক্ত পত্রযোগে স্বীয় পদ ত্যাগ করিতে পারিবেন।

(৪) কর্মাবসানের পর কোন সরকারী কর্ম কমিশনের সদস্য প্রজাতন্ত্রের কর্মে পুনরায়
    নিযুক্ত হইবার যোগ্য থাকিবেন না, তবে এই অনুচ্ছেদের (১) দফা-সাপেক্ষে

(ক) কর্মাবসানের পর কোন সভাপতি এক মেয়াদের জন্য পুনর্নিয়োগলাভের যোগ্য
    থাকিবেন; এবং

(খ) কর্মাবসানের পর কোন সদস্য (সভাপতি ব্যতীত) এক মেয়াদের জন্য কিংবা কোন
    সরকারী কর্ম কমিশনের সভাপতিরূপে নিয়োগলাভের যোগ্য থাকিবেন।

\subsection{১৪০। কমিশনের দায়িত্ব}
\label{sec:orgc5ef44a}
(১) কোন সরকারী কর্ম কমিশনের দায়িত্ব হইবে

(ক) প্রজাতন্ত্রের  কর্মে নিয়োগদানের জন্য উপযুক্ত  ব্যক্তিদিগকে মনোনয়নের উদ্দেশ্যে
    যাচাই ও পরীক্ষা-পরিচালনা;

(খ) এই অনুচ্ছেদের  (২) দফা অনুযায়ী রাষ্ট্রপতি কর্তৃক কোন  বিষয় সম্পর্কে কমিশনের
    পরামর্শ চাওয়া হইলে  কিংবা কমিশনের দায়িত্ব-সংক্রান্ত কোন  বিষয় কমিশনের নিকট
    প্রেরণ করা হইলে সেই সম্বন্ধে রাষ্ট্রপতিকে উপদেশ দান; এবং

(গ) আইনের দ্বারা নির্ধারিত অন্যান্য দায়িত্ব পালন।

(২) সংসদ কর্তৃক  প্রণীত কোন আইন এবং কোন কমিশনের  সহিত পরামর্শক্রমে রাষ্ট্রপতি
    কর্তৃক  প্রণীত   কোন  প্রবিধানের  (যাহা   অনুরূপ  আইনের  সহিত   অসমঞ্জস  নহে)
    বিধানাবলী-সাপেক্ষে রাষ্ট্রপতি নিম্নলিখিত ক্ষেত্রসমূহে কোন কমিশনের সহিত পরামর্শ
    করিবেন:

(ক)  প্রজাতন্ত্রের  কর্মের  জন্য  যোগ্যতা ও  তাহাতে  নিয়োগের  পদ্ধতি  সম্পর্কিত
    বিষয়াদি;

(খ)  প্রজাতন্ত্রের  কর্মে নিয়োগদান,  উক্ত  কর্মের  এক  শাখা হইতে  অন্য  শাখায়
    পদোন্নতিদান ও  বদলিকরণ এবং অনুরূপ  নিয়োগদান, পদোন্নতিদান বা  বদলিকরণের জন্য
    প্রার্থীর উপযোগিতা-নির্ণয় সম্পর্কে অনুসরণীয় নীতিসমূহ;

(গ) অবসর ভাতার  অধিকারসহ প্রজাতন্ত্রের কর্মের শর্তাবলীকে  প্রভাবিত করে, এইরূপ
    বিষয়াদি; এবং

(ঘ) প্রজাতন্ত্রের কর্মের শৃঙ্খলামূলক বিষয়াদি।

\subsection{১৪১। বার্ষিক রিপোর্ট}
\label{sec:orgd4ec1ae}
(১) প্রত্যেক কমিশন প্রতি বৎসর মার্চ মাসের প্রথম দিবসে বা তাহার পূর্বে পূর্ববর্তী
    একত্রিশে ডিসেম্বরে সমাপ্ত এক বৎসরে স্বীয় কার্যাবলী সম্বন্ধে রিপোর্ট প্রস্তুত
    করিবেন এবং তাহা রাষ্ট্রপতির নিকট পেশ করিবেন।

(২) রিপোর্টের সহিত একটি স্মারকলিপি থাকিবে, যাহাতে

(ক) কোন ক্ষেত্রে কমিশনের কোন পরামর্শ গৃহীত না হইয়া থাকিলে সেই ক্ষেত্র এবং
    পরামর্শ গৃহীত না হইবার কারণ; এবং

(খ) যে সকল ক্ষেত্রে কমিশনের সহিত পরামর্শ করা উচিত ছিল অথচ পরামর্শ করা হয়
    নাই, সেই সকল ক্ষেত্র এবং পরামর্শ না করিবার কারণ;

সম্বন্ধে কমিশন যতদূর অবগত, ততদূর লিপিবদ্ধ করিবেন।

(৩) যে বৎসর রিপোর্ট পেশ করা হইয়াছে, সেই বৎসর একত্রিশে মার্চের পর অনুষ্ঠিত
    সংসদের প্রথম বৈঠকে রাষ্ট্রপতি উক্ত রিপোর্ট ও স্মারকলিপি সংসদে উপস্থাপনের
    ব্যবস্থা করিবেন।

\subsection{১৪১ক। জরুরী-অবস্থা ঘোষণা}
\label{sec:orgc620780}
(১) রাষ্ট্রপতির নিকট যদি সন্তোষজনকভাবে প্রতীয়মান হয় যে, এমন জরুরী-অবস্থা
    বিদ্যমান রহিয়াছে, যাহাতে যুদ্ধ বা বহিরাক্রমণ বা অভ্যন্তরীণ গোলযোগের দ্বারা
    বাংলাদেশ বা উহার যে কোন অংশের নিরাপত্তা বা অর্থনৈতিক জীবন বিপদের
    সম্মুখীন, তাহা হইলে তিনি জরুরী-অবস্থা ঘোষণা করিতে পারিবে ৮৮ :

তবে শর্ত থাকে যে, অনুরূপ ঘোষণার বৈধতার জন্য ঘোষণার পূর্বেই প্রধানমন্ত্রীর
প্রতি-স্বাক্ষর প্রয়োজন হইবে।

৮৯ * * *

(২) জরুরী-অবস্থার ঘোষণা

(ক) পরবর্তী কোন ঘোষণার দ্বারা প্রত্যাহার করা যাইবে;

(খ) সংসদে উপস্থাপিত হইবে;

(গ) একশত কুড়ি দিন অতিবাহিত হইবার পূর্বে সংসদের প্রস্তাব-দ্বারা অনুমোদিত না
    হইলে উক্ত সময়ের অবসানে কার্যকর থাকিবে না:

তবে শর্ত থাকে যে, যদি সংসদ ভাঙ্গিয়া যাওয়া অবস্থায় অনুরূপ কোন ঘোষণা জারী করা
হয় কিংবা এই দফার (গ) উপ-দফায় বর্ণিত এক শত কুড়ি দিনের মধ্যে সংসদ ভাঙ্গিয়া
যায়, তাহা হইলে তাহা পুনর্গঠিত হইবার পর সংসদের প্রথম বৈঠকের তারিখ হইতে
ত্রিশ দিন অতিবাহিত হইবার পূর্বে ঘোষণাটি অনুমোদন করিয়া সংসদে প্রস্তাব গৃহীত না
হওয়া পর্যন্ত উক্ত ত্রিশ দিনের অবসানে অনুরূপ ঘোষণা কার্যকর থাকিবে না।

(৩) যুদ্ধ বা বহিরাক্রমণ বা অভ্যন্তরীণ গোলযোগের বিপদ আসন্ন বলিয়া রাষ্ট্রপতির
    নিকট সন্তোষজনকভাবে প্রতীয়মান হইলে প্রকৃত যুদ্ধ বা বহিরাক্রমণ বা অভ্যন্তরীণ
    গোলযোগ সংঘটিত হইবার পূর্বে তিনি অনুরূপ যুদ্ধ বা বহিরাক্রমণ বা অভ্যন্তরীণ
    গোলযোগের জন্য বাংলাদেশ বা উহার যে কোন অংশের নিরাপত্তা বিপন্ন বলিয়া
    জরুরী-অবস্থা ঘোষণা করিতে পারিবেন।

\subsection{১৪১খ। জরুরী-অবস্থার সময় সংবিধানের কতিপয় অনুচ্ছেদের বিধান স্থগিতকরণ}
\label{sec:orgf30df53}
এই সংবিধানের তৃতীয় ভাগের অন্তর্গত বিধানাবলীর কারণে রাষ্ট্র যে আইন প্রণয়ন
করিতে ও নির্বাহী ব্যবস্থা গ্রহণ করিতে সক্ষম নহেন, জরুরী-অবস্থা ঘোষণার
কার্যকরতা-কালে এই সংবিধানের ৩৬, ৩৭, ৩৮, ৩৯, ৪০ ও ৪২ অনুচ্ছেদসমূহের কোন
কিছুই সেইরূপ আইন-প্রণয়ন ও নির্বাহী ব্যবস্থা গ্রহণ সম্পর্কিত রাষ্ট্রের ক্ষমতাকে
সীমাবদ্ধ করিবে না; তবে অনুরূপভাবে প্রণীত কোন আইনের কর্তৃত্বে যাহা করা হইয়াছে
বা করা হয় নাই, তাহা ব্যতীত অনুরূপ আইন যে পরিমাণে কর্তৃত্বহীন, জরুরী-অবস্থার
ঘোষণা অকার্যকর হইবার অব্যবহিত পরে তাহা সেই পরিমাণে অকার্যকর হইবে।

\subsection{১৪১গ। জরুরী-অবস্থার সময় মৌলিক অধিকারসমূহ স্থগিতকরণ}
\label{sec:org5bc44a2}
(১) জরুরী-অবস্থা ঘোষণার ৯০ কার্যকরতা-কালে প্রধানমন্ত্রীর লিখিত পরামর্শ অনুযায়ী
    রাষ্ট্রপতি আদেশের দ্বারা ঘোষণা করিতে পারিবেন যে, আদেশে উল্লেখিত এবং
    সংবিধানের তৃতীয় ভাগের অন্তর্গত মৌলিক অধিকারসমূহ বলবৎকরণের জন্য আদালতে
    মামলা রুজু করিবার অধিকার এবং আদেশে অনুরূপভাবে উল্লেখিত কোন অধিকার বলবৎকরণের
    জন্য কোন আদালতে বিবেচনাধীন সকল মামলা জরুরী-অবস্থা ঘোষণার কার্যকরতা-কালে
    কিংবা উক্ত আদেশের দ্বারা নির্ধারিত স্বল্পতর কালের জন্য স্থগিত থাকিবে।

(২) সমগ্র বাংলাদেশ বা উহার যে কোন অংশে এই অনুচ্ছেদের অধীন প্রণীত আদেশ
    প্রযোজ্য হইতে পারিবে।

(৩) এই অনুচ্ছেদের অধীন প্রণীত প্রত্যেক আদেশ যথাসম্ভব শীঘ্র সংসদে উপস্থাপিত
    হইবে।

\subsection{১৪২। সংবিধানের বিধান সংশোধনের ক্ষমতা}
\label{sec:org0f12b86}

৯২ (১) এই সংবিধানে যাহা বলা হইয়াছে, তাহা সত্ত্বেও

(ক) সংসদের আইন-দ্বারা এই সংবিধানের কোন বিধান ৯৩ সংযোজন, পরিবর্তন,
    প্রতিস্থাপন বা রহিতকরণের দ্বারা সংশোধিত হইতে পারিবে:
    তবে শর্ত থাকে যে,
    (অ) অনুরূপ ৯৪ সংশোধনীর জন্য আনীত কোন বিলের সম্পূর্ণ শিরনামায় এই সংবিধানের
    কোন বিধান সংশোধন ৯৫ * * * করা হইবে বলিয়া স্পষ্টরূপে উল্লেখ না থাকিলে বিলটি
    বিবেচনার জন্য গ্রহণ করা যাইবে না;

(আ) সংসদের মোট সদস্য-সংখ্যার অন্যূন দুই-তৃতীয়াংশ ভোটে গৃহীত না হইলে অনুরূপ কোন
বিলে সম্মতিদানের জন্য তাহা রাষ্ট্রপতির নিকট উপস্থাপিত হইবে না;

(খ) উপরি-উক্ত উপায়ে কোন বিল গৃহীত হইবার পর সম্মতির জন্য রাষ্ট্রপতির নিকট
    তাহা উপস্থাপিত হইলে উপস্থাপনের সাত দিনের মধ্যে তিনি বিলটিতে সম্মতিদান
    করিবেন, এবং তিনি তাহা করিতে অসমর্থ হইলে উক্ত মেয়াদের অবসানে তিনি বিলটিতে
    সম্মতিদান করিয়াছেন বলিয়া গণ্য হইবে।

৯৬ (১ক) (১) দফায় যাহা বলা হইয়াছে, তাহা সত্ত্বেও এই সংবিধানের প্রস্তাবনার
অথবা ৮, ৪৮ ৯৭ বা ৫৬ ৯৮ * * * অনুচ্ছেদ অথবা এই অনুচ্ছেদের কোন বিধানাবলীর
সংশোধনের ব্যবস্থা রহিয়াছে এইরূপ কোন বিল উপরি-উক্ত উপায়ে গৃহীত হইবার পর
সম্মতির জন্য রাষ্ট্রপতির নিকট উপস্থাপিত হইলে উপস্থাপনের সাত দিনের মধ্যে তিনি
বিলটিতে সম্মতিদান করিবেন কি করিবেন না এই প্রশ্নটি গণ-ভোটে প্রেরণের ব্যবস্থা
করিবেন।

(১খ) এই অনুচ্ছেদের অধীন গণ-ভোট ৯৯ সংসদ নির্বাচনের জন্য প্রস্তুতকৃত ভোটার
তালিকাভু্ক্ত ব্যক্তিগণের মধ্যে নির্বাচন কমিশন কর্তৃক আইনের দ্বারা নির্ধারিত
মেয়াদের মধ্যে ও পদ্ধতিতে পরিচালিত হইবে।

(১গ) এই অনুচ্ছেদের অধীন কোন বিল সম্পর্কে পরিচালিত গণ-ভোটের ফলাফল যেদিন
ঘোষিত হয় সেইদিন-

(অ) প্রদত্ত সমুদয় ভোটের সংখ্যাগরিষ্ঠ ভোট উক্ত বিলে সম্মতিদানের পক্ষে প্রদান করা
হইয়া থাকিলে, রাষ্ট্রপতি বিলটিতে সম্মতিদান করিয়াছেন বলিয়া গণ্য হইবে, অথবা

(আ) প্রদত্ত সমুদয় ভোটের সংখ্যাগরিষ্ঠ ভোট উক্ত বিলে সম্মতিদানের পক্ষে প্রদান করা
না হইয়া থাকিলে, রাষ্ট্রপতি বিলটিতে সম্মতিদানে বিরত রহিয়াছেন বলিয়া গণ্য
হইবে।

১০০ (১ঘ) (১গ) দফার কোন কিছুই মন্ত্রিসভা বা সংসদের উপর আস্থা বা অনাস্থা
বলিয়া গণ্য হইবে না।

(২) এই অনুচ্ছেদের অধীন প্রণীত কোন সংশোধনের ক্ষেত্রে ২৬ অনুচ্ছেদের কোন কিছুই
    প্রযোজ্য হইবে না।

\subsection{১৪৩। প্রজাতন্ত্রের সম্পত্তি}
\label{sec:org1d5998e}
(১) আইনসঙ্গতভাবে প্রজাতন্ত্রের উপর ন্যস্ত যে কোন ভূমি বা সস্পত্তি ব্যতীত
   নিম্নলিখিত প্রজাতন্ত্রের উপর ন্যস্ত হইবে:

(ক) বাংলাদেশের যে কোন ভূমির অন্তঃস্থ সকল খনিজ ও অন্যান্য মূল্যসম্পন্ন সামগ্রী;
(খ) বাংলাদেশের রাষ্ট্রীয় জলসীমার অন্তর্বর্তী মহাসাগরের অন্তঃস্থ কিংবা
   বাংলাদেশের মহীসোপানের উপরিস্থ মহাসাগরের অন্তঃস্থ সকল ভূমি, খনিজ ও অন্যান্য
   মূল্যসম্পন্ন সামগ্রী; এবং
(গ) বাংলাদেশে অবস্থিত প্রকৃত মালিকবিহীন যে কোন সম্পত্তি।

(২) সংসদ সময়ে সময়ে আইনের দ্বারা বাংলাদেশের রাষ্ট্রীয় সীমানার এবং
বাংলাদেশের রাষ্ট্রীয় জলসীমা ও মহীসোপানের সীমা-নির্ধারণের বিধান করিতে
পারিবেন।

\subsection{১৪৪। সম্পত্তি ও কারবার প্রভৃতি-প্রসঙ্গে নির্বাহী কর্তৃত্ব}
\label{sec:org91f6a69}
প্রজাতন্ত্রের নির্বাহী কর্তৃত্বে সম্পত্তি গ্রহণ, বিক্রয়, হস্তান্তর, বন্ধকদান ও
বিলি-ব্যবস্থা, যে কোন কারবার বা ব্যবসায়-চালনা এবং যে কোন চুক্তি প্রণয়ন করা
যাইবে।

\subsection{১৪৫। চুক্তি ও দলিল}
\label{sec:orgbc90762}
(১) প্রজাতন্ত্রের নির্বাহী কর্তত্বে প্রণীত সকল চুক্তি ও দলিল রাষ্ট্রপতি কর্তৃক
    প্রণীত বলিয়া প্রকাশ করা হইবে এবং রাষ্ট্রপতি যেরূপ নির্দেশ বা ক্ষমতা প্রদান
    করিবেন, তাঁহার পক্ষে সেইরূপ ব্যক্তি কর্তৃক ও সেইরূপ প্রণালীতে তাহা সম্পাদিত
    হইবে।

(২) প্রজাতন্ত্রের নির্বাহী কর্তৃত্বে কোন চুক্তি বা দলিল প্রণয়ন বা সম্পাদন করা
    হইলে উক্ত কর্তৃত্বে অনুরূপ চুক্তি বা দলিল প্রণয়ন বা সম্পাদন করিবার জন্য রাষ্ট্রপতি
    কিংবা অন্য কোন ব্যক্তি ব্যক্তিগতভাবে দায়ী হইবেন না, তবে এই অনুচ্ছেদ সরকারের
    বিরুদ্ধে যথাযথ কার্যধারা আনয়নে কোন ব্যক্তির অধিকার ক্ষুণ্ন করিবে না।

\subsection{১৪৫ক। আন্তর্জাতিক চুক্তি}
\label{sec:org24bfbd6}
বিদেশের সহিত সম্পাদিত সকল চুক্তি রাষ্ট্রপতির নিকট পেশ করা হইবে, এবং
রাষ্ট্রপতি তাহা সংসদে পেশ করিবার ব্যবস্থা করিবেন:

তবে শর্ত থাকে যে, জাতীয় নিরাপত্তার সহিত সংশ্লিষ্ট অনুরূপ কোন চুক্তি কেবলমাত্র
সংসদের গোপন বৈঠকে পেশ করা হইবে।

\subsection{১৪৬। বাংলাদেশের নামে মামলা}
\label{sec:org33e2447}
“বাংলাদেশ”-এই নামে বাংলাদেশ সরকার কর্তৃক বা বাংলাদেশ সরকারের বিরুদ্ধে
মামলা দায়ের করা যাইতে পারিবে।

\subsection{১৪৭। কতিপয় পদাধিকারীর পারিশ্রমিক প্রভৃতি}
\label{sec:orgeaced2f}
(১) এই অনুচ্ছেদ প্রযোজ্য হয়, এইরূপ কোন পদে অধিষ্ঠিত বা কর্মরত ব্যক্তির
    পারিশ্রমিক, বিশেষ-অধিকার ও কর্মের অন্যান্য শর্ত সংসদের আইনের দ্বারা বা অধীন
    নির্ধারিত হইবে, তবে অনুরূপভাবে নির্ধারিত না হওয়া পর্যন্ত

(ক) এই সংবিধান প্রবর্তনের অব্যবহিত পূর্বে ক্ষেত্রমত সংশ্লিষ্ট পদে অধিষ্ঠিত বা
    কর্মরত ব্যক্তির ক্ষেত্রে তাহা যেরূপ প্রযোজ্য ছিল, সেইরূপ হইবে; অথবা

(খ) অব্যবহিত পূর্ববর্তী উপ-দফা প্রযোজ্য না হইলে রাষ্ট্রপতি আদেশের দ্বারা যেরূপ
    নির্ণয় করিবেন, সেইরূপ হইবে।

(২) এই অনুচ্ছেদ প্রযোজ্য হয়, এইরূপ কোন পদে অধিষ্ঠিত বা কর্মরত ব্যক্তির
    কার্যভারকালে তাঁহার পারিশ্রমিক, বিশেষ অধিকার ও কর্মের অন্যান্য শর্তের এমন
    তারতম্য করা যাইবে না, যাহা তাঁহার পক্ষে অসুবিধাজনক হইতে পারে।

(৩) এই অনুচ্ছেদ প্রযোজ্য হয়, এইরূপ কোন পদে নিযুক্ত বা কর্মরত ব্যক্তি কোন লাভজনক
    পদ কিংবা বেতনাদিযুক্ত পদ বা মর্যাদায় বহাল হইবেন না কিংবা মুনাফালাভের
    উদ্দেশ্যযুক্ত কোন কোম্পানী, সমিতি বা প্রতিষ্ঠানের ব্যবস্থাপনায় বা পরিচালনায়
    কোনরূপ অংশগ্রহণ করিবেন না:

তবে শর্ত থাকে যে, এই দফার উদ্দেশ্যসাধনকল্পে উপরের প্রথমোলি্লখিত পদে অধিষ্ঠিত
বা কর্মরত রহিয়াছেন, কেবল এই কারণে কোন ব্যক্তি অনুরূপ লাভজনক পদ বা
বেতনাদিযুক্ত পদ বা মর্যাদায় অধিষ্ঠিত বলিয়া গণ্য হইবেন না।

(৪) এই অনুচ্ছেদ নিম্নলিখিত পদসমূহে প্রযোজ্য হইবে:

(ক) রাষ্ট্রপতি,

(খ) প্রধানমন্ত্রী বা প্রধান উপদেষ্টা;

(গ) স্পীকার বা ডেপুটি স্পীকার,

(ঘ) মন্ত্রী, উপদেষ্টা, প্রতিমন্ত্রী বা উপ-মন্ত্রী;

(ঙ) সুপ্রীম কোর্টের বিচারক,

(চ) মহা হিসাব-নিরীক্ষক ও নিয়ন্ত্রক,

(ছ) নির্বাচন কমিশনার,

(জ) সরকারী কর্ম কমিশনের সদস্য।
\subsection{১৪৮। পদের শপথ}
\label{sec:org9c60d77}
(১) তৃতীয় তফসিলে উল্লিখিত যে কোন পদে নির্বাচিত বা নিযুক্ত ব্যক্তি কার্যভার
    গ্রহণের পূর্বে উক্ত তফসিল-অনুযায়ী শপথগ্রহণ বা ঘোষণা (এই অনুচ্ছেদে “শপথ” বলিয়া
    অভিহিত) করিবেন এবং অনুরূপ শপথপত্রে বা ঘোষণাপত্রে স্বাক্ষরদান করিবেন।

(২) এই সংবিধানের অধীন নির্দিষ্ট কোন ব্যক্তির নিকট শপথগ্রহণ আবশ্যক হইলে * * *
    অনুরূপ ব্যক্তি যেরূপ ব্যক্তি ও স্থান নির্ধারণ করিবেন, সেইরূপ ব্যক্তির নিকট সেইরূপ
    স্থানে শপথগ্রহণ করা যাইবে।

(ক) ১২৩ অনুচ্ছেদের (৩) দফার অধীন অনুষ্ঠিত সংসদ সদস্যদের সাধারণ নির্বাচনের
    ফলাফল সরকারী গেজেটে প্রজ্ঞাপিত হইবার তারিখ হইতে পরবর্তী তিন দিনের মধ্যে এই
    সংবিধানের অধীন এতদুদ্দেশ্যে নির্দিষ্ট ব্যক্তি বা তদুদ্দেশ্যে অনুরূপ ব্যক্তি কর্তৃক
    নির্ধারিত অন্য কোন ব্যক্তি যে কোন কারণে নির্বাচিত সদস্যদের শপথ পাঠ পরিচালনা
    করিতে ব্যর্থ হইলে বা না করিলে, প্রধান নির্বাচন কমিশনার উহার পরবর্তী তিন
    দিনের মধ্যে উক্ত শপথ পাঠ পরিচালনা করিবেন, যেন এই সংবিধানের অধীন তিনিই
    ইহার জন্য নির্দিষ্ট ব্যক্তি।

(৩) এই সংবিধানের অধীন যে ক্ষেত্রে কোন ব্যক্তির পক্ষে কার্যভার গ্রহণের পূর্বে
    শপথগ্রহণ আবশ্যক, সেই ক্ষেত্রে শপথ গ্রহণের অব্যবহিত পর তিনি কার্যভার গ্রহণ
    করিয়াছেন বলিয়া গণ্য হইবে।
\subsection{১৪৯। প্রচলিত আইনের হেফাজত}
\label{sec:org71248fd}
এই সংবিধানের বিধানাবলী-সাপেক্ষে সকল প্রচলিত আইনের কার্যকরতা অব্যাহত
থাকিবে, তবে অনুরূপ আইন এই সংবিধানের অধীন প্রণীত আইনের দ্বারা সংশোধিত বা
রহিত হইতে পারিবে।
\subsection{১৫০। ক্রান্তিকালীন ও অস্থায়ী বিধানাবলী}
\label{sec:orgf06dec5}
এই সংবিধানের অন্য কোন বিধান সত্ত্বেও চতুর্থ তফ্সিলে বর্ণিত ক্রান্তিকালীন ও
অস্থায়ী বিধানাবলী কার্যকর হইবে।
\subsection{১৫১। রহিতকরণ}
\label{sec:orgfe01753}
রাষ্ট্রপতির নিম্নলিখিত আদেশসমূহ এতদ্বারা রহিত করা হইল:

(ক) আইনের ধারাবাহিকতা বলবৎকরণ আদেশ (১৯৭১ সালের ১০ই এপ্রিল তারিখে প্রণীত);

(খ) ১৯৭২ সালের বাংলাদেশ অস্থায়ী সংবিধান আদেশ;

(গ) ১৯৭২ সালের বাংলাদেশ হাইকোর্ট আদেশ (১৯৭২ সালের পি.ও. নং ৫);

(ঘ) ১৯৭২ সালের বাংলদেশ মহা হিসাব-নিরীক্ষক ও নিয়ন্ত্রক আদেশ (১৯৭২ সালেরপি.ও. নং ১৫);

(ঙ) ১৯৭২ সালের বাংলাদেশ গণপরিষদ আদেশ (১৯৭২ সালের পি.ও. নং ২২);

(চ) ১৯৭২ সালের বাংলাদেশ নির্বাচন কমিশন আদেশ (১৯৭২ সালের পি.ও. নং ২৫);

(ছ) ১৯৭২ সালের বাংলাদেশ সরকারী কর্ম কমিশনসমূহ আদেশ (১৯৭২ সালের পি.ও. নং ৩৪);

(জ) ১৯৭২ সালের বাংলাদেশ (সরকারী কর্ম সম্পাদন) আদেশ (১৯৭২ সালের পি.ও. নং ৫৮)

\subsection{১৫২। ব্যাখ্যা}
\label{sec:org86cc91d}

“অধিবেশন” (সংসদ-প্রসঙ্গে) অর্থ এই সংবিধান-প্রবর্তনের পর কিংবা একবার স্থগিত
হইবার বা ভাঙ্গিয়া যাইবার পর সংসদ যখন প্রথম মিলিত হয়, তখন হইতে সংসদ স্থগিত
হওয়া বা ভাঙ্গিয়া যাওয়া পর্যন্ত বৈঠকসমূহ;

“অনুচ্ছেদ” অর্থ এই সংবিধানের কোন অনুচ্ছেদ;

“উপদেষ্টা” অর্থ ৫৮গ অনুচ্ছেদের অধীন উক্ত পদে নিযুক্ত কোন ব্যক্তি;

“অবসর-ভাতা” অর্থ আংশিকভাবে প্রদেয় হউক বা না হউক, যে কোন অবসর-ভাতা, যাহা
কোন ব্যক্তিকে বা ব্যক্তির ক্ষেত্রে দেয়; এবং কোন ভবিষ্য তহবিলের চাঁদা বা ইহার
সহিত সংযোজিত অতিরিক্ত অর্থ প্রত্যর্পণ-ব্যপদেশে দেয় অবসরকালীন বেতন বা আনুতোষিক
ইহার অন্তর্ভূ্ক্ত হইবে;

“অর্থ-বৎসর” অর্থ জুলাই মাসের প্রথম দিবসে যে বৎসরের আরম্ভ;

“আইন” অর্থ কোন আইন, অধ্যাদেশ, আদেশ, বিধি, প্রবিধান, উপ-আইন, বিজ্ঞপ্তি ও
অন্যান্য আইনগত দলিল এবং বাংলাদেশে আইনের ক্ষমতাসম্পন্ন যে কোন প্রথা বা রীতি;

“আদালত” অর্থ সুপ্রীম কোর্টসহ যে কোন আদালত;

“আপীল বিভাগ” অর্থ সুপ্রীম কোর্টের আপীল বিভাগ;

“উপ-দফা” অর্থ যে দফায় শব্দটি ব্যবহৃত, সেই দফার একটি উপ-দফা;

“ঋণগ্রহণ” বলিতে বাৎসরিক কিস্তিতে পরিশোধযোগ্য অর্থসংগ্রহ অন্তর্ভুক্ত হইবে; এবং
“ঋণ” বলিতে তদনুরূপ অর্থ বুঝাইবে;

“করারোপ” বলিতে সাধারণ, স্থানীয় বা বিশেষ-যে কোন কর, খাজনা, শুল্ক বা বিশেষ
করের আরোপ অন্তর্ভুক্ত হইবে; এবং “কর” বলিতে তদনুরূপ অর্থ বুঝাইবে;

“গ্যারান্টি” বলিতে কোন উদ্যোগের মুনাফা নির্ধারিত পরিমাণের অপেক্ষা কম হইলে
তাহার জন্য অর্থ প্রদান করিবার বাধ্যবাধকতা-যাহা এই সংবিধান-প্রবর্তনের পূর্বে
গৃহীত হইয়াছে-অন্তর্ভুক্ত হইবে;

“জেলা-বিচারক” বলিতে অতিরিক্ত জেলা-বিচারক অন্তর্ভুক্ত হইবেন;

“তফসিল” অর্থ এই সংবিদানের কোন তফসিল;

“দফা” অর্থ যে অনুচ্ছেদে শব্দটি ব্যবহৃত, সেই অনুচ্ছেদের একটি দফা;

“দেনা” বলিতে বাৎসরিক কিস্তি হিসাবে মূলধন পরিশোধের জন্য যে কোন
বাধ্যবাধকতাজনিত দায় এবং যে কোন গ্যারান্টিযুক্ত দায় অন্তর্ভুক্ত হইবে; এবং “দেনার
দায়” বলিতে তদনুরূপ অর্থ বুঝাইবে;

“নাগরিক” অর্থ নাগরিকত্ব-সম্পর্কিত আইনানুযায়ী যে ব্যক্তি বাংলাদেশের নাগরিক;

“প্রচলিত আইন” অর্থ এই সংবিধান-প্রবর্তনের অব্যবহিত পূর্বে বাংলাদেশের রাষ্ট্রীয়
সীমানায় বা উহার অংশবিশেষে আইনের ক্ষমতাসম্পন্ন কিন্তু কার্যক্ষেত্রে সক্রিয় থাকুক
বা না থাকুক, এমন যে কোন আইন;

“প্রজাতন্ত্র” অর্থ গণপ্রজাতন্ত্রী বাংলাদেশ;

“প্রজাতন্ত্রের কর্ম” অর্থ অসামরিক বা সামরিক ক্ষমতায় বাংলাদেশ সরকার-সংক্রান্ত
যে কোন কর্ম, চাকুরী বা পদ এবং আইনের দ্বারা প্রজাতন্ত্রের কর্ম বলিয়া ঘোষিত
হইতে পারে, এইরূপ অন্য কোন কর্ম;

“প্রধান উপদেষ্টা” অর্থ ৫৮গ অনুচ্ছেদের অধীন উক্ত পদে নিযুক্ত কোন ব্যক্তি;

“প্রধান নির্বাচন কমিশনার” অর্থ এই সংবিধানের ১১৮ অনুচ্ছেদের অধীন উক্ত পদে
নিযুক্ত কোন ব্যক্তি;

“প্রধান বিচারপতি” অর্থ বাংলাদেশের প্রধান বিচারপতি;

“প্রশাসনিক একাংশ” অর্থ জেলা কিংবা এই সংবিধানের ৫৯ অনুচ্ছেদের
উদ্দেশ্য-সাধনকল্পে আইনের দ্বারা অভিহিত অন্য কোন এলাকা;

“বিচারক” অর্থ সুপ্রীম কোর্টের কোন বিভাগের কোন বিচারক;

“বিচার-কর্মবিভাগ” অর্থ জেলা-বিচারক-পদের অনূর্ধ্ব কোন বিচারবিভাগীয় পদে
অধিষ্ঠিত ব্যক্তিদের লইয়া গঠিত কর্মবিভাগ;

“বৈঠক” (সংসদ-প্রসঙ্গে) অর্থ মূলতবী না করিয়া সংসদ যতক্ষণ ধারাবাহিকভাবে
বৈঠকরত থাকেন, সেইরূপ মেয়াদ;

“ভাগ” অর্থ এই সংবিধানের কোন ভাগ;

“রাজধানী” অর্থ এই সংবিধানের ৫ অনুচ্ছেদের রাজধানী বলিতে যে অর্থ করা হইয়াছে;

“রাজনৈতিক দল” বলিতে এমন একটি অধিসঙ্ঘ বা ব্যক্তিসমষ্টি অন্তর্ভুক্ত, যে অধিসঙ্ঘ
বা ব্যক্তিসমষ্টি সংসদের অভ্যন্তরে বা বাহিরে স্বাতন্ত্র্যসূচক কোন নামে কার্য করেন
এবং কোন রাজনৈতিক মত প্রচারের বা কোন রাজনৈতিক তৎপরতা পরিচালনার উদ্দেশ্যে
অন্যান্য অধিসঙ্ঘ হইতে পৃথক কোন অধিসঙ্ঘ হিসাবে নিজদিগকে প্রকাশ করেন;

“রাষ্ট্র” বলিতে সংসদ, সরকার ও সংবিধিবদ্ধ সরকারী কর্তৃপক্ষ অন্তর্ভুক্ত;

“রাষ্ট্রপতি” অর্থ এই সংবিধানের অধীন নির্বাচিত বাংলাদেশের রাষ্ট্রপতি কিংবা
সাময়িকভাবে উক্ত পদে কর্মরত কোন ব্যক্তি;

“শৃঙ্খলা-বাহিনী” অর্থ

(ক) স্থল, নৌ বা বিমান-বাহিনী;

(খ) পুলিশ-বাহিনী;

(গ) আইনের দ্বারা সংজ্ঞার অর্থের অন্তর্গত বলিয়া ঘোষিত যে কোন শৃঙ্খলা-বাহিনী;

“শৃঙ্খলামূলক আইন” অর্থ শৃঙ্খলা- বাহিনীর নিয়ন্ত্রণকারী কোন আইন;

“সংবিধিবদ্ধ সরকারী কতর্ৃপক্ষ” অর্থ যে কোন কর্তৃপক্ষ, সংস্থা বা প্রতিষ্ঠান, যাহার
কার্যাবলী বা প্রধান প্রধান কার্য কোন আইন, অধ্যাদেশ, আদেশ বা বাংলাদেশে আইনের
ক্ষমতাসম্পন্ন চুক্তিপত্র-দ্বারা অর্পিত হয়;

“সংসদ” অর্থ এই সংবিধানের ৬৫ অনুচ্ছেদ-দ্বারা প্রতিষ্ঠিত বাংলাদেশের সংসদ;

“সম্পত্তি” বলিতে সকল স্থাবর ও অস্থাবর, বস্তুগত ও নির্বস্তগত সকল প্রকার সম্পত্তি,
বাণিজ্যিক ও শিল্পগত উদ্যোগ এবং অনুরূপ সম্পত্তি বা উদ্যোগের সহিত সংশ্লিষ্ট যে কোন
স্বত্ব বা অংশ অন্তর্ভুক্ত হইবে;

“সরকারী কর্মচারী” অর্থ প্রজাতন্ত্রের কর্মে বেতনাদিযুক্ত পদে অধিষ্ঠিত বা কর্মরত
কোন ব্যক্তি;

“সরকারী বিজ্ঞপ্তি” অর্থ বাংলাদেশে গেজেটে প্রকাশিত কোন বিজ্ঞপ্তি;

“সিকিউরিটি” বলিতে স্টক অন্তর্ভুক্ত হইবে;

“সুপ্রীম কোর্ট” অর্থ এই সংবিধানের ৯৪ অনুচ্ছেদ-দ্বারা গঠিত বাংলাদেশের সুপ্রীম
কোর্ট;

“স্পীকার” অর্থ এই সংবিধানের ৭৪ অনুচ্ছেদ-অনুসারে সাময়িকভাবে স্পীকারের পদে
অধিষ্ঠিত ব্যক্তি;

“হাইকোর্ট বিভাগ” অর্থ সুপ্রীম কোর্টের হাইকোর্ট বিভাগ।

(২) ১৮৯৭ সালের জেনারেল ক্লজেস্ অ্যাক্ট

(ক) সংসদের কোন আইনের ক্ষেত্রে যেরূপ প্রযোজ্য, এই সংবিধানের ক্ষেত্রে সেইরূপ
    প্রযোজ্য হইবে;

(খ) সংসদের কোন আইনের দ্বারা রহিত কোন আইনের ক্ষেত্রে যেরূপ প্রযোজ্য, এই
    সংবিধানের দ্বারা রহিত কিংবা এই সংবিধানের কারণে বাতিল বা কার্যকরতালুপ্ত কোন
    আইনের ক্ষেত্রে সেইরূপ প্রযোজ্য হইবে।

\subsection{১৫৩। প্রবর্তন, উল্লেখ ও নির্ভরযোগ্য পাঠ}
\label{sec:orgca792c1}
(১) এই সংবিধানকে “গণপ্রজাতন্ত্রী বাংলাদেশের সংবিধান” বলিয়া উল্লেখ করা হইবে
    এবং ১৯৭২ সালের ডিসেম্বর মাসের ১৬ তারিখে ইহা বলবৎ হইবে, যাহাকে এই
    সংবিধানে “সংবিধান-প্রবর্তন” বলিয়া অভিহিত করা হইয়াছে।

(২) বাংলায় এই সংবিধানের একটি নির্ভরযোগ্য পাঠ ও ইংরাজীতে অনুদিত একটি
    নির্ভরযোগ্য অনুমোদিত পাঠ থাকিবে এবং উভয় পাঠ নির্ভরযোগ্য বলিয়া গণপরিষদের
    স্পীকার সার্টিফিকেট প্রদান করিবেন।

(৩) এই অনুচ্ছেদের (২) দফা-অনুযায়ী সার্টিফিকেটযুক্ত কোন পাঠ এই সংবিধানের
    বিধানাবলীর চূড়ান্ত প্রমাণ বলিয়া গণ্য হইবে:

তবে শর্ত থাকে যে, বাংলা ও ইংরাজী পাঠের মধ্যে বিরোধের ক্ষেত্রে বাংলা পাঠ
প্রাধান্য পাইবে।

\section{Central Bank}
\label{sec:orgce05cd2}
\begin{itemize}
\item USA   \textbf{Federal Reserve System}
\item Lebanon  \textbf{Bank of Lebanon/Banque du Liban}
\item Canada \textbf{Bank of Canada/Bank Du Canada}
\item Italy \textbf{The Bank of Italy}
\item France  \textbf{Bank of France/Banque de France}
\item United Kingdom  \textbf{Bank of England}
\item South Korea  \textbf{Bank of Korea}
\item China  \textbf{People's Bank of China}
\item India  \textbf{Reserve Bank of India}
\item Sri Lanka  \textbf{Central Bank of Sri Lanka}
\item Russia  \textbf{Central Bank of Russian Federation (CBR)}
\item Ukraine  \textbf{National Bank of Ukraine}
\item Qatar \textbf{Qatar Central Bank}
\item Venezuela \textbf{The Central Bank of Venezuela}
\item Turkey \textbf{Central Bank of the Republic of Turkey (CBRT)}
\item Iran  \textbf{Central Bank of Islamic Republic of Iran}
\item Saudi Arabia  \textbf{Saudi Central Bank}
\item Japan \textbf{The Bank of Japan}
\item Israel  \textbf{The Bank of Israel}
\item Malaysia \textbf{Central Bank of Malaysia/Bank Negara Malaysia}
\item Philippines \textbf{Central Bank of Philippines/Bangko Sentral ng Pilipinas}
\item Taiwan \textbf{Central Bank of the Republic of China (Taiwan)}
\item Brunei  \textbf{Brunei Darussalam Central Bank}
\item United Arab Emirates \textbf{Central Bank of the UAE}
\item Barbados \textbf{Central Bank of Barbados}
\item Mexico \textbf{The Bank of Mexico/Banco de Mexico}
\item Maldives \textbf{Maldives Monetary Authority}
\item Germany \textbf{Central Bank of the Federal Republic of Germany}
\item Yemen  \textbf{Central Bank of Yemen}
\item Nepal \textbf{Nepal Rastra Bank}
\item Brazil \textbf{The Central Bank of Brazil}
\end{itemize}

\section{পিএসসি নির্ধারিত ১১ জন কবি সাহিত্যিক [68\%]}
\label{sec:orge8c9c60}

\begin{itemize}
\item[{$\boxtimes$}] মীর মশাররফ হোসেন (১৮৪৭- ১৯১১)
\item[{$\boxtimes$}] বঙ্কিমচন্দ্র চট্টোপাধ্যায়  (১৮৩৮-১৮৯৪)
\item[{$\boxtimes$}] ঈশ্বরচন্দ্র বিদ্যাসাগর (১৮২০-১৮৯১)
\item[{$\boxtimes$}] মাইকেল মধুসূদন দত্ত (১৮২৪- ১৮৭৩)
\item[{$\boxtimes$}] রবীন্দ্রনাথ ঠাকুর (১৮৬১-১৯৪১)
\item[{$\square$}] দীনবন্ধু মিত্র (১৮৩০-১৮৭৩)
\item[{$\boxtimes$}] কাজী নজরুল ইসলাম (১৮৯৯ – ১৯৭৬)
\item[{$\boxtimes$}] জসীম উদ্দীন (১৯০৩-১৯৭৬)
\item[{$\boxtimes$}] ফররুক আহমদ (১৯১৮-১৯৭৪)
\item[{$\boxtimes$}] কায়কোবাদ (১৮৫৭-১৯৫১)
\item[{$\boxtimes$}] বেগম রোকেয়া সাখাওয়াত হোসেন (১৮৮০-১৯৩২)
\item[{$\square$}] অমিয় চক্রবর্তী (১৯০১-৮৭)
\item[{$\square$}] বুদ্ধদেব বসু (১৯০৮-৭৪)
\item[{$\boxtimes$}] জীবনান্দ দাশ (১৮৯৯-১৯৫৪)
\item[{$\square$}] বিষ্ণু দে (১৯০৯-৮২)
\item[{$\square$}] সুধীন্দ্রনাথ দত্ত (১৯০১-৬০)
\end{itemize}


\section{গুরুত্বপূর্ন সাহিত্য কর্ম এবং তাদের প্রকাশ কাল}
\label{sec:orgf4ec84e}

\subsection{মহাশ্মশান (১৯০৫)}
\label{sec:org31f9d47}
\subsection{অশ্রুমালা (১৮৯৫)}
\label{sec:orged2097b}
\subsection{শিবমন্দির (১৯২১)}
\label{sec:org11b19fa}
\subsection{অমিয়ধারা (১৯২৩)}
\label{sec:orgc3d9b85}
\subsection{মেঘনাদবধ কাব্য (১৮৬১)}
\label{sec:orga473d12}
\subsection{শর্মিষ্ঠা (১৮৫৮)}
\label{sec:orge8c28a6}
\subsection{পদ্মাবতী (১৮৬০)}
\label{sec:org374b636}
\subsection{কৃষ্ণকুমারী (১৮৬১)}
\label{sec:org08fa516}
\subsection{তিলোত্তমাসম্ভব (১৮৬০)}
\label{sec:orga70a880}
\subsection{ব্রজাঙ্গনা (১৮৬১)}
\label{sec:org681e7a7}
\subsection{বীরাঙ্গনা (১৮৬২)}
\label{sec:orgfc1043e}
\subsection{চতুর্দশপদী কবিতাবলী (১৮৬৬)}
\label{sec:orgd9fc787}
\subsection{অগ্নিবীণা (১৯২২)}
\label{sec:orgb4c3132}
\subsection{দোলন-চাপা (১৯২৩)}
\label{sec:orge16a95a}
\subsection{বিষের বাঁশী (১৯২৪)}
\label{sec:org911c75d}
\subsection{ছায়ানট (১৯২৪)}
\label{sec:org05d9f10}
\subsection{ভাঙার গান (১৯২৪)}
\label{sec:org4e829f4}
\subsection{সাম্যবাদী (১৯২৫)}
\label{sec:orgeed3220}
\subsection{সর্বহারা (১৯২৬)}
\label{sec:orgdeb85fc}
\subsection{ফণি মনসা (১৯২৭)}
\label{sec:orgc9761d6}
\subsection{সিন্দু হিন্দোল (১৯২৭)}
\label{sec:org5a82daf}
\subsection{সন্ধ্যা (১৯২৯)}
\label{sec:org956f8f7}
\subsection{চক্রবাক (১৯২৯)}
\label{sec:orgba4c6f8}
\subsection{প্রলয় শিখা (১৯৩০)}
\label{sec:orgb07493a}


\section{বাংলাদেশের রিজার্ভ পরিস্থিতি}
\label{sec:org9dc60bc}
\subsection{রিজার্ভ কাকে বলে}
\label{sec:org07c254e}
রপ্তানি, রেমিট্যান্স, ঋণ বা অন্যান্য উৎস থেকে আসা বৈদেশিক মুদ্রা থেকে আমদানি,
ঋণ ও সুদ পরিশোধ, বিদেশে শিক্ষা ইত্যাদি নানা খাতে যাওয়া বৈদেশিক মুদ্রা বাদ
দেয়ার পর কেন্দ্রীয় ব্যাংকের কাছে যে বৈদেশিক মুদ্রা সঞ্চিত থাকে, সেটাই
বৈদেশিক মুদ্রার রিজার্ভ।

\subsection{রিজার্ভের বর্তমান প্ররিস্থিতি}
\label{sec:orga6f4136}

বর্তমানে রিজার্ভ \textbf{৩৫ দশমিক ৮ বিলিয়ন ডলার} (২৬ অক্টোবর)। ইডিএফ ঋণ \textbf{প্রায় ৭
দশমিক ৮ বিলিয়ন ডলার} । এ ছাড়া রিজার্ভ থেকে বাংলাদেশ বিমান, পায়রা
সমুদ্রবন্দরও অর্থনৈতিক বিপর্যয়ে পড়া শ্রীলঙ্কাকে ঋণ দেওয়া হয়েছে। সব মিলিয়ে এ
জন্য মোট দেওয়া হয়েছে \textbf{প্রায় ৮ দশমিক ৪ বিলিয়ন ডলার} । আইএমএফের হিসাবে এই
অর্থ রিজার্ভ থেকে বিয়োগ করলে প্রকৃত রিজার্ভ দাঁড়ায় \textbf{২৭ দশমিক ৪ বিলিয়ন ডলার} ।
দেশের বিদ্যুৎ ও জ্বালানি খাতে সরকারের দেনা আছে \textbf{চার বিলিয়ন ডলার}, অপরাপর
দায় ও আমদানির জন্য এশিয়ান ক্লিয়ারিং হাউসে দেনা আরও \textbf{দুই বিলিয়ন ডলার} । সঙ্গে
আছে বৈদেশিক ঋণের কিস্তি বাবদ দেনা আরও \textbf{পাঁচ বিলিয়ন ডলার।}  সেই হিসেবে এই
মুহূর্তে ব্যবহারযোগ্য রিজার্ভ আছে \textbf{মাত্র ১৬ দশমিক ৪ বিলিয়ন ডলার} । এর সঙ্গে যোগ
হবে প্রবাসী আয় ও রপ্তানি আয়, কিন্তু সেটা আমদানি ব্যয়ে খরচ হয়ে পড়বে বলে
রিজার্ভে তেমন যোগ হবে না। \textbf{বাংলাদেশ বিগত অর্থবছরে ৩৩ বিলিয়ন ডলার}
বাণিজ্যঘাটতি প্রত্যক্ষ করেছে, অর্থাৎ আমাদের রপ্তানির চেয়ে আমদানি বিল ৩৩
বিলিয়ন ডলার বেশি ছিল।ফলে \textbf{প্রবাসী আয় হিসেবে বছরে আসা ২০ বিলিয়ন ডলারও
কিন্তু ঋণাত্মক বাণিজ্যঘাটতি} (আমদানি বিয়োগ রপ্তানি) ও বিদেশি ঋণের কিস্তি
পরিশোধের পরে রিজার্ভে যুক্ত হওয়ার সম্ভাবনা কম। স্বাভাবিক সময়ের হিসাবে
\textbf{বর্তমান রিজার্ভ মাত্র আড়াই মাসের আমদানি বিলের সমান কিংবা কিছু
বেশি।} বাংলাদেশ ব্যাংকের ফরেন এক্সচেঞ্জ পলিসি ডিপার্টমেন্টের প্রতিবেদনেমতে,
২০২১ সালের আমদানি দায় (মূলত বেসরকারি, তবে সরকারি প্রকল্পসংশ্লিষ্ট আমদানিও
আছে) পরিশোধ পিছিয়ে দেওয়ার (এলসি ডেফার্ড) ফলে \textbf{মোট আমদানি দায়ের ৩০ শতাংশই}
স্বল্পমেয়াদি বৈদেশিক ঋণে পরিণত হয়ে গেছে। ২০২১ সাল পর্যন্ত সৃষ্ট, \textbf{মোট ১৭
বিলিয়ন বৈদেশিক ঋণের ৫৭ শতাংশই এভাবে কিউমুলেটেড হয়েছে।} ফলে ২০২২ সালে এসে
বিদেশি ঋণের মোট সরকারি ও বেসরকারি দেনা অস্বাভাবিক পর্যায়ে চলে গেছে।
স্বল্পমেয়াদি দায়ের সঙ্গে দীর্ঘমেয়াদি এবং অন্যান্য অভ্যন্তরীণ ঋণ মিলে বর্তমানে
বিদেশি ঋণের কিস্তির \textbf{মোট দেনা ২৩ দশমিক ৪০ বিলিয়ন ডলার।}

\subsection{রিজার্ভের ব্যবহার কোথায় হয়}
\label{sec:org0a3d6c0}
বৈদেশিক মুদ্রার বিপরীতে টাকার বিনিময় হার এবং দেশের ঋণমান (ক্রেডিট রেটিং)
নির্ধারণে রিজার্ভের গুরুত্ব অপরিসীম। বাজারে বৈদেশিক মুদ্রার সংকট থাকলে
কেন্দ্রীয় ব্যাংকের রিজার্ভ থেকে জোগান হয়। বৈদেশিক মুদ্রার বিপরীতে টাকার
বিনিময় হার যৌক্তিক পর্যায়ে ধরে রাখতে বাজারের চাহিদা অনুযায়ী বৈদেশিক মুদ্রায়
তারল্য বাড়ানো হয়, ফলে বিনিময় হার দীর্ঘদিন ধরেই স্থিতিশীল।
\subsection{রিজার্ভের ব্যবহারে ভুল সিদ্ধান্ত}
\label{sec:orga930952}
হিমালয়ান মিহি পলিবাহিত বৃহৎ নদীগুলোর মোহনার মাত্র সাড়ে ১০ মিটার গভীর
চ্যানেলে নিয়মিত ক্যাপিটাল ড্রেজিং করে পায়রায় গভীর সমুদ্রবন্দর অসম্ভব।পায়রা
বন্দরে বৈদেশিক মুদ্রার রিজার্ভ থেকে ঋণ প্রদানের সিদ্ধান্তকে কোনোভাবেই যৌক্তিক
বলা যাবে না। পায়রা বন্দর নিজেই অকার্যকর থেকে যাবে বলে সরকারও ভুল জায়গায়
গভীর সমুদ্রবন্দরের পরিকল্পনা থেকে সরে এসেছে। তদুপরি এই বন্দর ডলারে আয় করবে
না বলে এখানে রিজার্ভ ঋণ পরিকল্পনা সঠিক ছিল না। ডলারে আয় করবে না, এমন
প্রকল্পে রিজার্ভ ঋণে বাংলাদেশ ব্যাংকেরও আপত্তি ছিল। শুধু ইনফ্রাস্ট্রাকচার
ডেভেলপমেন্ট ফান্ডে নয়; বরং সরকার রপ্তানি সহায়তা ফান্ড হিসেবেও ঋণ দিয়েছে
সরকার। বহু ব্যবসায়ী ঋণপত্র বা এলসির আমদানির ওভার ইনভয়েসিং করে অর্থ পাচার
করে। অর্থাৎ রিজার্ভ থেকে বাছবিচারহীন অবকাঠামো (আইডিএফ) ও রপ্তানি (ইডিএফ)
সহায়তা ঋণ দিয়ে দেশের রিজার্ভকে সংকটাপন্ন করার ঝুঁকি নেওয়া হয়েছিল।

\subsection{রিজার্ভ সংরক্ষণে করনীয়}
\label{sec:org9b25aec}
সরকারের হাতে সম্ভাব্য বিকল্প দুটি, আরও \textbf{অধিক হারে এলসি বন্ধ করা} এবং আরও \textbf{বেশি
বৈদেশিক ঋণ করা} । এলসি ক্রমাগত বন্ধ হলে রিজার্ভ কমার হার কমবে, কিন্তু
অর্থনীতিতে নতুন বহুমুখী সংকট শুরু হবে। আরও বেশি বৈদেশিক ঋণ নিলে দেনাও বাড়বে।
রপ্তানি খাতের কাঁচামাল, ক্যাপিটাল মেশিনারিসহ জরুরি আমদানি বন্ধ করতে হবে।
ইতিমধ্যেই সরকার রিজার্ভ বাঁচাতে প্রাথমিক জ্বালানি আমদানি কমিয়েছে, এতে
বিদ্যুৎকেন্দ্র বন্ধ রেখে পরিকল্পিত লোডশেডিং হচ্ছে। চলমান ভয়াবহ বিদ্যুৎ ও
জ্বালানিসংকট জনজীবন, কর্মসংস্থান, শিল্প উৎপাদনসহ সার্বিক অর্থনীতিতে বিপর্যয়
তৈরি করেছে।সরকার \textbf{ফেব্রুয়ারি থেকেই এলসি নিষ্পত্তি নিয়ন্ত্রণ করে আসছে} । এতে আমদানি ব্যয়
কিছুটা কমেছে, তবে খোলাবাজারে ডলার-সংকট বেড়েছে। অর্থাৎ বাজারে ডলার ছাড়তে
হচ্ছে বলে ঋণপত্র বন্ধ করেও রিজার্ভ হ্রাসের সমস্যা থামছে না। রিজার্ভ হ্রাসের
ঝুঁকির মধ্যেও সরকারকে খাদ্যনিরাপত্তা ও জ্বালানিনিরাপত্তা সুরক্ষিত রাখতে
খাদ্যপণ্য, সার, ডিজেল, গ্যাসসহ জ্বালানি ও অপরাপর জরুরি আমদানি চালিয়ে যেতে
হবে।

\section{রাশিয়া-ইউক্রেন যুদ্ধ: বাংলাদেশের উপর প্রভাব}
\label{sec:org19d460d}
\subsection{গম আমদানিতে প্রভাব}
\label{sec:org89f298b}
বাংলাদেশের মানুষ ক্যালরির জন্য ভাতের উপর নির্ভরশীল। তবে আটা-ময়দা উপরও ধীরে
ধীরে নির্ভরশীলতা বাড়ছে। আর এই আটা-ময়দা আসে গম থেকে। ইন্টারন্যাশনাল ফুড
পলিসি রিসার্চ ইনস্টিটিউট-এর এর গবেষণা প্রবন্ধে বলা হয়েছে, গত ২০ বছরে
বাংলাদেশে গমের চাহিদা প্রায় দ্বিগুণ হয়েছে। \textbf{বাংলাদেশে তার চাহিদার ৮০ শতাংশ
গম আমদানি করে।} \textbf{এর অর্ধেক আসে ইউক্রেন এবং রাশিয়া থেকে।} কিন্তু সে সরবরাহ
ব্যবস্থা বাধাগ্রস্ত হবার কারণে গমের দাম বেড়েছে। ফলে বাংলাদেশ আটা-ময়দার
দামও বেড়েছে। সেই সাথে আটা-ময়দা দিয়ে তৈরি খাদ্যর দাম বেড়েছে বেশ খানিকটা।

\subsection{ভোজ্য তেলের উপর প্রভাব}
\label{sec:orgfdb7450}
বাংলাদেশে ভোজ্য তেলের দাম এখন লাগামছাড়া। যদিও বাংলাদেশের বাংলাদেশে ভোজ্য
তেলের বেশিরভাগই আসে পাম অয়েল এবং সয়াবিন অয়েল থেকে। বিশ্বজুড়ে যেসব ভোজ্য
তেল ব্যবহার হয় তার মধ্যে সানফ্লাওয়ার তেল \textbf{প্রায় ১৩ শতাংশ।} এর \textbf{প্রায় ৭৫ শতাংশই}
আসে ইউক্রেন এবং রাশিয়া থেকে। যেহেতু এই সরবরাহ ব্যবস্থা বিঘ্নিত হচ্ছে, সেজন্য
বিশ্বজুড়ে ভেজিটেবল অয়েলের চেইনের দাম বৃদ্ধি পেয়েছে।ইন্টারন্যাশনাল ফুড পলিসি
রিসার্চ ইনিস্টিটিউট-এর এর গবেষণায় বলা হয়েছে বাংলাদেশ তার প্রয়োজনীয়
ভেজিটেবল অয়েল আমদানি করে হয় কাঁচামাল হিসেবে, নয়তো প্রক্রিয়াজাত পণ্য হিসেবে
( পাম অয়েল এবং সয়াবিন অয়েল)। অথবা তেলবীজ আমদানি করে সেটিকে স্থানীয়ভাবে
প্রক্রিয়াজাত করে। মালয়েশিয়া এবং ইন্দোনেশিয়া পাম অয়েল রপ্তানি করে। কিন্তু
একদিকে করোনাভাইরাস মহামারি-পরবর্তী শ্রমিক সংকটে কারণে দাম কিছুটা উর্ধ্বমুখী
ছিল। পরবর্তীতে ইউক্রেন যুদ্ধ সে দাম আরো বাড়িয়ে দিয়েছে।

\subsection{হাঁস-মুরগি ও গরুর খাবার}
\label{sec:org2e0982e}
পোল্ট্রি ফিড উৎপাদনের একটি গুরুত্বপূর্ণ উপাদান হচ্ছে ভুট্টা।বিশ্ববাজারে ইউক্রেন
\textbf{১৬ শতাংশ ভুট্টা} সরবরাহ করে। পৃথিবীর আরো অনেক দেশে ভুট্টা উৎপাদিত হয়। যেহেতু
ইউক্রেন থেকে ভুট্টা সরবরাহ আসতে পারছে না সেজন্য বিশ্বজুড়ে পোল্ট্রি ফিড-এর দাম
বেড়েছে। এর ফলে বাজারে মুরগী ও ডিমের দাম বেড়ে গেছে। বাংলাদেশ ফিড
ইন্ডাট্রিজ এসোসিয়েশনের উপদেষ্টা দেবাশিষ নাগ বিবিসি বাংলাকে বলেন, পোল্ট্রি
\textbf{ফিড উৎপাদনের ৬০ শতাংশ উপকরণ আমদানি} করতে হয়। এর মধ্যে সবচেয়ে বড় উপকরণ
হচ্ছে ভুট্টা। ভুট্টার দাম অনেক বেড়ে গেছে। \textbf{পোল্ট্রি ফিডের মূল উপাদানে মধ্যে ভুট্টা
এবং সয়াবিন মিল (সয়াবিনের ভুষি}) - এ দুটো হচ্ছে মূল উপাদান।

\subsection{সার আমদানিতে খরচ}
\label{sec:org7024d46}
\textbf{বিশ্ববাজারে সিংহভাগ সার রপ্তানির ক্ষেত্রে রাশিয়া এবং বেলারুশের ভূমিকা রয়েছে।}
রাশিয়ার উপর রপ্তানি নিষেধাজ্ঞা দেবার ফলে সেটি বাংলাদেশকেও প্রভাবিত করবে।
বিশেষজ্ঞরা বলছেন, বাংলাদেশ রাশিয়া থেকে সার আমদানি করে। সেটি ব্যাহত হলে
ভিন্ন কোন উৎস দেখতে হবে। এতে করে খরচ বাড়বে কৃষি খাতে।

\subsection{জ্বালানী তেলের দাম}
\label{sec:org763b8b7}
সারা দুনিয়ায় ব্যবহৃত \textbf{প্রতি ১০ ব্যারেল তেলের এক ব্যারেল} উৎপাদিত হয় বিশ্বের
\textbf{তৃতীয় বৃহত্তম জ্বালানি তেল উৎপাদনকারী} দেশ রাশিয়ায়।  রাশিয়া-ইউক্রেন যুদ্ধের
কারণে বিশ্বজুড়ে জ্বালানী তেলের দাম হু হু করে বেড়েছে। বিশ্ববাজারে তেল-গ্যাস
সরবরাহের ক্ষেত্রে রাশিয়া বেশ গুরুত্বপূর্ণ। জ্বালানী তেলের আমদানি ব্যয় মেটাতে
এখন হিমশিম খাচ্ছে বাংলাদেশ সরকার। অর্থনীতিবিদরা বলছেন, আসছে বাজেটে
জ্বালানী তেলের ভর্তুকির জন্য সরকারকে অনেক টাকা গুণতে হবে। অন্যথায় তেলের দাম
বাড়াতে হবে। কিন্তু তেলের দাম বাড়লে দ্রব্যমূল্য আরো বাড়বে।

\subsection{পোশাক রপ্তানিতে প্রভাব}
\label{sec:org54251d4}
বাংলাদেশের \textbf{গার্মেন্টস খাতের অর্ধেকের বেশি বা ৬৪ শতাংশ} তৈরি পোশাক রপ্তানি
হয় ইউরোপে। বলা হয়, ইউরোপে ব্যবহৃত প্রতি ৩টা ডেনিম বা জিন্সের পোশাকের
একটি তৈরি হয় বাংলাদেশে।যুদ্ধের কারণে ইউরোপে মন্দা দেখা দিলে বাংলাদেশের
পোশাক রপ্তানিতে বিরূপ প্রভাব পড়বে । \textbf{বাংলাদেশ থেকে রাশিয়ায় ৬৫০ মিলিয়ন
ডলারের তৈরি পোশাক} রপ্তানি হয়। রাশিয়ার বেশ কয়েকটি ব্যাংককে বৈশ্বিক
আন্তব্যাংক লেনদেনসংক্রান্ত সুইফট সিস্টেমে নিষিদ্ধ করার ফলে রাশিয়ায় তৈরি
পোশাক রপ্তানি হুমকির মধ্যে পড়তে পারে। ফলে যেসব পোশাকের অর্ডার শিপমেন্টের
জন্য প্রস্তুত রয়েছে, তার মূল্য পাওয়া নিয়ে শঙ্কা রয়েছে।

\subsection{গ্যাস আমদানিতে প্রভাব}
\label{sec:orgde6a472}
ইউরোপের একটি বড় অংশ অচল হয়ে যাবে যদি বিশ্বের \textbf{দ্বিতীয় বৃহত্তম প্রাকৃতিক গ্যাস}
উৎপাদনকারী দেশ রাশিয়া তার গ্যাস ও তেলের সরবরাহ বন্ধ করে দেয়।
রাশিয়া-ইউক্রেন যুদ্ধের কারণে দেশে তরলীকৃত প্রাকৃতিক গ্যাসের (এলএনজি) দাম
বাড়ার আশঙ্কা করা হচ্ছে। পররাষ্ট্র মন্ত্রণালয়ের প্রতিবেদনে বলা হয়, বাংলাদেশ
মূলত \textbf{কাতার ও ওমান} থেকে এলএনজি আমদানি করে থাকে। কিন্তু যুদ্ধের কারণে ইউরোপের
দেশগুলো রাশিয়ার ওপর নির্ভরশীলতা কমিয়ে কাতার, ওমান, ইন্দোনেশিয়ার মতো
দেশগুলোর দিকে ঝুঁকতে পারে। এতে এই বাজারগুলোতে এলএনজির দাম বেড়ে যাওয়ার
আশঙ্কা আছে।
\subsection{রূপপুর পারমাণবিক বিদ্যুৎকেন্দ্র ও বঙ্গবন্ধু স্যাটেলাইট-২}
\label{sec:org955ef91}
এই যুদ্ধের কারণে রূপপুর পারমাণবিক বিদ্যুৎকেন্দ্র নির্মাণ প্রকল্পে বিরূপ প্রভাব এবং
বঙ্গবন্ধু স্যাটেলাইট-২সহ যেসব প্রকল্পে রাশিয়ার বিভিন্ন প্রতিষ্ঠানের সঙ্গে চুক্তি
বা সমঝোতা স্মারক সই হয়েছে, সেগুলো দীর্ঘসূত্রতার কবলে পড়তে পারে বলে শঙ্কার
কথা জানিয়েছে পররাষ্ট্র মন্ত্রণালয়।

\section{Have to Read}
\label{sec:org269bda4}

\begin{enumerate}
\item \href{https://magoosh.com/gmat/gmat-probability-questions/}{Probability}
\item \href{//WWW.YOUTUBE.COM/WATCH?V=74HW-FS0IVI\&T=33S}{Divisibilit}
\item \href{//WWW.gmatprepnow.com/module/gmat-probability/video/754}{Probability practice Gmat Club}
\item \href{https://www.prothomalo.com/business/wjhmt1mvri}{রপ্তানি ও প্রবাসী আয় আয় কমল টানা দুই মাস}
\end{enumerate}

\section{Important Banking Terms}
\label{sec:org1b8b09c}

\begin{itemize}
\item ADB \textbf{Asian Development Bank}
\item AGM  \textbf{Annual General Meeting}
\item ALCO \textbf{Assests Liability Committee}
\item ATM \textbf{Automated Teller machine}
\item BACH \textbf{Bangladesh automated Clearing House}
\item BACPS \textbf{Bangladesh Automated Cheque Processing System}
\item BALCO \textbf{Branch Anti-Money Laundering Compliance Officer}
\item BB \textbf{Bangladesh Bank}
\item BCP  \textbf{Basel Core Principle}
\item BEFTN \textbf{Bangladesh Electronic Fund Transfer Network}
\item BRPD \textbf{Banking Regulatory Policy Department}
\item CAD \textbf{Credit Administrative Department}
\item CAMELS \textbf{Capital Assests Management Earning Liquidity Securities Analysis}
\item CAR \textbf{Capital Adequacy Ratio}
\item CBA \textbf{Collective Bargaining Agent}
\item CCU \textbf{Central Compliance Unit}
\item CIB \textbf{Credit Information Bureau}
\item CIF \textbf{Customer Information Form}
\item COI \textbf{Control Of Insurance}
\item CRAB \textbf{Credit Rating Agency of Bangladesh}
\item CRISL \textbf{Credit Rating Information Service Ltd}
\item CRM  \textbf{Credit Risk Management}
\item CRR \textbf{Cash Reserve Requirement}
\item CTR  \textbf{Cash Transaction Report}
\item DD \textbf{Demand Draft}
\item DDA \textbf{Direct Deposit Access}
\item DMB  \textbf{Deposit Money Bank}
\item DSCR  \textbf{Debt Service Cover Ratio}
\item EFT  \textbf{Electronic Fund Transfer}
\item EGM  \textbf{Extra- Ordinary General Meeting}
\item EPB  \textbf{Export Promotion Bureau}
\item EPS \textbf{Earning Per Share}
\item FDA  \textbf{Fixed Deposit Account}
\item FDBP  \textbf{Foreign Development Bill Purchase}
\item FDD  \textbf{Foreign Demand Draft}
\item FIs \textbf{Financial Institutions}
\item FIU  \textbf{Financial Intelligence Unit}
\item FSRP  \textbf{Financial Sector Reform Project}
\item FSS  \textbf{Financial Spread Sheet}
\item IBC  \textbf{Inward Bill For Collection}
\item ICC  \textbf{International Chamber of Commerce}
\item ICDEP  \textbf{Interest During Construction Period}
\item IMF \textbf{International Monetary Fund}
\item IPO  \textbf{Initial Public Offering}
\item KYC \textbf{Know Your Customer}
\item LAAB \textbf{Loan Against Accepted Bill}
\item LDBP \textbf{Local Development Bill Purchase}
\item LIM+ \textbf{Loan against Imported Merchandise}
\item LRA+ \textbf{Lending Risk Analysis}
\item LTR  \textbf{Loan Against Trust Receipt}
\item LTV \textbf{Loan To Value}
\item MDGs \textbf{Millennium Development Goals}
\item MICR  \textbf{Magnetic Ink Character Recognition}
\item MRA \textbf{Micro-Credit Regulatory Authority}
\item MTCN  \textbf{Money Transfer Control Number}
\item MTD  \textbf{Money Term Deposit Account}
\item NFCD \textbf{Non Resident Foreign Currency Deposit}
\item NIM \textbf{New Issue Market}
\item NPL  \textbf{Non Performing Loan}
\item NPL  \textbf{Non Performing Loan Account}
\item NPV  \textbf{Net Present Value}
\item OBC  \textbf{Outward Bill for Collection}
\item PDC+  \textbf{Product Development Council}
\item PES  \textbf{Price Earning Ratio}
\item POS  \textbf{Point Of Sale}
\item REER \textbf{Real Effecting Exchange Rate}
\item Repo \textbf{Re-purchase}
\item RJSC  \textbf{Registrar of Joint Stock Company}
\item RTGS \textbf{Real Time Gross Settlement}
\item RWA \textbf{Risk Weighted Assets}
\item SBS \textbf{Schedule Banking Statistics}
\item SEC \textbf{Securities and Exchange Commission}
\item SLR \textbf{Statutory Liquidity Ratio}
\item SMA \textbf{Special Mention Account}
\item SS  \textbf{Sub-Standard}
\item STL \textbf{Short Term Loan}
\item STR \textbf{Suspicious Transaction Report}
\item SWIFT \textbf{Society for World Wide Inter Financial Telecommunication}
\end{itemize}

\subsection{Bouncing of a cheque}
\label{sec:org772af0e}
When an account has insufficient funds the cheque is is not payable
and is returned by the bank with a reason “Exceeds arrangement” or
“funds insufficient”.
\subsection{Bank Rate}
\label{sec:orge64d517}
It is the rate of interest charged by a central bank to commercial
banks on the advances and the loans it extends.
\subsection{Cheque}
\label{sec:orgde2e222}
It is written by an individual to transfer amount between two accounts
of the same bank or a different bank and the money is withdrawn form
the account.
\subsection{Core Banking Solutions (CBS)}
\label{sec:org45ba0b8}
In this all the branches of the bank are connected together and the
customer can access his/her funds or transactions from any other
branch.
\subsection{CRR (Cash Reverse Ratio)}
\label{sec:orge302a7a}
the amount of funds that a bank keep with the RBI. If the percentage
of CRR increases then the amount with the bank comes down.
\subsection{Debit Card}
\label{sec:orga89626f}
It is a card issued by the bank so the customers can withdraw their
money from their account electronically.
\subsection{Demat Account}
\label{sec:orgda85b0c}
The way in which a bank keeps money in a deposit account in the same
way the Depository company converts share certificates into electronic
form and keep them in a Demat account.
\subsection{E-Banking}
\label{sec:orgefbbd68}
It is a type of banking in which we can conduct financial transactions
electronically. RTGS, Credit cards, Debit cards etc come under this
category.
\subsection{EFT – (Electronic Fund Transfer)}
\label{sec:org87d03ed}
In this we use Automatic teller machine, wire transfer and computers
to move funds between different accounts in different or same bank.
\subsection{Fiscal Deficit}
\label{sec:orgafcdfba}
It is the amount of Funds borrowed by the government to meet the
expenditures.
\subsection{Initial Public Offering (IPO)}
\label{sec:org3370903}
It is the time when a company makes the first offering of the shares
to the public.
\subsection{Leverage Ratio}
\label{sec:orgdcc7b65}
It is a financial ratio which gives us an idea or a measure of a
company’s ability to meet its financial losses.
\subsection{Liquidity}
\label{sec:orgb2d5d6b}
It is the ability of converting an investment quickly into cash with
no loss in value.
\subsection{Market Capitalization}
\label{sec:orgae18824}
The product of the share price and number of the company’s outstanding
ordinary shares.
\subsection{Mortgage}
\label{sec:org029140b}
It is a kind of security which one offers for taking an advance or
loan from someone.
\subsection{Mutual Fund}
\label{sec:org60bf598}
These are investment schemes. It pools money from various investors in
order to purchase securities.
\subsection{Pass Book}
\label{sec:org04f70dd}
It is a book where all the bank transactions are recorded.They are
mainly issued to Current or Savings Bank account holders.
\subsection{Repo Rate}
\label{sec:orgda28c42}
Commercial banks borrow funds by the RBI if there is any shortage in
the form of rupees. If this rate increases it becomes expensive to
borrow money from RBI and vice versa.
\subsection{Savings Bank Account}
\label{sec:org0329477}
It is account of nominal interest which can only be used for personal
purpose and which has some restrictions on withdrawal.
\subsection{SLR (Statutory Liquidity Ratio)}
\label{sec:org3a7c5ad}
It is amount that a commercial bank should have before giving credits
to its customers which should be either in the form of gold,money or
bonds.
\subsection{Teller}
\label{sec:org91d6486}
He/she is a staff member of the bank who cashes cheques, accepts
deposits and perform different banking services for the general mass.
\subsection{Universal Banking}
\label{sec:org4c8c2dc}
When financial institutions and banks undertake activities related to
banking like investment, issue of debit and credit card etc then it is
known as universal banking.
\subsection{Virtual Banking}
\label{sec:org62e757f}
Internet banking is sometimes known as virtual banking. It is called
so because it has no bricks and boundaries. It is controlled by the
world wide web.
\subsection{Wholesale Banking}
\label{sec:org66dbbf4}
It is similar to retail banking with a slight difference that it
mainly focuses on the financial needs of the institutional clients and
the industry.
\subsection{Zero Coupon Bond}
\label{sec:org9bdc324}
It is a bond that is sold at good discount as it has no coupon.

\section{বিদ্যুৎ ও জ্বালানি খাতের এ অবস্থা থেকে উত্তরণের জন্য নিম্নোক্ত ব্যবস্থাগুলো গ্রহণ করা প্রয়োজন}
\label{sec:org70b2956}

\begin{enumerate}
\item সরকারপ্রধানের বদলে একজন সৎ ও পেশাদার পূর্ণমন্ত্রীকে বিদ্যুৎ, জ্বালানি ও
খনিজ সম্পদ মন্ত্রণালয়ের দায়িত্ব প্রদান করতে হবে।
\item \textbf{বিদ্যুৎ ও জ্বালানির দ্রুতসরবরাহ বৃদ্ধি (বিশেষ বিধান) আইন ২০১০} অবিলম্বে
বাতিল করে এ খাতের সব সংগ্রহ উন্মুক্ত প্রতিযোগিতার মাধ্যমে করতে হবে।
এই আইনের মেয়াদ ২০২৬ সাল পর্যন্ত ।
\item জ্বালানি আমদানি হ্রাস করার জন্য নতুন গ্যাসক্ষেত্র আবিষ্কার ও নবায়নযোগ্য শক্তির
ব্যবহার বাড়াতে হবে।
\item অগ্রাধিকার ভিত্তিতে দেশীয় কয়লার মজুত ব্যবহার করার জন্য কয়লা নীতি চূড়ান্ত করতে হবে।
\item ভাসমান জাহাজের মাধ্যমে এলএনজি সংগ্রহে ভূমিতে এলএনজি প্ল্যান্ট স্থাপন করতে হবে।
\item রেন্টাল, কুইক রেন্টাল প্রকল্পগুলোর নবায়ন বন্ধ করতে হবে।
\item \textbf{আইপিপির} পরিবর্তে \textbf{এমপিপির} মাধ্যমে বেসরকারি খাতকে বিদ্যুৎ উৎপাদনে সম্পৃক্ত করতে হবে।
\item নিরপেক্ষ ক্যাপাসিটি টেস্টের মাধ্যমে বেসরকারি বিদ্যুৎকেন্দ্রগুলোর নির্ধারণ করতে হবে।
\item গ্রিড বিপর্যয় এড়ানোর জন্য \textbf{এজিসি(Automatic Generation Control)} সংগ্রহ করাসহ জাতীয় লোড
ডেসপাচ সেন্টারের আধুনিকায়ন করতে হবে।
\item বিদেশি বিদ্যুৎ সরবরাহকারী চুক্তির ন্যায্যতা পরীক্ষা করতে হবে।
\item প্রাথমিক জ্বালানি, বিদ্যুৎ উৎপাদন,সঞ্চালন ও বিতরণব্যবস্থার সমন্বিত উন্নয়নের জন্য মাস্টারপ্ল্যান
প্রণয়ন করে তা অনুসরণ করতে হবে।
\end{enumerate}

\section{বিদ্যুৎ সঙ্কট}
\label{sec:org567d7c8}

\begin{enumerate}
\item দেশের \textbf{১৩৩টি বিদ্যুৎ কেন্দ্রের মধ্যে ৬৩টি} কেন্দ্র এখন জ্বালানির অভাবে  বসে আছে।
\item \textbf{১৯ জুলাই থেকে}  সারাদেশে লোডশেডিং করে বিদ্যুতের রেশনিং শুরু হয়।
\item বিদ্যুৎ \textbf{প্রতিমন্ত্রী নসরুল হামিদ}
\item সারাদেশে বিদ্যুতের \textbf{চাহিদা ১৪ হাজার ৫০০ মেগাওয়াট।}
\item উৎপাদন করা যাচ্ছে \textbf{১৩ হাজার মেগাওয়াট।}
\item \textbf{এক হাজার ৫০০ মেগাওয়াটের} ঘাটতি ।
\item \textbf{গত ৪ অক্টোবর} বিদ্যুতের জাতীয় গ্রিড বিপর্যয়ের পর বিদ্যুৎ সংকট
ভয়াবহ আকার ধারণ করছে।
\item এর মধ্যে গ্যাসচালিত \textbf{৫৭টি বিদ্যুৎকেন্দ্রের উৎপাদন ক্ষমতা ১১ হাজার ১৭ মেগাওয়াট।} উৎপাদন হচ্ছে
মাত্র \textbf{পাঁচ হাজার মেগাওয়াট।}
\item ফার্নেস অয়েল চালিত \textbf{৫৬টি বিদ্যুৎকেন্দ্রের উৎপাদন ক্ষমতা পাঁচ হাজার ৫৪১ মেগাওয়াট}
\item \textbf{১১টি ডিজেল} কেন্দ্রের উৎপাদন ক্ষমতা \textbf{১ হাজার ৫১৫ মেগাওয়াট।}
\item সব মিলিয়ে জ্বালানি তেল থেকে বিদ্যুৎ উৎপাদন সক্ষমতা \textbf{সাত হাজার মেগাওয়াটের বেশি}
হলেও দিনে গড়ে উৎপাদন হচ্ছে \textbf{দুই হাজার ৫০০ মেগাওয়াট থেকে চার হজার মেগাওয়াট পর্যন্ত।}
\item \textbf{মার্চে} যদি \textbf{রামপালের( কয়লা বিদ্যুৎ কেন্দ্র)} \textbf{এক হাজার ২০০ মেগাওয়াট} এবং ভারতের \textbf{আদানির
৮০০ মেগাওয়াট} আসে তাহলে দুই হাজার মেগাওয়াট তখন পাওয়া যাবে।
এটা যদি যোগ হয় তাহলে পরিস্থিতির উন্নতি হবে। অন্যথায় নয়।
আর রূপপুরের বিদ্যুৎ পেতে আরো অনেক সময় লাগবে।”
\item তেলের বাকি আছে \textbf{১৬ হাজার কোটি টাকা,} এটা শোধ না করলে আমরা তেল পাব না।
\item \textbf{ড. তৌফিক-ই-ইলাহী চৌধুরী} বিদ্যুৎ, জ্বালানি ও খনিজ সম্পদ উপদেষ্টা
\item \textbf{২০২৬} সালের আগে চুক্তিবদ্ধ জ্বালানি ছাড়া নতুন চুক্তির ভিত্তিতে তেল-গ্যাস বেচবে না কাতার ।
\item বাংলাদেশ সাধারণত \textbf{৪০ শতাংশ এলএনজি গ্যাস আমদানি} করলেও চলতি বছর তা নামিয়ে \textbf{এনেছে ৩০ শতাংশে।}
\item সৌর বিদ্যুতায়নের মধ্যমানের কৌশলে \textbf{২০৪১ সালের মধ্যে বাংলাদেশে ২০ হাজার মেগাওয়াট সবুজ বিদ্যুৎ উৎপাদন সম্ভব।}
অন্যদিকে নদী অববাহিকা উন্নয়নের \textbf{৫ শতাংশ ভূমি, শিল্পাঞ্চলের রুফটপ ১৫ শতাংশ ভূমি প্রভৃতি নিয়ে একটি উচ্চপর্যায়ের
সৌর স্থাপনা মডেলে এই সক্ষমতা ৩০ হাজার মেগাওয়াটে পৌঁছানো সম্ভব।}
\item জ্বালানি আমদানির নির্ভরতা থেকে বের হতে চাইলে সরকারকে স্বচ্ছ ও দুর্নীতি মুক্ত বাস্তবায়নে \textbf{'ন্যাশনাল সোলার এনার্জি রোডম্যাপ ২০২১-৪১'কে}
আলোর পথে নিতে হবে।
\item টেকসই উন্নয়ন অভীষ্ট এসডিজি-৭ মতে, \textbf{স্বল্পোন্নত দেশগুলোর জন্য মোট ব্যবহৃত বিদ্যুতের ১২ শতাংশ নবায়নযোগ্য করার শর্ত আছে।}
বাংলাদেশ এলডিসি উত্তরণের শক্তিশালী প্রার্থী বলে ব্যবহৃত \textbf{বিদ্যুতের ১৭ শতাংশ নবায়নযোগ্য রাখার শর্ত মানা উচিত}, যা নিম্ন মধ্যআয়ের
দেশের জন্য প্রযোজ্য। বাংলাদেশ নবায়নযোগ্য বিদ্যুৎ উৎপাদনের বর্তমান অবস্থান \textbf{(মাত্র সাড়ে ৪ শতাংশ)} অগ্রহণযোগ্য।
\item এ পরিস্থিতিতে বাংলাদেশ তেল, গ্যাস ও খনিজ করপোরেশনের (পেট্রোবাংলা) তথ্য বলছে, \textbf{দেশে মাত্র ৯-১০ বছর ব্যবহারের জন্য গ্যাস মজুদ} আছে।
দেশের বার্ষিক গ্যাস ব্যবহার প্রায় ১ ট্রিলিয়ন ঘনফুট।
\item পেট্রোবাংলার তথ্য অনুযায়ী, দেশে এখন পর্যন্ত আবিষ্কৃত ২৮টি গ্যাসক্ষেত্রের \textbf{মোট প্রামাণিক মজুদের পরিমাণ ২৯ দশমিক ৯ ট্রিলিয়ন ঘনফুট,}
যা থেকে \textbf{চলতি মাস পর্যন্ত ১৯ দশমিক ১১ ট্রিলিয়ন ঘনফুট} গ্যাস উত্তোলন করা হয়েছে।
\item বর্তমানে দেশে বিদ্যমান মোট বিদ্যুতের ৪৩ \% গৃহস্থালিতে, ৪০.৮ \% কলকারখানায়, ১০.৯ \% বাণিজ্যিক খাতে,
২.৭ \% কৃষিখাতে এবং বাকি ২.৬ \% বিবিধ খাতে ব্যবহৃত হয়।
\end{enumerate}
\end{document}